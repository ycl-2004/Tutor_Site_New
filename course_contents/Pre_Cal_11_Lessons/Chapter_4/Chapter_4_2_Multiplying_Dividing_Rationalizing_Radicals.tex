\documentclass[aspectratio=169]{beamer}
\usepackage{amsmath}
\usepackage{amssymb}
\usepackage{graphicx}
\usepackage{tcolorbox}
\usepackage{booktabs}
\usepackage{colortbl}
\usepackage{xcolor}
\usepackage{tikz}
\usepackage[utf8]{inputenc}

% Custom colors
\definecolor{primary}{RGB}{41, 128, 185}
\definecolor{secondary}{RGB}{52, 152, 219}
\definecolor{accent}{RGB}{231, 76, 60}
\definecolor{lightgray}{RGB}{236, 240, 241}

% Theme customization
\usetheme{Madrid}
\usecolortheme{whale}
\setbeamercolor{structure}{fg=primary}
\setbeamercolor{background canvas}{bg=white}
\setbeamercolor{normal text}{fg=black}

\title{Chapter 4.2: Multiplying, Dividing, and Rationalizing Radicals}
\subtitle{Pre-Calculus 11 - Lesson 2}
\author{Created by Yi-Chen Lin}
\date{\today}

\begin{document}

\begin{frame}
\titlepage
\end{frame}

% Overview
\begin{frame}{Overview: Multiplying, Dividing, and Rationalizing Radicals}
\begin{tcolorbox}[colback=lightgray,colframe=primary,title=Key Concepts]
\footnotesize
\begin{itemize}
  \item Multiplying radicals: outside × outside, inside × inside; same radicals multiply to whole numbers
  \item Dividing radicals: divide outside and inside separately, never leave a radical in the denominator
  \item Rationalizing radicals: multiply numerator and denominator by the radical (monomial) or conjugate (binomial) to eliminate radicals in the denominator
\end{itemize}
\end{tcolorbox}
\end{frame}

% I) Multiplying Radicals
\begin{frame}{I) Multiplying Radicals}
\begin{tcolorbox}[colback=lightgray,colframe=primary,title=Rules]
\footnotesize
\begin{itemize}
  \item $\sqrt{a} \times \sqrt{a} = a$
  \item $\sqrt{a} \times \sqrt{b} = \sqrt{ab}$
  \item $(a\sqrt{b}) \times (c\sqrt{d}) = ac\sqrt{bd}$
  \item Numbers outside stay outside, numbers inside stay inside
\end{itemize}
\end{tcolorbox}
\end{frame}

\begin{frame}{Multiplying Radicals - Practice}
\begin{tcolorbox}[colback=lightgray,colframe=accent,title=Practice Problems]
\footnotesize
\begin{columns}[T]
\column{0.5\textwidth}
\begin{enumerate}[label=1\textbf{a})]
  \item $2\sqrt{3} \times 6\sqrt{2}$
\end{enumerate}
\begin{enumerate}[label=1\textbf{b})]
  \item $\sqrt{5} \times \sqrt{20}$
\end{enumerate}
\begin{enumerate}[label=1\textbf{c})]
  \item $3\sqrt{7} \times 8\sqrt{5}$
\end{enumerate}
\column{0.5\textwidth}
\begin{enumerate}[label=1\textbf{d})]
  \item $\sqrt[3]{4} \times \sqrt[3]{6}$
\end{enumerate}
\begin{enumerate}[label=1\textbf{e})]
  \item $2\sqrt{10} \times 7\sqrt{90}$
\end{enumerate}
\begin{enumerate}[label=1\textbf{f})]
  \item $\sqrt{8x} \times \sqrt{2x^3}$
\end{enumerate}
\end{columns}
\end{tcolorbox}
\end{frame}

\begin{frame}{Multiplying Radicals - Solutions}
\begin{tcolorbox}[colback=lightgray,colframe=primary,title=Solutions]
\footnotesize
\begin{columns}[T]
\column{0.5\textwidth}
\begin{enumerate}[label=1\textbf{a})]
  \item $2\sqrt{3} \times 6\sqrt{2} = 12\sqrt{6}$
\end{enumerate}
\begin{enumerate}[label=1\textbf{b})]
  \item $\sqrt{5} \times \sqrt{20} = \sqrt{100} = 10$
\end{enumerate}
\begin{enumerate}[label=1\textbf{c})]
  \item $3\sqrt{7} \times 8\sqrt{5} = 24\sqrt{35}$
\end{enumerate}
\column{0.5\textwidth}
\begin{enumerate}[label=1\textbf{d})]
  \item $\sqrt[3]{4} \times \sqrt[3]{6} = \sqrt[3]{24}$
\end{enumerate}
\begin{enumerate}[label=1\textbf{e})]
  \item $2\sqrt{10} \times 7\sqrt{90} = 14\sqrt{900} = 14 \times 30 = 420$
\end{enumerate}
\begin{enumerate}[label=1\textbf{f})]
  \item $\sqrt{8x} \times \sqrt{2x^3} = \sqrt{16x^4} = 4x^2$
\end{enumerate}
\end{columns}
\end{tcolorbox}
\end{frame}

% II) Dividing Radicals
\begin{frame}{II) Dividing Radicals}
\begin{tcolorbox}[colback=lightgray,colframe=primary,title=Rules]
\footnotesize
\begin{itemize}
  \item $\frac{a\sqrt{b}}{c\sqrt{d}} = \frac{a}{c} \times \sqrt{\frac{b}{d}}$
  \item Simplify outside and inside separately
  \item Never leave a radical in the denominator
\end{itemize}
\end{tcolorbox}
\end{frame}

\begin{frame}{Dividing Radicals - Practice}
\begin{tcolorbox}[colback=lightgray,colframe=accent,title=Practice Problems]
\footnotesize
\begin{columns}[T]
\column{0.5\textwidth}
\begin{enumerate}[label=2\textbf{a})]
  \item $\frac{6\sqrt{50}}{3\sqrt{2}}$
\end{enumerate}
\begin{enumerate}[label=2\textbf{b})]
  \item $\frac{8\sqrt{18}}{4\sqrt{3}}$
\end{enumerate}
\begin{enumerate}[label=2\textbf{c})]
  \item $\frac{5\sqrt{12x}}{10\sqrt{3x}}$
\end{enumerate}
\column{0.5\textwidth}
\begin{enumerate}[label=2\textbf{d})]
  \item $\frac{\sqrt[3]{54}}{\sqrt[3]{2}}$
\end{enumerate}
\begin{enumerate}[label=2\textbf{e})]
  \item $\frac{7\sqrt{45}}{14\sqrt{5}}$
\end{enumerate}
\begin{enumerate}[label=2\textbf{f})]
  \item $\frac{\sqrt{32y^3}}{4\sqrt{2y}}$
\end{enumerate}
\end{columns}
\end{tcolorbox}
\end{frame}

\begin{frame}{Dividing Radicals - Solutions}
\begin{tcolorbox}[colback=lightgray,colframe=primary,title=Solutions]
\footnotesize
\begin{columns}[T]
\column{0.5\textwidth}
\begin{enumerate}[label=2\textbf{a})]
  \item $\frac{6\sqrt{50}}{3\sqrt{2}} = 2\sqrt{25} = 10$
\end{enumerate}
\begin{enumerate}[label=2\textbf{b})]
  \item $\frac{8\sqrt{18}}{4\sqrt{3}} = 2\sqrt{6}$
\end{enumerate}
\begin{enumerate}[label=2\textbf{c})]
  \item $\frac{5\sqrt{12x}}{10\sqrt{3x}} = \frac{1}{2}\sqrt{4} = 1$
\end{enumerate}
\column{0.5\textwidth}
\begin{enumerate}[label=2\textbf{d})]
  \item $\frac{\sqrt[3]{54}}{\sqrt[3]{2}} = \sqrt[3]{27} = 3$
\end{enumerate}
\begin{enumerate}[label=2\textbf{e})]
  \item $\frac{7\sqrt{45}}{14\sqrt{5}} = \frac{1}{2}\sqrt{9} = \frac{3}{2}$
\end{enumerate}
\begin{enumerate}[label=2\textbf{f})]
  \item $\frac{\sqrt{32y^3}}{4\sqrt{2y}} = \frac{1}{4}\sqrt{16y^2} = \frac{1}{4} \times 4y = y$
\end{enumerate}
\end{columns}
\end{tcolorbox}
\end{frame}

% III) Rationalizing Radicals
\begin{frame}{III) Rationalizing Radicals}
\begin{tcolorbox}[colback=lightgray,colframe=primary,title=Rules]
\footnotesize
\begin{itemize}
  \item If denominator is a monomial: multiply top and bottom by the radical in the denominator
  \item If denominator is a binomial: multiply top and bottom by the conjugate of the denominator
  \item This eliminates the radical in the denominator
\end{itemize}
\end{tcolorbox}
\end{frame}

\begin{frame}{Rationalizing Radicals (Monomial) - Practice}
\begin{tcolorbox}[colback=lightgray,colframe=accent,title=Practice Problems]
\footnotesize
\begin{columns}[T]
\column{0.5\textwidth}
\begin{enumerate}[label=3\textbf{a})]
  \item $\frac{1}{\sqrt{3}}$
\end{enumerate}
\begin{enumerate}[label=3\textbf{b})]
  \item $\frac{5}{2\sqrt{5}}$
\end{enumerate}
\begin{enumerate}[label=3\textbf{c})]
  \item $\frac{7}{\sqrt{2x}}$
\end{enumerate}
\column{0.5\textwidth}
\begin{enumerate}[label=3\textbf{d})]
  \item $\frac{3}{\sqrt[3]{4}}$
\end{enumerate}
\begin{enumerate}[label=3\textbf{e})]
  \item $\frac{2x}{\sqrt{8x}}$
\end{enumerate}
\begin{enumerate}[label=3\textbf{f})]
  \item $\frac{4}{2\sqrt{y}}$
\end{enumerate}
\end{columns}
\end{tcolorbox}
\end{frame}

\begin{frame}{Rationalizing Radicals (Monomial) - Solutions}
\begin{tcolorbox}[colback=lightgray,colframe=primary,title=Solutions]
\footnotesize
\begin{columns}[T]
\column{0.5\textwidth}
\begin{enumerate}[label=3\textbf{a})]
  \item $\frac{1}{\sqrt{3}} = \frac{\sqrt{3}}{3}$
\end{enumerate}
\begin{enumerate}[label=3\textbf{b})]
  \item $\frac{5}{2\sqrt{5}} = \frac{5\sqrt{5}}{2\times 5} = \frac{\sqrt{5}}{2}$
\end{enumerate}
\begin{enumerate}[label=3\textbf{c})]
  \item $\frac{7}{\sqrt{2x}} = \frac{7\sqrt{2x}}{2x}$
\end{enumerate}
\column{0.5\textwidth}
\begin{enumerate}[label=3\textbf{d})]
  \item $\frac{3}{\sqrt[3]{4}} = \frac{3\sqrt[3]{16}}{4}$
\end{enumerate}
\begin{enumerate}[label=3\textbf{e})]
  \item $\frac{2x}{\sqrt{8x}} = \frac{2x\sqrt{8x}}{8x} = \frac{\sqrt{8x}}{4}$
\end{enumerate}
\begin{enumerate}[label=3\textbf{f})]
  \item $\frac{4}{2\sqrt{y}} = \frac{2}{\sqrt{y}} = \frac{2\sqrt{y}}{y}$
\end{enumerate}
\end{columns}
\end{tcolorbox}
\end{frame}

\begin{frame}{Rationalizing Radicals (Binomial) - Practice}
\begin{tcolorbox}[colback=lightgray,colframe=accent,title=Practice Problems]
\footnotesize
\begin{columns}[T]
\column{0.5\textwidth}
\begin{enumerate}[label=4\textbf{a})]
  \item $\frac{1}{1+\sqrt{2}}$
\end{enumerate}
\begin{enumerate}[label=4\textbf{b})]
  \item $\frac{3}{2-\sqrt{5}}$
\end{enumerate}
\begin{enumerate}[label=4\textbf{c})]
  \item $\frac{5}{\sqrt{3}-1}$
\end{enumerate}
\column{0.5\textwidth}
\begin{enumerate}[label=4\textbf{d})]
  \item $\frac{2}{1-\sqrt{7}}$
\end{enumerate}
\begin{enumerate}[label=4\textbf{e})]
  \item $\frac{4}{2+\sqrt{y}}$
\end{enumerate}
\begin{enumerate}[label=4\textbf{f})]
  \item $\frac{6}{\sqrt{5}+\sqrt{2}}$
\end{enumerate}
\end{columns}
\end{tcolorbox}
\end{frame}

\begin{frame}{Rationalizing Radicals (Binomial) - Solutions}
\begin{tcolorbox}[colback=lightgray,colframe=primary,title=Solutions]
\footnotesize
\begin{columns}[T]
\column{0.5\textwidth}
\begin{enumerate}[label=4\textbf{a})]
  \item $\frac{1}{1+\sqrt{2}} = \frac{1\times(1-\sqrt{2})}{(1+\sqrt{2})(1-\sqrt{2})} = \frac{1-\sqrt{2}}{1-2} = \frac{1-\sqrt{2}}{-1} = \sqrt{2}-1$
\end{enumerate}
\begin{enumerate}[label=4\textbf{b})]
  \item $\frac{3}{2-\sqrt{5}} = \frac{3(2+\sqrt{5})}{(2-\sqrt{5})(2+\sqrt{5})} = \frac{6+3\sqrt{5}}{4-5} = -6-3\sqrt{5}$
\end{enumerate}
\begin{enumerate}[label=4\textbf{c})]
  \item $\frac{5}{\sqrt{3}-1} = \frac{5(\sqrt{3}+1)}{(\sqrt{3}-1)(\sqrt{3}+1)} = \frac{5\sqrt{3}+5}{3-1} = \frac{5\sqrt{3}+5}{2}$
\end{enumerate}
\column{0.5\textwidth}
\begin{enumerate}[label=4\textbf{d})]
  \item $\frac{2}{1-\sqrt{7}} = \frac{2(1+\sqrt{7})}{(1-\sqrt{7})(1+\sqrt{7})} = \frac{2+2\sqrt{7}}{1-7} = -\frac{2+2\sqrt{7}}{6}$
\end{enumerate}
\begin{enumerate}[label=4\textbf{e})]
  \item $\frac{4}{2+\sqrt{y}} = \frac{4(2-\sqrt{y})}{(2+\sqrt{y})(2-\sqrt{y})} = \frac{8-4\sqrt{y}}{4-y}$
\end{enumerate}
\begin{enumerate}[label=4\textbf{f})]
  \item $\frac{6}{\sqrt{5}+\sqrt{2}} = \frac{6(\sqrt{5}-\sqrt{2})}{(\sqrt{5}+\sqrt{2})(\sqrt{5}-\sqrt{2})} = \frac{6\sqrt{5}-6\sqrt{2}}{5-2} = 2\sqrt{5}-2\sqrt{2}$
\end{enumerate}
\end{columns}
\end{tcolorbox}
\end{frame}

% Rationalize (cube roots and binomials, new values)
\begin{frame}{Rationalize:}
\begin{tcolorbox}[colback=lightgray,colframe=accent,title=Practice Problems]
\footnotesize
\begin{columns}[T]
\column{0.5\textwidth}
\begin{enumerate}[label=5\textbf{a})]
  \item $\dfrac{2+\sqrt[3]{5}}{\sqrt[3]{16}}$
\end{enumerate}
\column{0.5\textwidth}
\begin{enumerate}[label=5\textbf{b})]
  \item $\dfrac{7}{\sqrt[3]{9}+\sqrt[3]{12}}$
\end{enumerate}
\end{columns}
\end{tcolorbox}
\end{frame}

\begin{frame}{Rationalize - Solutions}
\begin{tcolorbox}[colback=lightgray,colframe=primary,title=Solutions]
\footnotesize
\begin{enumerate}[label=5\textbf{a})]
  \item $\dfrac{2+\sqrt[3]{5}}{\sqrt[3]{16}} = \dfrac{2+\sqrt[3]{5}}{2} = 1+\dfrac{1}{2}\sqrt[3]{5}$
\end{enumerate}
\begin{enumerate}[label=5\textbf{b})]
  \item $\dfrac{7}{\sqrt[3]{9}+\sqrt[3]{12}}$ (Multiply numerator and denominator by $\sqrt[3]{9^2}-\sqrt[3]{9}\sqrt[3]{12}+\sqrt[3]{12^2}$ to rationalize, or leave as a challenge for students)
\end{enumerate}
\end{tcolorbox}
\end{frame}

% FOIL: Expand the following (radical binomials, new values)
\begin{frame}{FOIL: Expand the Following}
\begin{tcolorbox}[colback=lightgray,colframe=accent,title=Practice Problems]
\footnotesize
\begin{columns}[T]
\column{0.5\textwidth}
\begin{enumerate}[label=6\textbf{a})]
  \item $\left(2+\sqrt{3}\right)\left(5-\sqrt{3}\right)$
\end{enumerate}
\begin{enumerate}[label=6\textbf{b})]
  \item $\left(4+2\sqrt{2}\right)\left(4-2\sqrt{2}\right)$
\end{enumerate}
\column{0.5\textwidth}
\begin{enumerate}[label=6\textbf{c})]
  \item $\left(6+\sqrt{7}\right)\left(2-\sqrt{7}\right)$
\end{enumerate}
\begin{enumerate}[label=6\textbf{d})]
  \item $\left(8-\sqrt{5}\right)\left(8+\sqrt{5}\right)$
\end{enumerate}
\end{columns}
\end{tcolorbox}
\end{frame}

\begin{frame}{FOIL: Expand the Following - Solutions}
\begin{tcolorbox}[colback=lightgray,colframe=primary,title=Solutions]
\footnotesize
\begin{enumerate}[label=6\textbf{a})]
  \item $\left(2+\sqrt{3}\right)\left(5-\sqrt{3}\right) = 2\times5 - 2\times\sqrt{3} + 5\times\sqrt{3} - (\sqrt{3})^2 = 10 + 3\sqrt{3} - 3 = 7 + 3\sqrt{3}$
\end{enumerate}
\begin{enumerate}[label=6\textbf{b})]
  \item $\left(4+2\sqrt{2}\right)\left(4-2\sqrt{2}\right) = 16 - 8\sqrt{2} + 8\sqrt{2} - 8 = 16 - 8 = 8$
\end{enumerate}
\begin{enumerate}[label=6\textbf{c})]
  \item $\left(6+\sqrt{7}\right)\left(2-\sqrt{7}\right) = 12 - 6\sqrt{7} + 2\sqrt{7} - 7 = 12 - 7 - 4\sqrt{7} = 5 - 4\sqrt{7}$
\end{enumerate}
\begin{enumerate}[label=6\textbf{d})]
  \item $\left(8-\sqrt{5}\right)\left(8+\sqrt{5}\right) = 64 + 8\sqrt{5} - 8\sqrt{5} - 5 = 64 - 5 = 59$
\end{enumerate}
\end{tcolorbox}
\end{frame}

\end{document} 