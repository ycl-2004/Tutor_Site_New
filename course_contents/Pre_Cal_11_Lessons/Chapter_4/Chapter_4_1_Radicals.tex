\documentclass[aspectratio=169]{beamer}
\usepackage{amsmath}
\usepackage{amssymb}
\usepackage{graphicx}
\usepackage{tcolorbox}
\usepackage{booktabs}
\usepackage{colortbl}
\usepackage{xcolor}
\usepackage{tikz}
\usepackage[utf8]{inputenc}

% Custom colors
\definecolor{primary}{RGB}{41, 128, 185}
\definecolor{secondary}{RGB}{52, 152, 219}
\definecolor{accent}{RGB}{231, 76, 60}
\definecolor{lightgray}{RGB}{236, 240, 241}

% Theme customization
\usetheme{Madrid}
\usecolortheme{whale}
\setbeamercolor{structure}{fg=primary}
\setbeamercolor{background canvas}{bg=white}
\setbeamercolor{normal text}{fg=black}

\title{Chapter 4.1: Radicals}
\subtitle{Pre-Calculus 11}
\author{Created by Yi-Chen Lin}
\date{\today}

\begin{document}

\begin{frame}
\titlepage
\end{frame}

\begin{frame}{Review: Basic Radicals}
\begin{tcolorbox}[colback=lightgray,colframe=primary,title=Evaluate each of the following without a calculator]
\footnotesize
\begin{columns}[T]
\column{0.25\textwidth}
\begin{enumerate}
  \item $\sqrt{4}$
  \item $\sqrt{25}$
  \item $\sqrt{121}$
  \item $\sqrt{64}$
\end{enumerate}
\column{0.25\textwidth}
\begin{enumerate}
  \item $\sqrt{49}$
  \item $\sqrt[3]{8}$
  \item $\sqrt[3]{27}$
  \item $\sqrt[3]{64}$
\end{enumerate}
\column{0.25\textwidth}
\begin{enumerate}
  \item $\sqrt[4]{16}$
  \item $\sqrt{2} \times \sqrt{25}$
  \item $\sqrt{242}$
  \item $\sqrt{192}$
\end{enumerate}
\column{0.25\textwidth}
\begin{enumerate}
  \item $\sqrt{98}$
  \item $\sqrt[3]{24}$
  \item $\sqrt[3]{54}$
  \item $\sqrt[3]{320}$
  \item $\sqrt[4]{48}$
\end{enumerate}
\end{columns}
\end{tcolorbox}
\end{frame}

\begin{frame}{Review: Basic Radicals - Answers}
\begin{tcolorbox}[colback=lightgray,colframe=accent,title=Solutions]
\footnotesize
\begin{columns}[T]
\column{0.25\textwidth}
\begin{enumerate}
  \item $\sqrt{4} = 2$
  \item $\sqrt{25} = 5$
  \item $\sqrt{121} = 11$
  \item $\sqrt{64} = 8$
\end{enumerate}
\column{0.25\textwidth}
\begin{enumerate}
  \item $\sqrt{49} = 7$
  \item $\sqrt[3]{8} = 2$
  \item $\sqrt[3]{27} = 3$
  \item $\sqrt[3]{64} = 4$
\end{enumerate}
\column{0.25\textwidth}
\begin{enumerate}
  \item $\sqrt[4]{16} = 2$
  \item $\sqrt{2} \times \sqrt{25} = 5\sqrt{2}$
  \item $\sqrt{242} = 11\sqrt{2}$
  \item $\sqrt{192} = 8\sqrt{3}$
\end{enumerate}
\column{0.25\textwidth}
\begin{enumerate}
  \item $\sqrt{98} = 7\sqrt{2}$
  \item $\sqrt[3]{24} = 2\sqrt[3]{3}$
  \item $\sqrt[3]{54} = 3\sqrt[3]{2}$
  \item $\sqrt[3]{320} = 4\sqrt[3]{5}$
  \item $\sqrt[4]{48} = 2\sqrt[4]{3}$
\end{enumerate}
\end{columns}
\end{tcolorbox}
\end{frame}

\begin{frame}{Fractional Exponents and Nth Roots}
\begin{tcolorbox}[colback=lightgray,colframe=primary,title=Key Concepts]
\footnotesize
\begin{columns}
\column{0.6\textwidth}
\begin{itemize}
  \item The square root of a number is equivalent to having an exponent of $\frac{1}{2}$
  \item The cube root of a number is equivalent to having an exponent of $\frac{1}{3}$
  \item The fourth root of a number is equivalent to having an exponent of $\frac{1}{4}$
  \item The nth root of a number is equivalent to having an exponent of $\frac{1}{n}$
\end{itemize}
\column{0.4\textwidth}
\begin{align*}
\sqrt{x} &= x^{\frac{1}{2}} \\
\sqrt[3]{x} &= x^{\frac{1}{3}} \\
\sqrt[4]{x} &= x^{\frac{1}{4}} \\
\sqrt[n]{x} &= x^{\frac{1}{n}}
\end{align*}
\end{columns}
\end{tcolorbox}
\end{frame}

\begin{frame}{Simplifying Radicals}
\begin{tcolorbox}[colback=lightgray,colframe=accent,title=Practice Problems]
\footnotesize
\begin{columns}[T]
\column{0.5\textwidth}
\begin{enumerate}
  \item $\sqrt{5} \times \sqrt{5}$
  \item $\sqrt[3]{8} \times \sqrt[3]{8} \times \sqrt[3]{8}$
\end{enumerate}
\column{0.5\textwidth}
\begin{enumerate}
  \item $\sqrt{150}$
  \item $\sqrt[3]{108}$
\end{enumerate}
\end{columns}
\end{tcolorbox}
\end{frame}

\begin{frame}{Simplifying Radicals - Answers}
\begin{tcolorbox}[colback=lightgray,colframe=primary,title=Solutions]
\footnotesize
\begin{columns}[T]
\column{0.5\textwidth}
\begin{enumerate}
  \item $\sqrt{5} \times \sqrt{5} = 5$
  \item $\sqrt[3]{8} \times \sqrt[3]{8} \times \sqrt[3]{8} = 8$
\end{enumerate}
\column{0.5\textwidth}
\begin{enumerate}
  \item $\sqrt{150} = \sqrt{25 \times 6} = 5\sqrt{6}$
  \item $\sqrt[3]{108} = \sqrt[3]{27 \times 4} = 3\sqrt[3]{4}$
\end{enumerate}
\end{columns}
\end{tcolorbox}
\end{frame}

\begin{frame}{Adding \& Subtracting Radicals}
\begin{tcolorbox}[colback=lightgray,colframe=accent,title=Practice Problems]
\footnotesize
\begin{columns}[T]
\column{0.5\textwidth}
\begin{enumerate}
  \item $3\sqrt{2} + 5\sqrt{2} - 7\sqrt{2} + 11\sqrt{2}$
  \item $7\sqrt{3} + 9\sqrt{3} - 12\sqrt{3} - 15\sqrt{3}$
\end{enumerate}
\column{0.5\textwidth}
\begin{enumerate}
  \item $3\sqrt{5} - \sqrt{20} + \sqrt{45} + 12\sqrt{5} - \sqrt{180}$
  \item $3\sqrt{12} - 2\sqrt{45} + \sqrt{75} - 6\sqrt{80} + \sqrt{243}$
\end{enumerate}
\end{columns}
\end{tcolorbox}
\end{frame}

\begin{frame}{Adding \& Subtracting Radicals - Answers}
\begin{tcolorbox}[colback=lightgray,colframe=primary,title=Solutions]
\footnotesize
\begin{columns}[T]
\column{0.5\textwidth}
\begin{enumerate}
  \item $3\sqrt{2} + 5\sqrt{2} - 7\sqrt{2} + 11\sqrt{2} = 12\sqrt{2}$
  \item $7\sqrt{3} + 9\sqrt{3} - 12\sqrt{3} - 15\sqrt{3} = -11\sqrt{3}$
\end{enumerate}
\column{0.5\textwidth}
\begin{enumerate}
  \item $3\sqrt{5} - \sqrt{20} + \sqrt{45} + 12\sqrt{5} - \sqrt{180} = 10\sqrt{5}$
  \item $3\sqrt{12} - 2\sqrt{45} + \sqrt{75} - 6\sqrt{80} + \sqrt{243} = 20\sqrt{3} - 30\sqrt{5}$
\end{enumerate}
\end{columns}
\end{tcolorbox}
\end{frame}

\begin{frame}{Different Types of Roots - Practice}
\begin{tcolorbox}[colback=lightgray,colframe=accent,title=Practice Problems]
\footnotesize
\begin{columns}[T]
\column{0.5\textwidth}
\begin{enumerate}
  \item $\sqrt{72}$
  \item $\sqrt[3]{81}$
  \item $\sqrt[4]{48}$
  \item $\sqrt[5]{96}$
\end{enumerate}
\column{0.5\textwidth}
\begin{enumerate}
  \item $\sqrt{200}$
  \item $\sqrt[3]{128}$
  \item $\sqrt[4]{162}$
  \item $\sqrt[5]{243}$
\end{enumerate}
\end{columns}
\end{tcolorbox}
\end{frame}

\begin{frame}{Different Types of Roots - Answers}
\begin{tcolorbox}[colback=lightgray,colframe=primary,title=Solutions]
\footnotesize
\begin{columns}[T]
\column{0.5\textwidth}
\begin{enumerate}
  \item $\sqrt{72} = 6\sqrt{2}$
  \item $\sqrt[3]{81} = 3\sqrt[3]{3}$
  \item $\sqrt[4]{48} = 2\sqrt[4]{3}$
  \item $\sqrt[5]{96} = 2\sqrt[5]{3}$
\end{enumerate}
\column{0.5\textwidth}
\begin{enumerate}
  \item $\sqrt{200} = 10\sqrt{2}$
  \item $\sqrt[3]{128} = 4\sqrt[3]{2}$
  \item $\sqrt[4]{162} = 3\sqrt[4]{2}$
  \item $\sqrt[5]{243} = 3$
\end{enumerate}
\end{columns}
\end{tcolorbox}
\end{frame}

\begin{frame}{Variable Radicals - Practice}
\begin{tcolorbox}[colback=lightgray,colframe=accent,title=Practice Problems]
\footnotesize
\begin{columns}[T]
\column{0.5\textwidth}
\begin{enumerate}
  \item $\sqrt{x^2}$
  \item $\sqrt[3]{x^3}$
  \item $\sqrt{x^4}$
  \item $\sqrt[3]{x^6}$
\end{enumerate}
\column{0.5\textwidth}
\begin{enumerate}
  \item $\sqrt{x^2y^4}$
  \item $\sqrt[3]{x^3y^6}$
  \item $\sqrt{x^4y^2}$
  \item $\sqrt[3]{x^6y^9}$
\end{enumerate}
\end{columns}
\end{tcolorbox}
\end{frame}

\begin{frame}{Variable Radicals - Answers}
\begin{tcolorbox}[colback=lightgray,colframe=primary,title=Solutions]
\footnotesize
\begin{columns}[T]
\column{0.5\textwidth}
\begin{enumerate}
  \item $\sqrt{x^2} = |x|$
  \item $\sqrt[3]{x^3} = x$
  \item $\sqrt{x^4} = x^2$
  \item $\sqrt[3]{x^6} = x^2$
\end{enumerate}
\column{0.5\textwidth}
\begin{enumerate}
  \item $\sqrt{x^2y^4} = |x|y^2$
  \item $\sqrt[3]{x^3y^6} = xy^2$
  \item $\sqrt{x^4y^2} = x^2|y|$
  \item $\sqrt[3]{x^6y^9} = x^2y^3$
\end{enumerate}
\end{columns}
\end{tcolorbox}
\end{frame}

\begin{frame}{Mixed to Entire Radicals - Practice}
\begin{tcolorbox}[colback=lightgray,colframe=accent,title=Practice Problems]
\footnotesize
\begin{columns}[T]
\column{0.5\textwidth}
\begin{enumerate}
  \item $2\sqrt{3}$
  \item $3\sqrt[3]{2}$
  \item $4\sqrt{5}$
  \item $2\sqrt[3]{3}$
\end{enumerate}
\column{0.5\textwidth}
\begin{enumerate}
  \item $3\sqrt{2}$
  \item $2\sqrt[3]{4}$
  \item $5\sqrt{3}$
  \item $3\sqrt[3]{2}$
\end{enumerate}
\end{columns}
\end{tcolorbox}
\end{frame}

\begin{frame}{Mixed to Entire Radicals - Answers}
\begin{tcolorbox}[colback=lightgray,colframe=primary,title=Solutions]
\footnotesize
\begin{columns}[T]
\column{0.5\textwidth}
\begin{enumerate}
  \item $2\sqrt{3} = \sqrt{12}$
  \item $3\sqrt[3]{2} = \sqrt[3]{54}$
  \item $4\sqrt{5} = \sqrt{80}$
  \item $2\sqrt[3]{3} = \sqrt[3]{24}$
\end{enumerate}
\column{0.5\textwidth}
\begin{enumerate}
  \item $3\sqrt{2} = \sqrt{18}$
  \item $2\sqrt[3]{4} = \sqrt[3]{32}$
  \item $5\sqrt{3} = \sqrt{75}$
  \item $3\sqrt[3]{2} = \sqrt[3]{54}$
\end{enumerate}
\end{columns}
\end{tcolorbox}
\end{frame}

\begin{frame}{Entire to Mixed Radicals - Practice}
\begin{tcolorbox}[colback=lightgray,colframe=accent,title=Practice Problems]
\footnotesize
\begin{columns}[T]
\column{0.5\textwidth}
\begin{enumerate}
  \item $\sqrt{12}$
  \item $\sqrt[3]{54}$
  \item $\sqrt{80}$
  \item $\sqrt[3]{24}$
\end{enumerate}
\column{0.5\textwidth}
\begin{enumerate}
  \item $\sqrt{18}$
  \item $\sqrt[3]{32}$
  \item $\sqrt{75}$
  \item $\sqrt[3]{54}$
\end{enumerate}
\end{columns}
\end{tcolorbox}
\end{frame}

\begin{frame}{Entire to Mixed Radicals - Answers}
\begin{tcolorbox}[colback=lightgray,colframe=primary,title=Solutions]
\footnotesize
\begin{columns}[T]
\column{0.5\textwidth}
\begin{enumerate}
  \item $\sqrt{12} = 2\sqrt{3}$
  \item $\sqrt[3]{54} = 3\sqrt[3]{2}$
  \item $\sqrt{80} = 4\sqrt{5}$
  \item $\sqrt[3]{24} = 2\sqrt[3]{3}$
\end{enumerate}
\column{0.5\textwidth}
\begin{enumerate}
  \item $\sqrt{18} = 3\sqrt{2}$
  \item $\sqrt[3]{32} = 2\sqrt[3]{4}$
  \item $\sqrt{75} = 5\sqrt{3}$
  \item $\sqrt[3]{54} = 3\sqrt[3]{2}$
\end{enumerate}
\end{columns}
\end{tcolorbox}
\end{frame}

% Practice: Simplify each entire radical to a mixed radical
\begin{frame}{Simplify Each Entire Radical to a Mixed Radical}
\begin{tcolorbox}[colback=lightgray,colframe=accent,title=Practice Problems]
\footnotesize
\begin{enumerate}
  \item $\sqrt[3]{a^5 b^7}$
  \item $\sqrt[4]{x^9 y^5}$
  \item $\sqrt[5]{m^8 n^{12}}$
  \item $\sqrt[3]{-b^6}$
\end{enumerate}
\end{tcolorbox}
\end{frame}

% Split solutions for 'Simplify Each Entire Radical to a Mixed Radical'
\begin{frame}{Simplify Each Entire Radical to a Mixed Radical - Solutions (1/2)}
\begin{tcolorbox}[colback=lightgray,colframe=primary,title=Solutions]
\footnotesize
\begin{enumerate}
  \item $\sqrt[3]{a^5 b^7}$
  \begin{align*}
    &= \sqrt[3]{a^3 \cdot a^2 \cdot b^6 \cdot b} \\
    &= \textcolor{accent}{a b^2} \times \sqrt[3]{a^2 b}
  \end{align*}
  \item $\sqrt[4]{x^9 y^5}$
  \begin{align*}
    &= \sqrt[4]{x^8 \cdot x \cdot y^4 \cdot y} \\
    &= \textcolor{accent}{x^2 y} \times \sqrt[4]{x y}
  \end{align*}
\end{enumerate}
\end{tcolorbox}
\end{frame}

\begin{frame}{Simplify Each Entire Radical to a Mixed Radical - Solutions (2/2)}
\begin{tcolorbox}[colback=lightgray,colframe=primary,title=Solutions]
\footnotesize
\begin{enumerate}
  \setcounter{enumi}{2}
  \item $\sqrt[5]{m^8 n^{12}}$
  \begin{align*}
    &= \sqrt[5]{m^5 \cdot m^3 \cdot n^{10} \cdot n^2} \\
    &= \textcolor{accent}{m n^2} \times \sqrt[5]{m^3 n^2}
  \end{align*}
  \item $\sqrt[3]{-b^6}$
  \begin{align*}
    &= \sqrt[3]{(-1) b^6} \\
    &= \textcolor{accent}{-1 \cdot b^2} \\
    &= \textcolor{accent}{-b^2}
  \end{align*}
\end{enumerate}
\end{tcolorbox}
\end{frame}

% Practice: Simplify each radical to a mixed radical (new values)
\begin{frame}{Practice: Simplify each radical to a mixed radical}
\begin{tcolorbox}[colback=lightgray,colframe=accent,title=Practice Problems]
\footnotesize
\begin{columns}[T]
\column{0.33\textwidth}
\begin{enumerate}
  \item $\sqrt{a^6 b^3 c^5}$
  \item $\sqrt{18 a^2 b^7 c^3}$
\end{enumerate}
\column{0.33\textwidth}
\begin{enumerate}
  \setcounter{enumi}{2}
  \item $\sqrt[3]{-a^4 b^5 c^2}$
  \item $\sqrt[3]{-54 a^7 b^6}$
\end{enumerate}
\column{0.33\textwidth}
\begin{enumerate}
  \setcounter{enumi}{4}
  \item $\sqrt[4]{a^5 b^8 c^3}$
\end{enumerate}
\end{columns}
\end{tcolorbox}
\end{frame}

\begin{frame}{Practice: Simplify each radical to a mixed radical - Solutions}
\begin{tcolorbox}[colback=lightgray,colframe=primary,title=Solutions]
\footnotesize
\begin{enumerate}
  \item $\sqrt{a^6 b^3 c^5} = a^3 b c^2 \sqrt{b c}$
  \item $\sqrt{18 a^2 b^7 c^3} = 3 a b^3 c \sqrt{2 b c}$
  \item $\sqrt[3]{-a^4 b^5 c^2} = -a b \sqrt[3]{a b^2 c^2}$
  \item $\sqrt[3]{-54 a^7 b^6} = -3 a^2 b^2 \sqrt[3]{2 a b^2}$
  \item $\sqrt[4]{a^5 b^8 c^3} = a b^2 c \sqrt[4]{a c^3}$
\end{enumerate}
\end{tcolorbox}
\end{frame}

% Practice: Match each radical with its corresponding mixed radical (new values)
\begin{frame}{Practice: Match each radical with its corresponding Mixed radical}
\begin{tcolorbox}[colback=lightgray,colframe=accent,title=Match Problems]
\footnotesize
\begin{columns}[T]
\column{0.5\textwidth}
\begin{enumerate}
  \item $\sqrt{72}$
  \item $\sqrt{98}$
  \item $-\sqrt{128}$
  \item $\sqrt{242}$
  \item $\sqrt[3]{-216}$
\end{enumerate}
\column{0.5\textwidth}
\begin{enumerate}
  \item $6\sqrt{2}$
  \item $7\sqrt{2}$
  \item $-8\sqrt{2}$
  \item $11\sqrt{2}$
  \item $-6\sqrt{2}$
  \item $2\sqrt{6}$
  \item $-2\sqrt[3]{27}$
  \item $-6$
  \item $-2\sqrt[3]{27}$
  \item $-2\sqrt{2}$
\end{enumerate}
\end{columns}
\end{tcolorbox}
\end{frame}

\begin{frame}{Practice: Match each radical with its corresponding Mixed radical - Answers}
\begin{tcolorbox}[colback=lightgray,colframe=primary,title=Solutions]
\footnotesize
\begin{enumerate}
  \item $\sqrt{72} = 6\sqrt{2}$
  \item $\sqrt{98} = 7\sqrt{2}$
  \item $-\sqrt{128} = -8\sqrt{2}$
  \item $\sqrt{242} = 11\sqrt{2}$
  \item $\sqrt[3]{-216} = -6$
\end{enumerate}
\end{tcolorbox}
\end{frame}

% Challenging Radical Simplification Practice
\begin{frame}{Simplify the Following (Challenging)}
\begin{tcolorbox}[colback=lightgray,colframe=accent,title=Practice Problems]
\footnotesize
\begin{enumerate}
  \item $3\sqrt{72} + 5\sqrt{32}$
  \item $2\sqrt[3]{54} - 4\sqrt[3]{128} + 7\sqrt[3]{250}$
  \item $4\sqrt{98} - 2\sqrt{50} + 6\sqrt{18} - 3\sqrt{200}$
\end{enumerate}
\end{tcolorbox}
\end{frame}

% Challenging Radical Simplification Solutions (split)
\begin{frame}{Simplify the Following (Challenging) - Solutions (1/2)}
\begin{tcolorbox}[colback=lightgray,colframe=primary,title=Full Solutions]
\footnotesize
\begin{enumerate}
  \item $3\sqrt{72} + 5\sqrt{32}$
  \begin{align*}
    &= 3\sqrt{36 \times 2} + 5\sqrt{16 \times 2} \\
    &= 3 \times 6\sqrt{2} + 5 \times 4\sqrt{2} \\
    &= \textcolor{accent}{18\sqrt{2} + 20\sqrt{2}} \\
    &= \textcolor{accent}{38\sqrt{2}}
  \end{align*}
  \item $2\sqrt[3]{54} - 4\sqrt[3]{128} + 7\sqrt[3]{250}$
  \begin{align*}
    &= 2\sqrt[3]{27 \times 2} - 4\sqrt[3]{64 \times 2} + 7\sqrt[3]{125 \times 2} \\
    &= 2 \times 3\sqrt[3]{2} - 4 \times 4\sqrt[3]{2} + 7 \times 5\sqrt[3]{2} \\
    &= \textcolor{accent}{6\sqrt[3]{2} - 16\sqrt[3]{2} + 35\sqrt[3]{2}} \\
    &= \textcolor{accent}{25\sqrt[3]{2}}
  \end{align*}
\end{enumerate}
\end{tcolorbox}
\end{frame}

\begin{frame}{Simplify the Following (Challenging) - Solutions (2/2)}
\begin{tcolorbox}[colback=lightgray,colframe=primary,title=Full Solutions]
\footnotesize
\begin{enumerate}
  \setcounter{enumi}{2}
  \item $4\sqrt{98} - 2\sqrt{50} + 6\sqrt{18} - 3\sqrt{200}$
  \begin{align*}
    &= 4\sqrt{49 \times 2} - 2\sqrt{25 \times 2} + 6\sqrt{9 \times 2} - 3\sqrt{100 \times 2} \\
    &= 4 \times 7\sqrt{2} - 2 \times 5\sqrt{2} + 6 \times 3\sqrt{2} - 3 \times 10\sqrt{2} \\
    &= \textcolor{accent}{28\sqrt{2} - 10\sqrt{2} + 18\sqrt{2} - 30\sqrt{2}} \\
    &= \textcolor{accent}{6\sqrt{2}}
  \end{align*}
\end{enumerate}
\end{tcolorbox}
\end{frame}

\end{document} 