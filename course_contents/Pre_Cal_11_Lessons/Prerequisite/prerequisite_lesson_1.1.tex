\documentclass[aspectratio=169]{beamer}
\usepackage{amsmath}
\usepackage{amssymb}
\usepackage{graphicx}
\usepackage{tcolorbox}
\usepackage{booktabs}
\usepackage{colortbl}
\usepackage{xcolor}
\usepackage{tikz}
\usetikzlibrary{shapes,arrows,positioning}

% Custom colors
\definecolor{primary}{RGB}{41, 128, 185}
\definecolor{secondary}{RGB}{52, 152, 219}
\definecolor{accent}{RGB}{231, 76, 60}
\definecolor{lightgray}{RGB}{236, 240, 241}

% Theme customization
\usetheme{Madrid}
\usecolortheme{whale}
\setbeamercolor{structure}{fg=primary}
\setbeamercolor{background canvas}{bg=white}
\setbeamercolor{normal text}{fg=black}

% Title page info
\title{Pre-Calculus 11}
\subtitle{Prerequisite Skills Review}
\author{Created by Yi-Chen Lin}
\date{\today}

\begin{document}

\begin{frame}
    \titlepage
    \begin{tikzpicture}[remember picture,overlay]
        \fill[primary,opacity=0.1] (current page.south west) rectangle (current page.north east);
    \end{tikzpicture}
\end{frame}

\begin{frame}{BEDMAS}
    \begin{tcolorbox}[colback=lightgray,colframe=primary,title=BEDMAS Order]
        \begin{itemize}
            \item Brackets
            \item Exponents
            \item Divide
            \item Multiply
            \item Add
            \item Subtract
        \end{itemize}
    \end{tcolorbox}
\end{frame}

\begin{frame}{BEDMAS Practice}
    \begin{tcolorbox}[colback=lightgray,colframe=primary,title=Practice Problems]
        Evaluate each expression:
        \begin{enumerate}
            \item $4 \times 7$
            \item $(15 + 9)^2$
            \item $(-6)^2 - 3 \times 4$
            \item $(14 + 5)^2$
            \item $\frac{8^2 + 5^2}{4 - 5}$
            \item $\frac{3^3}{4^2}$
        \end{enumerate}
    \end{tcolorbox}
\end{frame}

\begin{frame}{BEDMAS Solutions - Part 1}
    \begin{tcolorbox}[colback=lightgray,colframe=accent,title=Solutions]
        \begin{enumerate}
            \item $4 \times 7 = 28$
            \item $(15 + 9)^2 = 24^2 = 576$
            \item $(-6)^2 - 3 \times 4 = 36 - 12 = 24$
        \end{enumerate}
    \end{tcolorbox}
\end{frame}

\begin{frame}{BEDMAS Solutions - Part 2}
    \begin{tcolorbox}[colback=lightgray,colframe=accent,title=Solutions]
        \begin{enumerate}
            \setcounter{enumi}{3}
            \item $(14 + 5)^2 = 19^2 = 361$
            \item $\frac{8^2 + 5^2}{4 - 5} = \frac{64 + 25}{-1} = -89$
            \item $\frac{3^3}{4^2} = \frac{27}{16}$
        \end{enumerate}
    \end{tcolorbox}
\end{frame}

\begin{frame}{Solving Equations - Practice}
    \begin{tcolorbox}[colback=lightgray,colframe=primary,title=Key Steps]
        \begin{itemize}
            \item Isolate the variable
            \item Perform the same operation on both sides
            \item Simplify step by step
        \end{itemize}
    \end{tcolorbox}
    
    \begin{tcolorbox}[colback=lightgray,colframe=primary,title=Practice Problems]
        Solve each equation:
        \begin{enumerate}
            \item $4x = 16$
            \item $\frac{7x}{5} = 14$
            \item $5x - 15 = -35$
            \item $6x + 17 = -13$
            \item $5(9x - 4) = 8x + 7$
        \end{enumerate}
    \end{tcolorbox}
\end{frame}

\begin{frame}{Solving Equations - Solutions Part 1}
    \begin{tcolorbox}[colback=lightgray,colframe=accent,title=Detailed Solutions]
        \begin{enumerate}
            \item $4x = 16$ \\
                Step 1: Divide both sides by 4 \\
                $\frac{4x}{4} = \frac{16}{4}$ \\
                $x = 4$
            
            \item $\frac{7x}{5} = 14$ \\
                Step 1: Multiply both sides by 5 \\
                $7x = 70$ \\
                Step 2: Divide both sides by 7 \\
                $x = 10$
        \end{enumerate}
    \end{tcolorbox}
\end{frame}

\begin{frame}{Solving Equations - Solutions Part 2}
    \begin{tcolorbox}[colback=lightgray,colframe=accent,title=Detailed Solutions]
        \begin{enumerate}
            \setcounter{enumi}{2}
            \item $5x - 15 = -35$ \\
                Step 1: Add 15 to both sides \\
                $5x = -20$ \\
                Step 2: Divide both sides by 5 \\
                $x = -4$
            
            \item $6x + 17 = -13$ \\
                Step 1: Subtract 17 from both sides \\
                $6x = -30$ \\
                Step 2: Divide both sides by 6 \\
                $x = -5$
        \end{enumerate}
    \end{tcolorbox}
\end{frame}

\begin{frame}{Solving Equations - Solutions Part 3}
    \begin{tcolorbox}[colback=lightgray,colframe=accent,title=Detailed Solutions]
        \begin{enumerate}
            \setcounter{enumi}{4}
            \item $5(9x - 4) = 8x + 7$ \\
                Step 1: Distribute the 5 \\
                $45x - 20 = 8x + 7$ \\
                Step 2: Move all x terms to one side \\
                $45x - 8x = 7 + 20$ \\
                Step 3: Combine like terms \\
                $37x = 27$ \\
                Step 4: Divide both sides by 37 \\
                $x = \frac{27}{37}$
        \end{enumerate}
    \end{tcolorbox}
\end{frame}

\begin{frame}{Table of Values - Practice}
    \begin{tcolorbox}[colback=lightgray,colframe=primary,title=Practice Problems]
        \footnotesize
        Complete the table of values for each equation:
        
        1. $y = 6x - 4$
        \vspace*{-1.5em}
        \begin{center}
        \begin{tabular}{|c|c|}
            \hline
            \rowcolor{primary!20} x & y \\
            \hline
            2 & \\
            5 & \\
            8 & \\
            \hline
        \end{tabular}
        \end{center}
        
        2. $4x + 5y = 30$
        \vspace*{-1.5em}
        \begin{center}
        \begin{tabular}{|c|c|}
            \hline
            \rowcolor{primary!20} x & y \\
            \hline
            0 & \\
            3 & \\
            6 & \\
            9 & \\
            \hline
        \end{tabular}
        \end{center}
    \end{tcolorbox}
    \vfill
\end{frame}

\begin{frame}{Table of Values - Solutions Part 1}
    \begin{tcolorbox}[colback=lightgray,colframe=accent,title=Solution for $y = 6x - 4$]
        \footnotesize
        Step 1: For each x value, substitute into the equation \\
        When x = 2: $y = 6(2) - 4 = 12 - 4 = 8$ \\
        When x = 5: $y = 6(5) - 4 = 30 - 4 = 26$ \\
        When x = 8: $y = 6(8) - 4 = 48 - 4 = 44$
        \vspace*{-1.5em}
        \begin{center}
        \begin{tabular}{|c|c|}
            \hline
            \rowcolor{primary!20} x & y \\
            \hline
            2 & 8 \\
            5 & 26 \\
            8 & 44 \\
            \hline
        \end{tabular}
        \end{center}
    \end{tcolorbox}
    \vfill
\end{frame}

\begin{frame}{Table of Values - Solutions Part 2}
    \begin{tcolorbox}[colback=lightgray,colframe=accent,title=Solution for $4x + 5y = 30$]
        \footnotesize
        \begin{columns}
            \begin{column}{0.6\textwidth}
                Step 1: Solve for y \\
                $5y = 30 - 4x$ \\[2em]
                $y = 6 - \frac{4x}{5}$ \\[2.5em]
                
                Step 2: Substitute x values \\
                When x = 0: $y = 6 - \frac{4(0)}{5} = 6$ \\[1.5em]
                When x = 3: $y = 6 - \frac{4(3)}{5} = 3.6$ \\[1.5em]
                When x = 6: $y = 6 - \frac{4(6)}{5} = 1.2$ \\[1.5em]
                When x = 9: $y = 6 - \frac{4(9)}{5} = -1.2$
            \end{column}
            \begin{column}{0.4\textwidth}
                \vspace*{-1.5em} % Adjust vertical space for the table
                \begin{center}
                \begin{tabular}{|c|c|}
                    \hline
                    \rowcolor{primary!20} x & y \\
                    \hline
                    0 & 6 \\
                    3 & 3.6 \\
                    6 & 1.2 \\
                    9 & -1.2 \\
                    \hline
                \end{tabular}
                \end{center}
            \end{column}
        \end{columns}
    \end{tcolorbox}
    \vfill
\end{frame}

\begin{frame}{Lines: Slopes and Y-intercepts - Practice}
    \begin{tcolorbox}[colback=lightgray,colframe=primary,title=Standard Form: $y = mx + b$]
        \footnotesize
        \begin{itemize}
            \setlength{\itemsep}{0pt}
            \addtolength{\itemsep}{-0.5em}
            \item $m$ = slope
            \item $b$ = y-intercept
        \end{itemize}
    \end{tcolorbox}
    \vspace*{-1em}
    \begin{tcolorbox}[colback=lightgray,colframe=primary,title=Practice Problems]
        \footnotesize
        Find slope and y-intercept for each equation:
        \begin{enumerate}
            \setlength{\itemsep}{0pt}
            \addtolength{\itemsep}{-0.5em}
            \item $y = 4x + 3$
            \item $y = -7x + 5$
            \item $y = 0.8x + 12$
            \item $3x + y = 15$
            \item $\frac{18x - 9}{3} = y$
            \item $9x + 12y = -6$
        \end{enumerate}
    \end{tcolorbox}
    \vfill
\end{frame}

\begin{frame}{Lines: Slopes and Y-intercepts - Solutions Part 1}
    \begin{tcolorbox}[colback=lightgray,colframe=accent,title=Detailed Solutions]
        \begin{enumerate}
            \item $y = 4x + 3$ \\
                Slope = 4 (coefficient of x) \\
                Y-intercept = 3 (constant term)
            
            \item $y = -7x + 5$ \\
                Slope = -7 (coefficient of x) \\
                Y-intercept = 5 (constant term)
            
            \item $y = 0.8x + 12$ \\
                Slope = 0.8 (coefficient of x) \\
                Y-intercept = 12 (constant term)
        \end{enumerate}
    \end{tcolorbox}
\end{frame}

\begin{frame}{Lines: Slopes and Y-intercepts - Solutions Part 2}
    \begin{tcolorbox}[colback=lightgray,colframe=accent,title=Detailed Solutions]
        \begin{enumerate}
            \setcounter{enumi}{3}
            \item $3x + y = 15$ \\
                Step 1: Solve for y \\
                $y = -3x + 15$ \\
                Slope = -3 \\
                Y-intercept = 15
            
            \item $\frac{18x - 9}{3} = y$ \\
                Step 1: Simplify \\
                $y = 6x - 3$ \\
                Slope = 6 \\
                Y-intercept = -3
        \end{enumerate}
    \end{tcolorbox}
\end{frame}

\begin{frame}{Lines: Slopes and Y-intercepts - Solutions Part 3}
    \begin{tcolorbox}[colback=lightgray,colframe=accent,title=Detailed Solutions]
        \begin{enumerate}
            \setcounter{enumi}{5}
            \item $9x + 12y = -6$ \\
                Step 1: Solve for y \\
                $12y = -9x - 6$ \\
                Step 2: Divide by 12 \\
                $y = -\frac{9}{12}x - \frac{6}{12}$ \\
                Step 3: Simplify \\
                $y = -\frac{3}{4}x - \frac{1}{2}$ \\
                Slope = $-\frac{3}{4}$ \\
                Y-intercept = $-\frac{1}{2}$
        \end{enumerate}
    \end{tcolorbox}
\end{frame}

\begin{frame}{Isolating Variables - Practice}
    \begin{tcolorbox}[colback=lightgray,colframe=primary,title=Key Steps]
        \begin{itemize}
            \item Move all terms with the variable to one side
            \item Move all other terms to the opposite side
            \item Factor out the variable if necessary
            \item Divide by the coefficient
        \end{itemize}
    \end{tcolorbox}
    
    \begin{tcolorbox}[colback=lightgray,colframe=primary,title=Practice Problems]
        Isolate x in each equation:
        \begin{enumerate}
            \item $bd + cx = e$
            \item $bx + c = fx + d$
        \end{enumerate}
    \end{tcolorbox}
\end{frame}

\begin{frame}{Isolating Variables - Solutions}
    \begin{tcolorbox}[colback=lightgray,colframe=accent,title=Detailed Solutions]
        \begin{enumerate}
            \item $bd + cx = e$ \\
                Step 1: Move bd to the other side \\
                $cx = e - bd$ \\
                Step 2: Divide by c \\
                $x = \frac{e - bd}{c}$
            
            \item $bx + c = fx + d$ \\
                Step 1: Move all x terms to one side \\
                $bx - fx = d - c$ \\
                Step 2: Factor out x \\
                $x(b - f) = d - c$ \\
                Step 3: Divide by (b - f) \\
                $x = \frac{d - c}{b - f}$
        \end{enumerate}
    \end{tcolorbox}
\end{frame}

\begin{frame}{Summary}
    \begin{tcolorbox}[colback=lightgray,colframe=primary,title=Key Concepts]
        \begin{itemize}
            \item BEDMAS order of operations
            \item Solving equations step by step
            \item Creating and using tables of values
            \item Understanding slopes and y-intercepts
            \item Isolating variables in equations
        \end{itemize}
    \end{tcolorbox}
\end{frame}

\end{document} 