\documentclass[aspectratio=169]{beamer}

% Theme and colors
\usetheme{Madrid}
\usecolortheme{whale}
% Custom colors
\definecolor{primary}{RGB}{41, 128, 185}
\definecolor{secondary}{RGB}{52, 152, 219}
\definecolor{accent}{RGB}{231, 76, 60}
\definecolor{lightgray}{RGB}{236, 240, 241}

% Packages
\usepackage[utf8]{inputenc}
\usepackage{graphicx}
\usepackage{amsmath}
\usepackage{amsfonts}
\usepackage{amssymb}
\usepackage{mathpazo} % Palatino font
\usepackage{multirow}
\usepackage{tcolorbox}
\usepackage{tikz} % For diagrams
\usepackage{caption} % For figure captions
\usepackage{adjustbox} % For adjusting box
\usepackage{float} % For better float control

% Beamer specific configurations
\setbeamercolor{structure}{fg=primary}
\setbeamercolor{background canvas}{bg=white}
\setbeamercolor{normal text}{fg=black}

% Information for the title page
\title{Lesson 4: The Quadratic Formula}
\author{Yi-Chen Lin}
\date{June 10, 2025}

\begin{document}

% Title page
\begin{frame}
    \titlepage
\end{frame}

% Table of Contents
\begin{frame}{Table of Contents}
    \tableofcontents
\end{frame}

% Section I: Quadratic Functions in Standard Form
\section{Quadratic Functions in Standard Form}

\begin{frame}{I) Quadratic Functions in Standard Form: (A, B, C)}
    \begin{tcolorbox}[colback=lightgray,colframe=primary,title=Key Concepts]
        \footnotesize
        \begin{itemize}
            \item Most quadratic functions are written in \textbf{standard form}:
                \[ y = ax^2 + bx + c \]
            \item We know it's quadratic because the largest exponent of "$x$" is 2.
            \item The letters "$a$", "$b$", and "$c$" are \textbf{coefficients} (real numbers).
            \item The constant "$c$" is the \textbf{Y-intercept}.
                \begin{itemize}
                    \item If "$x$" is equal to zero, then $y = c$.
                \end{itemize}
            \item The coefficient "$a$" indicates which way the graph opens:
                \begin{itemize}
                    \item If $a > 0$, the parabola opens \textbf{up}.
                    \item If $a < 0$, the parabola opens \textbf{down}.
                \end{itemize}
        \end{itemize}
    \end{tcolorbox}
\end{frame}

% Practice Problems for Standard Form
\subsection{Practice: Standard Form Properties}

\begin{frame}{Practice: Find Coefficients and Properties (Problem 1)}
    \begin{tcolorbox}[colback=lightgray,colframe=primary,title=Problem 1]
        \footnotesize
        For the given quadratic function, identify the coefficients "$a$", "$b$", "$c$", the Y-intercept, and state which way the graph opens:\\
        \[ y = 2x^2 - 5x + 7 \]
    \end{tcolorbox}
\end{frame}

\begin{frame}{Practice: Find Coefficients and Properties (Solution 1)}
    \begin{tcolorbox}[colback=lightgray,colframe=accent,title=Solution 1]
        \footnotesize
        For $y = 2x^2 - 5x + 7$:
        \begin{itemize}
            \item $a = 2$
            \item $b = -5$
            \item $c = 7$
            \item \textbf{Y-intercept:} $(0, 7)$ (Since $x=0 \implies y=7$)
            \item \textbf{Opens:} Up (Since $a = 2 > 0$)
        \end{itemize}
    \end{tcolorbox}
\end{frame}

\begin{frame}{Practice: Find Coefficients and Properties (Problem 2)}
    \begin{tcolorbox}[colback=lightgray,colframe=primary,title=Problem 2]
        \footnotesize
        For the given quadratic function, identify the coefficients "$a$", "$b$", "$c$", the Y-intercept, and state which way the graph opens:\\
        \[ y = -3x^2 + x - 9 \]
    \end{tcolorbox}
\end{frame}

\begin{frame}{Practice: Find Coefficients and Properties (Solution 2)}
    \begin{tcolorbox}[colback=lightgray,colframe=accent,title=Solution 2]
        \footnotesize
        For $y = -3x^2 + x - 9$:
        \begin{itemize}
            \item $a = -3$
            \item $b = 1$
            \item $c = -9$
            \item \textbf{Y-intercept:} $(0, -9)$ (Since $x=0 \implies y=-9$)
            \item \textbf{Opens:} Down (Since $a = -3 < 0$)
        \end{itemize}
    \end{tcolorbox}
\end{frame}

\begin{frame}{Practice: Find Coefficients and Properties (Problem 3)}
    \begin{tcolorbox}[colback=lightgray,colframe=primary,title=Problem 3]
        \footnotesize
        For the given quadratic function, identify the coefficients "$a$", "$b$", "$c$", the Y-intercept, and state which way the graph opens:\\
        \[ y = 5(x-2)^2 + 1 \]
        \textit{(Hint: Expand and simplify to standard form first.)}
    \end{tcolorbox}
\end{frame}

\begin{frame}{Practice: Find Coefficients and Properties (Solution 3)}
    \begin{tcolorbox}[colback=lightgray,colframe=accent,title=Solution 3]
        \footnotesize
        First, expand and simplify $y = 5(x-2)^2 + 1$:
        \begin{align*}
            y &= 5(x^2 - 4x + 4) + 1 \\
            y &= 5x^2 - 20x + 20 + 1 \\
            y &= 5x^2 - 20x + 21
        \end{align*}
        Now, for $y = 5x^2 - 20x + 21$:
        \begin{itemize}
            \item $a = 5$
            \item $b = -20$
            \item $c = 21$
            \item \textbf{Y-intercept:} $(0, 21)$
            \item \textbf{Opens:} Up (Since $a = 5 > 0$)
        \end{itemize}
    \end{tcolorbox}
\end{frame}


% Section II: Quadratic Formula
\section{The Quadratic Formula}

\begin{frame}{II) The Quadratic Formula}
    \begin{tcolorbox}[colback=lightgray,colframe=primary,title=Solving Quadratic Equations]
        \footnotesize
        \begin{itemize}
            \item When solving a quadratic equation in standard form ($ax^2 + bx + c = 0$), we can use the Quadratic Formula to solve for the "$x$" variable:
                \[ x = \frac{-b \pm \sqrt{b^2 - 4ac}}{2a} \]
            \item The letters "$a$", "$b$", and "$c$" are the coefficients from the standard form.
            \item \textbf{Important Conditions:}
                \begin{itemize}
                    \item $a \neq 0$ (Cannot divide by zero!)
                    \item $b^2 - 4ac \ge 0$ (Cannot square root a negative number in real numbers)
                \end{itemize}
            \item The Quadratic Formula can be used to find the "roots" (x-intercepts) without a graphing calculator.
            \item It is particularly useful for equations that \textbf{cannot be factored}.
        \end{itemize}
    \end{tcolorbox}
\end{frame}

\begin{frame}{Conditions for Using the Quadratic Formula}
    \begin{tcolorbox}[colback=lightgray,colframe=primary,title=Key Rules]
        \footnotesize
        \begin{itemize}
            \item \textbf{One side of the equation must be zero!} (Move all terms to one side).
                \begin{itemize}
                    \item Example: If $3x^2 + 7 = 2x$, rearrange to $3x^2 - 2x + 7 = 0$.
                \end{itemize}
            \item \textbf{Equation must be a Quadratic Function and in "Standard Form":} $ax^2 + bx + c = 0$.
            \item \textbf{Discriminant Condition:} If $(b^2 - 4ac)$ is negative, then you will have "NO Real Solutions"! (No real answer for $x$).
                \begin{itemize}
                    \item Example: $x^2 + x + 1 = 0$ has $b^2 - 4ac = 1^2 - 4(1)(1) = 1 - 4 = -3 < 0$. So, no real solutions.
                \end{itemize}
        \end{itemize}
    \end{tcolorbox}
\end{frame}

\subsection{Solving with the Quadratic Formula}

\begin{frame}{Ex: Solve for "$x$" (Problem 1)}
    \begin{tcolorbox}[colback=lightgray,colframe=primary,title=Problem 1]
        \footnotesize
        Solve for "$x$":
        \[ x^2 + 3x - 10 = 0 \]
    \end{tcolorbox}
\end{frame}

\begin{frame}{Ex: Solve for "$x$" (Solution 1) - Part 1}
    \begin{tcolorbox}[colback=lightgray,colframe=accent,title=Solution 1: Identify Coefficients]
        \footnotesize
        For $x^2 + 3x - 10 = 0$:
        \begin{itemize}
            \item First, identify coefficients: $a=1$, $b=3$, $c=-10$.
        \end{itemize}
    \end{tcolorbox}
\end{frame}

\begin{frame}{Ex: Solve for "$x$" (Solution 1) - Part 2}
    \begin{tcolorbox}[colback=lightgray,colframe=accent,title=Solution 1: Plug into Formula]
        \footnotesize
        Plug coefficients into the formula:
        \begin{align*}
            x &= \frac{-b \pm \sqrt{b^2 - 4ac}}{2a} \\
            x &= \frac{-(3) \pm \sqrt{(3)^2 - 4(1)(-10)}}{2(1)} \\
            x &= \frac{-3 \pm \sqrt{9 + 40}}{2} \\
            x &= \frac{-3 \pm \sqrt{49}}{2} \\
            x &= \frac{-3 \pm 7}{2}
        \end{align*}
    \end{tcolorbox}
\end{frame}

\begin{frame}{Ex: Solve for "$x$" (Solution 1) - Part 3}
    \begin{tcolorbox}[colback=lightgray,colframe=accent,title=Solution 1: Final Answers]
        \footnotesize
        You get two answers:
        \begin{align*}
            x_1 &= \frac{-3 + 7}{2} = \frac{4}{2} = 2 \\
            x_2 &= \frac{-3 - 7}{2} = \frac{-10}{2} = -5
        \end{align*}
    \end{tcolorbox}
\end{frame}

\begin{frame}{Ex: Solve for "$x$" to 2 decimal places (Problem 2)}
    \begin{tcolorbox}[colback=lightgray,colframe=primary,title=Problem 2]
        \footnotesize
        Solve for "$x$" to 2 decimal places:
        \[ 2x^2 - 6x - 5 = 0 \]
    \end{tcolorbox}
\end{frame}

\begin{frame}{Ex: Solve for "$x$" to 2 decimal places (Solution 2) - Part 1}
    \begin{tcolorbox}[colback=lightgray,colframe=accent,title=Solution 2: Identify Coefficients]
        \footnotesize
        For $2x^2 - 6x - 5 = 0$:
        \begin{itemize}
            \item First, identify coefficients: $a=2$, $b=-6$, $c=-5$.
        \end{itemize}
    \end{tcolorbox}
\end{frame}

\begin{frame}{Ex: Solve for "$x$" to 2 decimal places (Solution 2) - Part 2}
    \begin{tcolorbox}[colback=lightgray,colframe=accent,title=Solution 2: Plug into Formula]
        \footnotesize
        Plug coefficients into the formula:
        \begin{align*}
            x &= \frac{-b \pm \sqrt{b^2 - 4ac}}{2a} \\
            x &= \frac{-(-6) \pm \sqrt{(-6)^2 - 4(2)(-5)}}{2(2)} \\
            x &= \frac{6 \pm \sqrt{36 + 40}}{4} \\
            x &= \frac{6 \pm \sqrt{76}}{4}
        \end{align*}
    \end{tcolorbox}
\end{frame}

\begin{frame}{Ex: Solve for "$x$" to 2 decimal places (Solution 2) - Part 3}
    \begin{tcolorbox}[colback=lightgray,colframe=accent,title=Solution 2: Final Answers]
        \footnotesize
        Continue solving for $x$:
        \begin{align*}
            x &= \frac{6 \pm \sqrt{76}}{4} \\
            x &= \frac{6 \pm 8.7178}{4} \quad \text{(approx.)}
        \end{align*}
        You get two answers:
        \begin{align*}
            x_1 &= \frac{6 + 8.7178}{4} = \frac{14.7178}{4} \approx 3.68 \\
            x_2 &= \frac{6 - 8.7178}{4} = \frac{-2.7178}{4} \approx -0.68
        \end{align*}
    \end{tcolorbox}
\end{frame}

\begin{frame}{Ex: Solve for "$x$" (Problem 3 - No Real Solutions)}
    \begin{tcolorbox}[colback=lightgray,colframe=primary,title=Problem 3]
        \footnotesize
        Solve for "$x$":
        \[ x^2 - 4x + 7 = 0 \]
    \end{tcolorbox}
\end{frame}

\begin{frame}{Ex: Solve for "$x$" (Solution 3 - No Real Solutions)}
    \begin{tcolorbox}[colback=lightgray,colframe=accent,title=Solution 3]
        \footnotesize
        For $x^2 - 4x + 7 = 0$:
        \begin{itemize}
            \item First, identify coefficients: $a=1$, $b=-4$, $c=7$.
            \item Plug coefficients into the formula:
                \begin{align*}
                    x &= \frac{-b \pm \sqrt{b^2 - 4ac}}{2a} \\
                    x &= \frac{-(-4) \pm \sqrt{(-4)^2 - 4(1)(7)}}{2(1)} \\
                    x &= \frac{4 \pm \sqrt{16 - 28}}{2} \\
                    x &= \frac{4 \pm \sqrt{-12}}{2}
                \end{align*}
            \item Since we are taking the square root of a negative number ($-12$), there are \textbf{No Real Solutions}.
        \end{itemize}
    \end{tcolorbox}
\end{frame}

\subsection{Using the Quadratic Formula to Find the Vertex}

\begin{frame}{Using the Quadratic Formula to Find the Vertex - Part 1}
    \begin{tcolorbox}[colback=lightgray,colframe=primary,title=Key Ideas]
        \footnotesize
        \begin{itemize}
            \item The quadratic formula gives you the two x-intercepts (if they exist).
            \item The vertex is exactly in the middle of the two x-intercepts.
            \item We can find the x-coordinate of the vertex by averaging the two x-intercepts from the quadratic formula. Let's see how this works:
        \end{itemize}
    \end{tcolorbox}
\end{frame}

\begin{frame}{Using the Quadratic Formula to Find the Vertex - Part 2 (Derivation)}
    \begin{tcolorbox}[colback=lightgray,colframe=primary,title=Derivation of Vertex Formula]
        \footnotesize
        Given the two x-intercepts $x_1 = \frac{-b + \sqrt{b^2 - 4ac}}{2a}$ and $x_2 = \frac{-b - \sqrt{b^2 - 4ac}}{2a}$:
        \begin{align*}
            x_{\text{vertex}} &= \frac{x_1 + x_2}{2} \\
            x_{\text{vertex}} &= \frac{\left(\frac{-b + \sqrt{b^2 - 4ac}}{2a}\right) + \left(\frac{-b - \sqrt{b^2 - 4ac}}{2a}\right)}{2}
        \end{align*}
    \end{tcolorbox}
\end{frame}

\begin{frame}{Using the Quadratic Formula to Find the Vertex - Part 3 (Derivation)}
    \begin{tcolorbox}[colback=lightgray,colframe=primary,title=Derivation of Vertex Formula (Cont.)]
        \footnotesize
        Continuing the derivation:
        \begin{align*}
            x_{\text{vertex}} &= \frac{\frac{-b + \sqrt{b^2 - 4ac} - b - \sqrt{b^2 - 4ac}}{2a}}{2} \\
            x_{\text{vertex}} &= \frac{\frac{-2b}{2a}}{2} \\
            x_{\text{vertex}} &= \frac{-b}{2a}
        \end{align*}
        So, the x-coordinate of the vertex is given by:
        \[ x_{\text{vertex}} = \frac{-b}{2a} \]
    \end{tcolorbox}
\end{frame}

\begin{frame}{Using the Quadratic Formula to Find the Vertex - Part 4}
    \begin{tcolorbox}[colback=lightgray,colframe=primary,title=Finding the Y-coordinate]
        \footnotesize
        \begin{itemize}
            \item To find the y-coordinate of the vertex, plug this $x_{\text{vertex}}$ value ($\frac{-b}{2a}$) back into the original quadratic equation $y = ax^2 + bx + c$.
            \item This formula $x_{\text{vertex}} = \frac{-b}{2a}$ is also known as the equation of the \textbf{Axis of Symmetry}.
        \end{itemize}
    \end{tcolorbox}
\end{frame}

\begin{frame}{Ex: Find Axis of Symmetry and Vertex (Problem 1)}
    \begin{tcolorbox}[colback=lightgray,colframe=primary,title=Problem 1]
        \footnotesize
        Given the equation below, find the equation of the Axis of Symmetry (AOS) and the coordinates of the Vertex:\\
        \[ y = x^2 - 8x + 15 \]
    \end{tcolorbox}
\end{frame}

\begin{frame}{Ex: Find Axis of Symmetry and Vertex (Solution 1) - Part 1}
    \begin{tcolorbox}[colback=lightgray,colframe=accent,title=Solution 1: Identify Coefficients]
        \footnotesize
        For $y = x^2 - 8x + 15$:
        \begin{itemize}
            \item Identify coefficients: $a=1$, $b=-8$, $c=15$.
        \end{itemize}
    \end{tcolorbox}
\end{frame}

\begin{frame}{Ex: Find Axis of Symmetry and Vertex (Solution 1) - Part 2}
    \begin{tcolorbox}[colback=lightgray,colframe=accent,title=Solution 1: Axis of Symmetry (AOS)]
        \footnotesize
        \textbf{Axis of Symmetry (AOS):}
        \begin{align*}
            x &= \frac{-b}{2a} \\
            x &= \frac{-(-8)}{2(1)} \\
            x &= \frac{8}{2} \\
            x &= 4
        \end{align*}
    \end{tcolorbox}
\end{frame}

\begin{frame}{Ex: Find Axis of Symmetry and Vertex (Solution 1) - Part 3}
    \begin{tcolorbox}[colback=lightgray,colframe=accent,title=Solution 1: Vertex Coordinates]
        \footnotesize
        \textbf{Vertex:}
        \begin{itemize}
            \item Plug $x=4$ into the original equation to find the y-coordinate:
                \begin{align*}
                    y &= (4)^2 - 8(4) + 15 \\
                    y &= 16 - 32 + 15 \\
                    y &= -1
                \end{align*}
            \item The coordinates of the Vertex are $(4, -1)$.
        \end{itemize}
    \end{tcolorbox}
\end{frame}

\begin{frame}{Ex: Find Axis of Symmetry and Vertex (Problem 2)}
    \begin{tcolorbox}[colback=lightgray,colframe=primary,title=Problem 2]
        \footnotesize
        Given the equation below, find the equation of the Axis of Symmetry (AOS) and the coordinates of the Vertex:\\
        \[ y = -2x^2 - 12x - 10 \]
    \end{tcolorbox}
\end{frame}

\begin{frame}{Ex: Find Axis of Symmetry and Vertex (Solution 2)}
    \begin{tcolorbox}[colback=lightgray,colframe=accent,title=Solution 2]
        \footnotesize
        For $y = -2x^2 - 12x - 10$:
        \begin{itemize}
            \item Identify coefficients: $a=-2$, $b=-12$, $c=-10$.
            \item \textbf{Axis of Symmetry (AOS):}
                \begin{align*}
                    x &= \frac{-b}{2a} \\
                    x &= \frac{-(-12)}{2(-2)} \\
                    x &= \frac{12}{-4} \\
                    x &= -3
                \end{align*}
            \item \textbf{Vertex:}
                \begin{itemize}
                    \item Plug $x=-3$ into the original equation to find the y-coordinate:
                        \begin{align*}
                            y &= -2(-3)^2 - 12(-3) - 10 \\
                            y &= -2(9) + 36 - 10 \\
                            y &= -18 + 36 - 10 \\
                            y &= 8
                        \end{align*}
                    \item The coordinates of the Vertex are $(-3, 8)$.
                \end{itemize}
        \end{itemize}
    \end{tcolorbox}
\end{frame}

% Section IV: Applications of the Quadratic Formula (Word Problems)
\section{Applications of the Quadratic Formula}

\begin{frame}{Word Problem: Ball Toss (Problem)}
    \begin{tcolorbox}[colback=lightgray,colframe=primary,title=Problem]
        \footnotesize
        A ball is thrown upward from a 1.5-meter platform with an initial velocity of 14 meters per second. The height $h$ (in meters) of the ball after $t$ seconds is given by the formula:
        \[ h(t) = -4.9t^2 + 14t + 1.5 \]
        \begin{enumerate}
            \item When does the ball reach its maximum height?
            \item What is the maximum height that the ball reaches?
            \item When does the ball hit the ground after it is thrown? (Round to 2 decimal places)
        \end{enumerate}
    \end{tcolorbox}
\end{frame}

\begin{frame}{Word Problem: Ball Toss (Solution Part 1A)}
    \begin{tcolorbox}[colback=lightgray,colframe=accent,title=Solution Part 1A: Max Height Time]
        \footnotesize
        For $h(t) = -4.9t^2 + 14t + 1.5$, we have $a=-4.9$, $b=14$, $c=1.5$.
        \begin{enumerate}
            \item \textbf{When does the ball reach its maximum height?}
            \item To find the time to max height, use $t = \frac{-b}{2a}$:
                \begin{align*}
                    t &= \frac{-(14)}{2(-4.9)} \\
                    t &= \frac{-14}{-9.8} \\
                    t &\approx 1.43 \text{ seconds}
                \end{align*}
        \end{enumerate}
    \end{tcolorbox}
\end{frame}

\begin{frame}{Word Problem: Ball Toss (Solution Part 1B)}
    \begin{tcolorbox}[colback=lightgray,colframe=accent,title=Solution Part 1B: Max Height Value]
        \footnotesize
        For $h(t) = -4.9t^2 + 14t + 1.5$, we have $a=-4.9$, $b=14$, $c=1.5$.
        \begin{enumerate}
            \setcounter{enumi}{1} % Continue numbering from previous part
            \item \textbf{What is the maximum height that the ball reaches?}
            \item Plug $t \approx 1.43$ seconds into the height formula:
                \begin{align*}
                    h(1.43) &= -4.9(1.43)^2 + 14(1.43) + 1.5 \\
                    h(1.43) &= -4.9(2.0449) + 20.02 + 1.5 \\
                    h(1.43) &= -10.02 + 20.02 + 1.5 \\
                    h(1.43) &\approx 11.5 \text{ meters}
                \end{align*}
        \end{enumerate}
    \end{tcolorbox}
\end{frame}

\begin{frame}{Word Problem: Ball Toss (Solution Part 2A.1)}
    \begin{tcolorbox}[colback=lightgray,colframe=accent,title=Solution Part 2A.1: Set up Equation]
        \footnotesize
        For $h(t) = -4.9t^2 + 14t + 1.5$, we have $a=-4.9$, $b=14$, $c=1.5$.
        \begin{enumerate}
            \item \textbf{When does the ball hit the ground after it is thrown?}
            \item When the ball hits the ground, the height $h(t) = 0$. So, we solve:
                \[ 0 = -4.9t^2 + 14t + 1.5 \]
        \end{enumerate}
    \end{tcolorbox}
\end{frame}

\begin{frame}{Word Problem: Ball Toss (Solution Part 2A.2A)}
    \begin{tcolorbox}[colback=lightgray,colframe=accent,title=Solution 2A.2A: Apply Quadratic Formula]
        \footnotesize
        For $0 = -4.9t^2 + 14t + 1.5$, we have $a=-4.9$, $b=14$, $c=1.5$.
        \begin{itemize}
            \item Use the Quadratic Formula:
                \begin{align*}
                    t &= \frac{-b \pm \sqrt{b^2 - 4ac}}{2a} \\
                    t &= \frac{-(14) \pm \sqrt{(14)^2 - 4(-4.9)(1.5)}}{2(-4.9)}
                \end{align*}
        \end{itemize}
    \end{tcolorbox}
\end{frame}

\begin{frame}{Word Problem: Ball Toss (Solution Part 2A.2B.1A)}
    \begin{tcolorbox}[colback=lightgray,colframe=accent,title=Solution 2A.2B.1A: Simplify Square Root]
        \footnotesize
        For $0 = -4.9t^2 + 14t + 1.5$, we have $a=-4.9$, $b=14$, $c=1.5$.
        \begin{itemize}
            \item Continuing the simplification of the Quadratic Formula:
                \begin{align*}
                    t &= \frac{-14 \pm \sqrt{196 + 29.4}}{-9.8}
                \end{align*}
        \end{itemize}
    \end{tcolorbox}
\end{frame}

\begin{frame}{Word Problem: Ball Toss (Solution Part 2A.2B.1B)}
    \begin{tcolorbox}[colback=lightgray,colframe=accent,title=Solution 2A.2B.1B: Intermediate Result]
        \footnotesize
        For $0 = -4.9t^2 + 14t + 1.5$, we have $a=-4.9$, $b=14$, $c=1.5$.
        \begin{itemize}
            \item Continuing the simplification:
                \begin{align*}
                    t &= \frac{-14 \pm \sqrt{225.4}}{-9.8}
                \end{align*}
        \end{itemize}
    \end{tcolorbox}
\end{frame}

\begin{frame}{Word Problem: Ball Toss (Solution Part 2B.1)}
    \begin{tcolorbox}[colback=lightgray,colframe=accent,title=Solution Part 2B.1: Approximate Value]
        \footnotesize
        For $h(t) = -4.9t^2 + 14t + 1.5$, we have $a=-4.9$, $b=14$, $c=1.5$.
        \begin{itemize}
            \item Calculating the approximate value:
                \begin{align*}
                    t &= \frac{-14 \pm \sqrt{225.4}}{-9.8} \\
                    t &= \frac{-14 \pm 15.013}{-9.8} \quad \text{(approx.)}
                \end{align*}
        \end{itemize}
    \end{tcolorbox}
\end{frame}

\begin{frame}{Word Problem: Ball Toss (Solution Part 2B.2)}
    \begin{tcolorbox}[colback=lightgray,colframe=accent,title=Solution Part 2B.2: Final Answers and Conclusion]
        \footnotesize
        For $h(t) = -4.9t^2 + 14t + 1.5$, we have $a=-4.9$, $b=14$, $c=1.5$.
        \begin{itemize}
            \item Two possible values for $t$:
                \begin{align*}
                    t_1 &= \frac{-14 + 15.013}{-9.8} = \frac{1.013}{-9.8} \approx -0.10 \text{ seconds (extraneous)} \\
                    t_2 &= \frac{-14 - 15.013}{-9.8} = \frac{-29.013}{-9.8} \approx 2.96 \text{ seconds}
                \end{align*}
            \item Since time cannot be negative, the ball hits the ground after approximately \textbf{2.96 seconds}.
        \end{itemize}
    \end{tcolorbox}
\end{frame}

\begin{frame}{Word Problem: Rocket Launch (Problem)}
    \begin{tcolorbox}[colback=lightgray,colframe=primary,title=Problem]
        \footnotesize
        A small rocket is launched from a height of 5 meters above the ground. Its height $h$ (in meters) above the ground $t$ seconds after launch is modeled by the equation:
        \[ h(t) = -3t^2 + 18t + 5 \]
        The rocket\'s tracking device is designed to activate when the rocket is at a height of 20 meters. After how many seconds should the tracking device activate on its way down? (Round to 2 decimal places)
    \end{tcolorbox}
\end{frame}

\begin{frame}{Word Problem: Rocket Launch (Solution Part 1)}
    \begin{tcolorbox}[colback=lightgray,colframe=accent,title=Solution Part 1: Set up Equation and Identify Coefficients]
        \footnotesize
        We want to find $t$ when $h(t) = 20$. So, set the equation:
        \begin{align*}
            20 &= -3t^2 + 18t + 5 \\
            0 &= -3t^2 + 18t + 5 - 20 \\
            0 &= -3t^2 + 18t - 15
        \end{align*}
        For $0 = -3t^2 + 18t - 15$, we have $a=-3$, $b=18$, $c=-15$.
    \end{tcolorbox}
\end{frame}

\begin{frame}{Word Problem: Rocket Launch (Solution Part 2A.1)}
    \begin{tcolorbox}[colback=lightgray,colframe=accent,title=Solution Part 2A.1: Apply Quadratic Formula]
        \footnotesize
        For $0 = -3t^2 + 18t - 15$, we have $a=-3$, $b=18$, $c=-15$.
        \begin{itemize}
            \item Use the Quadratic Formula:
                \begin{align*}
                    t &= \frac{-b \pm \sqrt{b^2 - 4ac}}{2a} \\
                    t &= \frac{-(18) \pm \sqrt{(18)^2 - 4(-3)(-15)}}{2(-3)}
                \end{align*}
        \end{itemize}
    \end{tcolorbox}
\end{frame}

\begin{frame}{Word Problem: Rocket Launch (Solution Part 2A.2)}
    \begin{tcolorbox}[colback=lightgray,colframe=accent,title=Solution Part 2A.2: Simplify Quadratic Formula]
        \footnotesize
        For $0 = -3t^2 + 18t - 15$, we have $a=-3$, $b=18$, $c=-15$.
        \begin{itemize}
            \item Continuing the simplification:
                \begin{align*}
                    t &= \frac{-18 \pm \sqrt{324 - 180}}{-6} \\
                    t &= \frac{-18 \pm \sqrt{144}}{-6}
                \end{align*}
        \end{itemize}
    \end{tcolorbox}
\end{frame}

\begin{frame}{Word Problem: Rocket Launch (Solution Part 2B.1)}
    \begin{tcolorbox}[colback=lightgray,colframe=accent,title=Solution Part 2B.1: Calculate Values]
        \footnotesize
        For $0 = -3t^2 + 18t - 15$, we have $a=-3$, $b=18$, $c=-15$.
        \begin{itemize}
            \item Continuing the calculation:
                \begin{align*}
                    t &= \frac{-18 \pm 12}{-6}
                \end{align*}
        \end{itemize}
    \end{tcolorbox}
\end{frame}

\begin{frame}{Word Problem: Rocket Launch (Solution Part 2B.2)}
    \begin{tcolorbox}[colback=lightgray,colframe=accent,title=Solution Part 2B.2: Final Conclusion]
        \footnotesize
        For $0 = -3t^2 + 18t - 15$, we have $a=-3$, $b=18$, $c=-15$.
        \begin{itemize}
            \item Two possible values for $t$:
                \begin{align*}
                    t_1 &= \frac{-18 + 12}{-6} = \frac{-6}{-6} = 1 \text{ second (on the way up)} \\
                    t_2 &= \frac{-18 - 12}{-6} = \frac{-30}{-6} = 5 \text{ seconds (on the way down)}
                \end{align*}
            \item The tracking device should activate on its way down after approximately \textbf{5 seconds}.
        \end{itemize}
    \end{tcolorbox}
\end{frame}

% Section III: Where does the QF come From? (Derivation)
\section{Derivation of the Quadratic Formula}

\begin{frame}{III) Where Does the Quadratic Formula Come From?}
    \begin{tcolorbox}[colback=lightgray,colframe=primary,title=Derivation using Completing the Square]
        \footnotesize
        We will derive the Quadratic Formula by taking the standard quadratic equation and applying the method of \textbf{Completing the Square}. Then, we will isolate "$x$".

        Starting with the standard form:
        \[ ax^2 + bx + c = 0 \]
        \textit{Follow the next slides for step-by-step derivation.}
    \end{tcolorbox}
\end{frame}

\begin{frame}{Derivation Step 1: Isolate Constant Term}
    \begin{tcolorbox}[colback=lightgray,colframe=accent,title=Step 1]
        \footnotesize
        Move the constant term "$c$" to the right side of the equation:
        \begin{align*}
            ax^2 + bx + c &= 0 \\
            ax^2 + bx &= -c
        \end{align*}
    \end{tcolorbox}
\end{frame}

\begin{frame}{Derivation Step 2: Divide by "$a$"}
    \begin{tcolorbox}[colback=lightgray,colframe=accent,title=Step 2]
        \footnotesize
        Divide the entire equation by the coefficient "$a$" (since $a \neq 0$):
        \begin{align*}
            \frac{ax^2}{a} + \frac{bx}{a} &= \frac{-c}{a} \\
            x^2 + \frac{b}{a}x &= -\frac{c}{a}
        \end{align*}
    \end{tcolorbox}
\end{frame}

\begin{frame}{Derivation Step 3: Complete the Square}
    \begin{tcolorbox}[colback=lightgray,colframe=accent,title=Step 3]
        \footnotesize
        To complete the square on the left side, take half of the coefficient of "$x$" (which is $\frac{b}{a}$), square it, and add it to both sides.
        Half of $\frac{b}{a}$ is $\frac{b}{2a}$. Squaring it gives $\left(\frac{b}{2a}\right)^2 = \frac{b^2}{4a^2}$.
        \begin{align*}
            x^2 + \frac{b}{a}x + \left(\frac{b}{2a}\right)^2 &= -\frac{c}{a} + \left(\frac{b}{2a}\right)^2 \\
            x^2 + \frac{b}{a}x + \frac{b^2}{4a^2} &= -\frac{c}{a} + \frac{b^2}{4a^2}
        \end{align*}
    \end{tcolorbox}
\end{frame}

\begin{frame}{Derivation Step 4: Factor and Combine Terms}
    \begin{tcolorbox}[colback=lightgray,colframe=accent,title=Step 4]
        \footnotesize
        Factor the perfect square trinomial on the left side and combine the terms on the right side by finding a common denominator ($4a^2$):
        \begin{align*}
            \left(x + \frac{b}{2a}\right)^2 &= -\frac{c}{a} \cdot \frac{4a}{4a} + \frac{b^2}{4a^2} \\
            \left(x + \frac{b}{2a}\right)^2 &= -\frac{4ac}{4a^2} + \frac{b^2}{4a^2} \\
            \left(x + \frac{b}{2a}\right)^2 &= \frac{b^2 - 4ac}{4a^2}
        \end{align*}
    \end{tcolorbox}
\end{frame}

\begin{frame}{Derivation Step 5: Take the Square Root}
    \begin{tcolorbox}[colback=lightgray,colframe=accent,title=Step 5]
        \footnotesize
        Take the square root of both sides. Remember to include the $\pm$ sign on the right side:
        \begin{align*}
            \sqrt{\left(x + \frac{b}{2a}\right)^2} &= \pm\sqrt{\frac{b^2 - 4ac}{4a^2}} \\
            x + \frac{b}{2a} &= \pm\frac{\sqrt{b^2 - 4ac}}{\sqrt{4a^2}} \\
            x + \frac{b}{2a} &= \pm\frac{\sqrt{b^2 - 4ac}}{2|a|}
        \end{align*}
        Note: We can use $2a$ instead of $2|a|$ because the $\pm$ already accounts for both positive and negative cases.
    \end{tcolorbox}
\end{frame}

\begin{frame}{Derivation Step 6: Isolate "$x$"}
    \begin{tcolorbox}[colback=lightgray,colframe=accent,title=Step 6]
        \footnotesize
        Subtract $\frac{b}{2a}$ from both sides to isolate "$x$":
        \begin{align*}
            x &= -\frac{b}{2a} \pm\frac{\sqrt{b^2 - 4ac}}{2a} \\
            x &= \frac{-b \pm \sqrt{b^2 - 4ac}}{2a}
        \end{align*}
        This is the Quadratic Formula.
    \end{tcolorbox}
\end{frame}

\end{document} 