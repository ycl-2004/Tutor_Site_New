\documentclass[aspectratio=169]{beamer}
\usepackage{amsmath}
\usepackage{amssymb}
\usepackage{graphicx}
\usepackage{tcolorbox}
\usepackage{booktabs}
\usepackage{colortbl}
\usepackage{xcolor}

% Custom colors
\definecolor{primary}{RGB}{41, 128, 185}
\definecolor{secondary}{RGB}{52, 152, 219}
\definecolor{accent}{RGB}{231, 76, 60}
\definecolor{lightgray}{RGB}{236, 240, 241}

% Theme customization
\usetheme{Madrid}
\usecolortheme{whale}
\setbeamercolor{structure}{fg=primary}
\setbeamercolor{background canvas}{bg=white}
\setbeamercolor{normal text}{fg=black}

% Title page info
\title{Pre-Calculus 11}
\subtitle{Lesson 3: Graphing Quadratic Functions}
\author{Created by Yi-Chen Lin}
\date{\today}

\begin{document}

\begin{frame}
    \titlepage
\end{frame}

% REVIEW: LINEAR AND QUADRATIC FUNCTIONS
\begin{frame}{REVIEW: LINEAR AND QUADRATIC FUNCTIONS}
    \begin{tcolorbox}[colback=lightgray,colframe=primary,title=Key Concepts]
        \footnotesize
        \begin{itemize}
            \item \textbf{Linear Functions}
            \begin{itemize}
                \item Straight Lines
                \item General Form: $y = mx + b$
                \item Highest degree for "x" is one
                \item m: Slope
                \item b: Y-intercept
            \end{itemize}
            \item \textbf{Quadratic Functions}
            \begin{itemize}
                \item Curved, Shape of a "Parabola"
                \item Highest Degree for "x" is two
                \item General Form: $y = ax^2 + bx + c$, where $a, b, c$ are real numbers
                \item Example: $y = x^2 + x - 10$ or $y = -x^2 - 4x + 10$
            \end{itemize}
        \end{itemize}
    \end{tcolorbox}
\end{frame}

% I) WHY IS A QUADRATIC FUNCTION U-SHAPED?
\begin{frame}{I) WHY IS A QUADRATIC FUNCTION U-SHAPED?}
    \begin{tcolorbox}[colback=lightgray,colframe=primary,title=Understanding the Parabola Shape]
        \footnotesize
        \begin{itemize}
            \item If we make a Table of Values (TOV), plot the coordinates, and connect the dots, the resulting shape is a Parabola.
        \end{itemize}
        
        \textbf{Example TOV for $y = x^2$}
        \begin{center}
        \begin{tabular}{|c|c|}
            \hline
            \textbf{x} & \textbf{y} \\
            \hline
            -3 & 9 \\
            -2 & 4 \\
            -1 & 1 \\
            0 & 0 \\
            1 & 1 \\
            2 & 4 \\
            3 & 9 \\
            \hline
        \end{tabular}
        \end{center}
        \vspace{0.5em}
        \textit{(Graph will be on the next page)}
    \end{tcolorbox}
\end{frame}

\begin{frame}{Graph of $y = x^2$}
    \begin{tcolorbox}[colback=lightgray,colframe=primary,title=Visual Representation]
        \centering
        \includegraphics[width=0.7\textwidth]{y_xpower2.png}
    \end{tcolorbox}
\end{frame}

% II) COMPONENTS OF A PARABOLA
\begin{frame}{II) COMPONENTS OF A PARABOLA}
    \begin{tcolorbox}[colback=lightgray,colframe=primary,title=Key Components]
        \footnotesize
        \begin{itemize}
            \item \textbf{Vertex:} The coordinates at either the top (maximum) or bottom (minimum) of the parabola. Always in the middle.
            \item \textbf{Axis of Symmetry (AOS):} A vertical line that cuts the graph in the middle. Must be an equation (e.g., $x = k$).
            \item \textbf{X-intercepts:} Intersection points between the parabola and the x-axis (where $y=0$). Can have zero, one, or two x-intercepts. Represented as $(x_1, 0)$ and $(x_2, 0)$.
            \item \textbf{Y-intercept:} Intersection point between the parabola and the y-axis (where $x=0$). Always one y-intercept. Represented as $(0, y_0)$.
        \end{itemize}
        \vspace{0.5em}
        \textit{(Insert diagram of a parabola with labeled components: Vertex, Axis of Symmetry, X-intercepts, Y-intercept)}
    \end{tcolorbox}
\end{frame}

\begin{frame}{Ex: Given each parabola, indicate the components}
    \begin{tcolorbox}[colback=lightgray,colframe=primary,title=Example 1 - Graph]
        \footnotesize
        \textbf{Parabola 1: (Opening Up)}\
        \centering
        \includegraphics[width=0.7\textwidth]{given_para1.png}
    \end{tcolorbox}
\end{frame}

\begin{frame}{Ex: Given each parabola, indicate the components - Example 1 Details}
    \begin{tcolorbox}[colback=lightgray,colframe=primary,title=Example 1 - Components]
        \footnotesize
        \textbf{Parabola 1: (Opening Up)}
        
        \begin{itemize}
            \item \textbf{Vertex:} (1, -3)
            \item \textbf{Axis of Symmetry:} $x = 1$
            \item \textbf{X-intercepts:} (-0.5, 0) and (2.5, 0)
            \item \textbf{Y-intercept:} (0, -2.5)
        \end{itemize}
    \end{tcolorbox}
\end{frame}

\begin{frame}{Ex: Given each parabola, indicate the components}
    \begin{tcolorbox}[colback=lightgray,colframe=primary,title=Example 2 - Graph]
        \footnotesize
        \textbf{Parabola 2: (Opening Down)}\
        \centering
        \includegraphics[width=0.7\textwidth]{given_para2.png}
    \end{tcolorbox}
\end{frame}

\begin{frame}{Ex: Given each parabola, indicate the components - Example 2 Details}
    \begin{tcolorbox}[colback=lightgray,colframe=primary,title=Example 2 - Components]
        \footnotesize
        \textbf{Parabola 2: (Opening Down)}
        
        \begin{itemize}
            \item \textbf{Vertex:} (-2, 6)
            \item \textbf{Axis of Symmetry:} $x = -2$
            \item \textbf{X-intercepts:} (-4, 0) and (0, 0)
            \item \textbf{Y-intercept:} (0, 0)
        \end{itemize}
    \end{tcolorbox}
\end{frame}

% GRAPHING PARABOLAS WITH XAVIER'S METHOD
\begin{frame}{GRAPHING PARABOLAS WITH XAVIER'S METHOD}
    \begin{tcolorbox}[colback=lightgray,colframe=primary,title=Xavier's Method Steps]
        \footnotesize
        \begin{itemize}
            \item First find the vertex using \textcolor{accent}{\textbf{X.A.V.}}
            \begin{itemize}
                \item \textbf{X:} x-intercepts by factoring (set $y=0$)
                \item \textbf{A:} Axis of Symmetry (average of x-intercepts: $x = \frac{x_1 + x_2}{2}$)
                \item \textbf{V:} Vertex (substitute AOS x-value into the original equation to find y-coordinate)
            \end{itemize}
            \item Use the constant "a" from $y=ax^2+bx+c$ to determine which way the graph opens:
            \begin{itemize}
                \item If $a > 0$ (positive), graph Opens Up (U-shape)
                \item If $a < 0$ (negative), graph Opens Down (inverted U-shape)
            \end{itemize}
            \item Plot a couple of extra points for a better graph (e.g., y-intercept and its symmetric point).
            \item \textbf{Note:} For Quadratic Functions that do not have x-intercepts, we will learn to graph them in the next section.
        \end{itemize}
    \end{tcolorbox}
\end{frame}

% EX: FIND THE X INTERCEPTS, AXIS OF SYMMETRY, VERTEX AND GRAPH y = x^2 - x - 6
\begin{frame}{EX: FIND THE X INTERCEPTS, AOS, VERTEX, AND GRAPH $y = x^2 - 4x - 12$}
    \begin{tcolorbox}[colback=lightgray,colframe=primary,title=Step-by-Step Solution]
        \footnotesize
        Given: $y = x^2 - 4x - 12$
        
        \begin{enumerate}
            \item \textbf{Find X-intercepts (by factoring):}
            Set $y=0$:
            \begin{align*}
                x^2 - 4x - 12 &= 0 \\
                (x - 6)(x + 2) &= 0
            \end{align*}
            So, $x - 6 = 0 \Rightarrow x = 6$ (X-intercept: $(6,0)$)
            And, $x + 2 = 0 \Rightarrow x = -2$ (X-intercept: $(-2,0)$)
            
            \item \textbf{Find Axis of Symmetry (AOS) (Equation):}
            Average of x-intercepts:
            \begin{align*}
                x &= \frac{6 + (-2)}{2} \\
                x &= \frac{4}{2} \\
                x &= 2
            \end{align*}
            AOS: $x = 2$
        \end{enumerate}
    \end{tcolorbox}
\end{frame}

\begin{frame}{EX: FIND THE X INTERCEPTS, AOS, VERTEX, AND GRAPH $y = x^2 - 4x - 12$}
    \begin{tcolorbox}[colback=lightgray,colframe=primary,title=Step-by-Step Solution (Cont.)]
        \footnotesize
        Given: $y = x^2 - 4x - 12$
        
        \begin{enumerate}
            \setcounter{enumi}{2}
            \item \textbf{Find Vertex (Coordinates):}
            Substitute AOS $x=2$ into the original equation:
            \begin{align*}
                y &= (2)^2 - 4(2) - 12 \\
                y &= 4 - 8 - 12 \\
                y &= -16
            \end{align*}
            Vertex: $(2, -16)$
            
            \item \textbf{Determine opening direction:}
            Since $a=1$ (positive), the parabola opens UP.
        \end{enumerate}
    \end{tcolorbox}
\end{frame}

\begin{frame}{EX: FIND THE X INTERCEPTS, AOS, VERTEX, AND GRAPH $y = x^2 - 4x - 12$}
    \begin{tcolorbox}[colback=lightgray,colframe=primary,title=Graph - $y = x^2 - 4x - 12$]
        \footnotesize
        \textbf{Points to plot:}\
        \begin{itemize}
            \item X-intercepts: $(-2,0)$ and $(6,0)$
            \item Vertex: $(2, -16)$
            \item Y-intercept: Set $x=0$, $y = (0)^2 - 4(0) - 12 = -12$. So $(0,-12)$.
            \item Additional points (symmetric to y-intercept): Since AOS is $x=2$, point symmetric to $(0,-12)$ is at $x=4$ (distance of 2 from AOS). For $x=4$, $y = (4)^2 - 4(4) - 12 = 16 - 16 - 12 = -12$. So $(4,-12)$.
        \end{itemize}
        \vspace{0.5em}
        \textit{(Graph will be on the next page)}
        
        \textbf{Domain and Range:}\
        \begin{itemize}
            \item \textbf{Domain (D):} $x \in \mathbb{R}$ (All real numbers)
            \item \textbf{Range (R):} $y \ge -16$ (Since it opens up, y-values are greater than or equal to the vertex's y-coordinate)
        \end{itemize}
    \end{tcolorbox}
\end{frame}

\begin{frame}{Graph for $y = x^2 - 4x - 12$}
    \begin{tcolorbox}[colback=lightgray,colframe=primary,title=Visual Representation]
        \centering
        \includegraphics[width=0.7\textwidth]{y_xp_4x_12.png}
    \end{tcolorbox}
\end{frame}

% PRACTICE: FIND THE X INTERCEPTS, AXIS OF SYMMETRY, VERTEX AND GRAPH y = 2x^2 + 3x - 9
\begin{frame}{PRACTICE: FIND THE X INTERCEPTS, AOS, VERTEX, AND GRAPH $y = 3x^2 + 5x - 2$}
    \begin{tcolorbox}[colback=lightgray,colframe=primary,title=Problem]
        \footnotesize
        Find the X-intercepts, Axis of Symmetry, Vertex and Graph the following:\
        $y = 3x^2 + 5x - 2$
    \end{tcolorbox}
\end{frame}

\begin{frame}{PRACTICE: Solution $y = 3x^2 + 5x - 2$ (Part 1)}
    \begin{tcolorbox}[colback=lightgray,colframe=accent,title=Detailed Solution]
        \footnotesize
        Given: $y = 3x^2 + 5x - 2$
        
        \begin{enumerate}
            \item \textbf{Find X-intercepts (by factoring):}
            Set $y=0$:
            \begin{align*}
                3x^2 + 5x - 2 &= 0 \\
                (3x - 1)(x + 2) &= 0
            \end{align*}
            So, $3x - 1 = 0 \Rightarrow 3x = 1 \Rightarrow x = \frac{1}{3}$ (X-intercept: $(\frac{1}{3},0)$)\
            And, $x + 2 = 0 \Rightarrow x = -2$ (X-intercept: $(-2,0)$)
        \end{enumerate}
    \end{tcolorbox}
\end{frame}

\begin{frame}{PRACTICE: Solution $y = 3x^2 + 5x - 2$ (Part 2)}
    \begin{tcolorbox}[colback=lightgray,colframe=accent,title=Detailed Solution (Part 2)]
        \footnotesize
        Given: $y = 3x^2 + 5x - 2$
        
        \begin{enumerate}
            \setcounter{enumi}{1}
            \item \textbf{Find Axis of Symmetry (AOS) (Equation):}
            Average of x-intercepts:
            \begin{align*}
                x &= \frac{\frac{1}{3} + (-2)}{2} \\
                x &= \frac{\frac{1}{3} - \frac{6}{3}}{2} \\
                x &= \frac{-\frac{5}{3}}{2} \\
                x &= -\frac{5}{6}
            \end{align*}
            AOS: $x = -\frac{5}{6}$
        \end{enumerate}
    \end{tcolorbox}
\end{frame}

\begin{frame}{PRACTICE: Solution $y = 3x^2 + 5x - 2$ (Part 3)}
    \begin{tcolorbox}[colback=lightgray,colframe=accent,title=Detailed Solution (Cont.)]
        \footnotesize
        Given: $y = 3x^2 + 5x - 2$
        
        \begin{enumerate}
            \setcounter{enumi}{2}
            \item \textbf{Find Vertex (Coordinates):}
            Substitute AOS $x=-\frac{5}{6}$ into the original equation:
            \begin{align*}
                y &= 3(-\frac{5}{6})^2 + 5(-\frac{5}{6}) - 2 \\
                y &= 3(\frac{25}{36}) - \frac{25}{6} - 2 \\
                y &= \frac{25}{12} - \frac{50}{12} - \frac{24}{12} \\
                y &= \frac{25 - 50 - 24}{12} \\
                y &= -\frac{49}{12}
            \end{align*}
            Vertex: $(-\frac{5}{6}, -\frac{49}{12})$
            
            \item \textbf{Determine opening direction:}
            Since $a=3$ (positive), the parabola opens UP.
        \end{enumerate}
    \end{tcolorbox}
\end{frame}

\begin{frame}{PRACTICE: Graph $y = 3x^2 + 5x - 2$}
    \begin{tcolorbox}[colback=lightgray,colframe=primary,title=Graph - $y = 3x^2 + 5x - 2$]
        \footnotesize
        \textbf{Points to plot:}\
        \begin{itemize}
            \item X-intercepts: $(-2,0)$ and $(\frac{1}{3},0)$
            \item Vertex: $(-\frac{5}{6}, -\frac{49}{12})$
            \item Y-intercept: Set $x=0$, $y = 3(0)^2 + 5(0) - 2 = -2$. So $(0,-2)$.
            \item Additional points (symmetric to y-intercept): Since AOS is $x=-\frac{5}{6}$, point symmetric to $(0,-2)$ is at $x = -\frac{5}{6} + (0 - (-\frac{5}{6})) = -\frac{5}{6} + \frac{5}{6} = -\frac{10}{6} = -\frac{5}{3}$. For $x=-\frac{5}{3}$, $y = 3(-\frac{5}{3})^2 + 5(-\frac{5}{3}) - 2 = 3(\frac{25}{9}) - \frac{25}{3} - 2 = \frac{25}{3} - \frac{25}{3} - 2 = -2$. So $(-\frac{5}{3},-2)$.
        \end{itemize}
        \vspace{0.5em}
        \textit{(Graph will be on the next page)}
        
        \textbf{Domain and Range:}\
        \begin{itemize}
            \item \textbf{Domain (D):} $x \in \mathbb{R}$ (All real numbers)
            \item \textbf{Range (R):} $y \ge -\frac{49}{12}$ (Since it opens up, y-values are greater than or equal to the vertex's y-coordinate)
        \end{itemize}
    \end{tcolorbox}
\end{frame}

\begin{frame}{Graph for $y = 3x^2 + 5x - 2$}
    \begin{tcolorbox}[colback=lightgray,colframe=primary,title=Visual Representation]
        \centering
        \includegraphics[width=0.7\textwidth]{y_3xp_5x_2.png}
    \end{tcolorbox}
\end{frame}

% REVIEW: DOMAIN AND RANGE
\begin{frame}{REVIEW: DOMAIN AND RANGE}
    \begin{tcolorbox}[colback=lightgray,colframe=primary,title=Definitions]
        \footnotesize
        \begin{itemize}
            \item \textbf{Domain:} The collection of all possible X-values (input) that a function can have. Look at the graph horizontally.
            \item \textbf{Range:} The collection of all possible Y-values (output) that a function can have. Look at the graph vertically.
        \end{itemize}
        
        \textbf{Example for finding Domain:}\
        \textit{(Insert image of a graph that starts at x=-2 and continues to the right, e.g., a ray)}\
        \begin{itemize}
            \item This graph starts at $x=-2$ and continues to the right.
            \item \textbf{Domain:} $x \ge -2$
        \end{itemize}
        
        \textbf{Example for finding Range:}\
        \textit{(Insert image of a graph with a lowest point at y=0, and then goes up, e.g., an upward opening parabola fragment or a V-shape)}\
        \begin{itemize}
            \item The lowest point of this graph is at $y=0$, and then it goes up.
            \item \textbf{Range:} $y \ge 0$
        \end{itemize}
    \end{tcolorbox}
\end{frame}

% DOMAIN AND RANGE OF PARABOLAS:
\begin{frame}{DOMAIN AND RANGE OF PARABOLAS:}
    \begin{tcolorbox}[colback=lightgray,colframe=primary,title=Key Rules for Parabolas]
        \footnotesize
        \begin{itemize}
            \item Since a parabola that opens up or down extends infinitely to the left and right, it can take on any x-value.
            \item So, the \textbf{Domain} of a parabola is always: \textbf{All Real Numbers} ($x \in \mathbb{R}$).
            \item The \textbf{Range} of a parabola depends on which way the graph opens and the y-coordinate of its vertex:
            \begin{itemize}
                \item If it opens \textbf{up} ($a > 0$), range will be: $y \ge \text{lowest y-value (vertex y-coordinate)}$
                \item If it opens \textbf{down} ($a < 0$), range will be: $y \le \text{highest y-value (vertex y-coordinate)}$
            \end{itemize}
        \end{itemize}
        \vspace{0.5em}
        \textit{(Insert diagram of an upward opening parabola with Domain: $x \in \mathbb{R}$ and Range: $y \ge -3$ labeled)}
    \end{tcolorbox}
\end{frame}

\begin{frame}{drex1: Parabola 1 Graph}
    \begin{tcolorbox}[colback=lightgray,colframe=primary,title=Visual Representation]
        \centering
        \includegraphics[width=0.7\textwidth]{drex1.png}
    \end{tcolorbox}
\end{frame}

\begin{frame}{Ex: Indicate the domain and range for Parabola 1}
    \begin{tcolorbox}[colback=lightgray,colframe=primary,title=Problem]
        \footnotesize
        Given the parabola, indicate its domain and range.\
        \textbf{Parabola 1: (Opening Up, Vertex at (2, -5))}
        \textit{(Refer to previous page for graph)}
    \end{tcolorbox}
\end{frame}

\begin{frame}{Ex: Indicate the domain and range for Parabola 1 - Solution}
    \begin{tcolorbox}[colback=lightgray,colframe=accent,title=Detailed Solution]
        \footnotesize
        For Parabola 1:\
        \begin{itemize}
            \item \textbf{Domain:} $x \in \mathbb{R}$
            \item \textbf{Range:} $y \ge -5$
        \end{itemize}
    \end{tcolorbox}
\end{frame}

\begin{frame}{drex2: Parabola 2 Graph}
    \begin{tcolorbox}[colback=lightgray,colframe=primary,title=Visual Representation]
        \centering
        \includegraphics[width=0.7\textwidth]{drex2.png}
    \end{tcolorbox}
\end{frame}

\begin{frame}{Ex: Indicate the domain and range for Parabola 2}
    \begin{tcolorbox}[colback=lightgray,colframe=primary,title=Problem]
        \footnotesize
        Given the parabola, indicate its domain and range.\
        \textbf{Parabola 2: (Opening Down, Vertex at (3, 7))}
        \textit{(Refer to previous page for graph)}
    \end{tcolorbox}
\end{frame}

\begin{frame}{Ex: Indicate the domain and range for Parabola 2 - Solution}
    \begin{tcolorbox}[colback=lightgray,colframe=accent,title=Detailed Solution]
        \footnotesize
        For Parabola 2:\
        \begin{itemize}
            \item \textbf{Domain:} $x \in \mathbb{R}$
            \item \textbf{Range:} $y \le 7$
        \end{itemize}
    \end{tcolorbox}
\end{frame}

\begin{frame}{drex3: Parabola 3 Graph}
    \begin{tcolorbox}[colback=lightgray,colframe=primary,title=Visual Representation]
        \centering
        \includegraphics[width=0.7\textwidth]{drex3.png}
    \end{tcolorbox}
\end{frame}

\begin{frame}{Ex: Indicate the domain and range for Parabola 3}
    \begin{tcolorbox}[colback=lightgray,colframe=primary,title=Problem]
        \footnotesize
        Given the parabola, indicate its domain and range.\
        \textbf{Parabola 3: (Opening Up, Vertex at (-1, 0))}
        \textit{(Refer to previous page for graph)}
    \end{tcolorbox}
\end{frame}

\begin{frame}{Ex: Indicate the domain and range for Parabola 3 - Solution}
    \begin{tcolorbox}[colback=lightgray,colframe=accent,title=Detailed Solution]
        \footnotesize
        For Parabola 3:\
        \begin{itemize}
            \item \textbf{Domain:} $x \in \mathbb{R}$
            \item \textbf{Range:} $y \ge 0$
        \end{itemize}
    \end{tcolorbox}
\end{frame}

% EX: MATCH EACH EQUATION WITH THE DESCRIPTION
\begin{frame}{Ex: Match each equation with the description}
    \begin{tcolorbox}[colback=lightgray,colframe=primary,title=Match the following]
        \footnotesize
        \begin{columns}
            \begin{column}{0.5\textwidth}
                \begin{itemize}
                    \item[] \textbf{Equations:}
                    \item[a)] $y = x^2 - 9$
                    \item[b)] $y = -2x^2 + 4x + 6$
                    \item[c)] $y = x^2 + 4x + 3$
                \end{itemize}
            \end{column}
            \begin{column}{0.5\textwidth}
                \begin{itemize}
                    \item[] \textbf{Descriptions:}
                    \item[i)] Range is $y \ge -1$, Axis of symmetry is $x = -2$
                    \item[ii)] X-intercepts at 3 and -3, Axis of symmetry is the Y-axis
                    \item[iii)] Range is $y \le 8$, Axis of symmetry is $x = 1$
                    \item[iv)] Graph opens up and vertex at (4,-5)
                \end{itemize}
            \end{column}
        \end{columns}
    \end{tcolorbox}
\end{frame}

\begin{frame}{Ex: Match each equation with the description - Solution}
    \begin{tcolorbox}[colback=lightgray,colframe=accent,title=Detailed Solutions]
        \footnotesize
        \begin{enumerate}
            \item[a)] $y = x^2 - 9$
            \begin{itemize}
                \item Vertex: $(0,-9)$
                \item AOS: $x=0$
                \item X-intercepts: $x^2-9=0 \Rightarrow x^2=9 \Rightarrow x = \pm 3$. So $(3,0), (-3,0)$.
                \item Range: $y \ge -9$ (opens up)
                \item Matches \textbf{ii) X-intercepts at 3 and -3, Axis of symmetry is the Y-axis}
            \end{itemize}
            \item[b)] $y = -2x^2 + 4x + 6$
            \begin{itemize}
                \item Opens down ($a=-2$)
                \item AOS: $x = \frac{-4}{2(-2)} = \frac{-4}{-4} = 1$. So $x=1$.
                \item Vertex y-coord: $y = -2(1)^2 + 4(1) + 6 = -2 + 4 + 6 = 8$. Vertex: $(1,8)$.
                \item Range: $y \le 8$
                \item Matches \textbf{iii) Range is $y \le 8$, Axis of symmetry is $x = 1$}
            \end{itemize}
        \end{enumerate}
    \end{tcolorbox}
\end{frame}

\begin{frame}{Ex: Match each equation with the description - Solution (Cont.)}
    \begin{tcolorbox}[colback=lightgray,colframe=accent,title=Detailed Solutions (Cont.)]
        \footnotesize
        \begin{enumerate}
            \setcounter{enumi}{2}
            \item[c)] $y = x^2 + 4x + 3$
            \begin{itemize}
                \item Opens up ($a=1$)
                \item AOS: $x = \frac{-4}{2(1)} = \frac{-4}{2} = -2$. So $x=-2$.
                \item Vertex y-coord: $y = (-2)^2 + 4(-2) + 3 = 4 - 8 + 3 = -1$. Vertex: $(-2,-1)$.
                \item Range: $y \ge -1$
                \item Matches \textbf{i) Range is $y \ge -1$, Axis of symmetry is $x = -2$}
            \end{itemize}
            
            \item[] Description \textbf{iv) Graph opens up and vertex at (4,-5)} does not match any equation provided.
        \end{enumerate}
    \end{tcolorbox}
\end{frame}

% THINGS TO REMEMBER:
\begin{frame}{THINGS TO REMEMBER:}
    \begin{tcolorbox}[colback=lightgray,colframe=primary,title=Key Reminders]
        \footnotesize
        \begin{itemize}
            \item The \textbf{vertex} is always in the middle between the two X-intercepts (if they exist).
            \item The \textbf{Axis of Symmetry} must always be an equation: $x = k$, where "k" is the x-coordinate of the vertex.
            \item The \textbf{domain} of a parabola that opens up or down will always be: $x \in \mathbb{R}$ (All Real Numbers).
            \item To find the \textbf{Y-coordinate of the vertex}, plug the x-value of the AOS into the original equation and solve for "y".
            \item The vertex, x-intercepts, and y-intercept should be provided as a \textbf{pair of coordinates} (a,b).
            \item The y-coordinate of the Vertex will be used for the \textbf{range} of the function.
        \end{itemize}
    \end{tcolorbox}
\end{frame}

% WHAT DO YOU DO IF YOU CAN'T FACTOR THE TRINOMIAL?
\begin{frame}{WHAT DO YOU DO IF YOU CAN'T FACTOR THE TRINOMIAL?}
    \begin{tcolorbox}[colback=lightgray,colframe=primary,title=Next Steps]
        \footnotesize
        \begin{itemize}
            \item \textbf{Answer:} Use the Quadratic Formula (next lesson)!
            \item \textbf{Quadratic Formula:}
            \begin{align*}
                x &= \frac{-b \pm \sqrt{b^2 - 4ac}}{2a}
            \end{align*}
        \end{itemize}
    \end{tcolorbox}
\end{frame}

\end{document} 