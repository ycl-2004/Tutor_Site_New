\documentclass[aspectratio=169]{beamer}

% Theme and colors
\usetheme{Madrid}
\usecolortheme{whale}
% Custom colors
\definecolor{primary}{RGB}{41, 128, 185}
\definecolor{secondary}{RGB}{52, 152, 219}
\definecolor{accent}{RGB}{231, 76, 60}
\definecolor{lightgray}{RGB}{236, 240, 241}

% Packages
\usepackage[utf8]{inputenc}
\usepackage{graphicx}
\usepackage{amsmath}
\usepackage{amsfonts}
\usepackage{amssymb}
\usepackage{mathpazo} % Palatino font
\usepackage{multirow}
\usepackage{tcolorbox}
\usepackage{tikz} % For diagrams
\usepackage{caption} % For figure captions
\usepackage{adjustbox} % For adjusting box
\usepackage{float} % For better float control

% Beamer specific configurations
\setbeamercolor{structure}{fg=primary}
\setbeamercolor{background canvas}{bg=white}
\setbeamercolor{normal text}{fg=black}

% Information for the title page
\title{Lesson 6: Completing the Square \\& Converting to APQ Form}
\author{Yi-Chen Lin}
\date{June 10, 2025}

\begin{document}

% Title page
\begin{frame}
    \titlepage
\end{frame}

% Table of Contents
\begin{frame}{Table of Contents}
    \tableofcontents
\end{frame}

% Section I: Perfect Trinomials
\section{Perfect Trinomials}

\begin{frame}{I) Perfect Trinomials - Part 1}
    \begin{tcolorbox}[colback=lightgray,colframe=primary,title=Key Concepts]
        \footnotesize
        \begin{itemize}
            \item \textbf{When a perfect trinomial is factored, both binomials will be equal}
                \begin{itemize}
                    \item \(x^2 + 12x + 36 = (x+6)(x+6) = (x+6)^2\)
                    \item \(x^2 - 14x + 49 = (x-7)(x-7) = (x-7)^2\)
                    \item \(x^2 + 4x + 4 = (x+2)(x+2) = (x+2)^2\)
                \end{itemize}
        \end{itemize}
    \end{tcolorbox}
\end{frame}

\begin{frame}{I) Perfect Trinomials - Part 2}
    \begin{tcolorbox}[colback=lightgray,colframe=primary,title=Key Concepts (Cont.)]
        \footnotesize
        \begin{itemize}
            \item \textbf{The third term in a perfect trinomial is equal to the second term divided by 2 and then squared}
                \begin{itemize}
                    \item \(\left(\frac{12}{2}\right)^2 = 6^2 = 36\)
                    \item \(\left(\frac{-14}{2}\right)^2 = (-7)^2 = 49\)
                    \item \(\left(\frac{4}{2}\right)^2 = 2^2 = 4\)
                \end{itemize}
            \item \textbf{The term in the binomial is equal to the second term divided by 2}
                \begin{itemize}
                    \item \(\frac{12}{2} = 6\)
                \end{itemize}
            \item \textbf{When we CTS, we change the trinomial into a perfect trinomial}
        \end{itemize}
    \end{tcolorbox}
\end{frame}

% Section II: What is Completing the Square
\section{What is Completing the Square}

\begin{frame}{II) What is Completing the Square?}
    \begin{tcolorbox}[colback=lightgray,colframe=primary,title=Key Concept]
        \footnotesize
        \textbf{Completing the Square} is a process that changes a quadratic function from:
        \[ \textbf{Standard Form: } y = Ax^2 + Bx + C \]
        to
        \[ \textbf{Vertex Form: } y = a(x-p)^2 + q \]
    \end{tcolorbox}
\end{frame}

\begin{frame}{Example: Completing the Square (Part 1)}
    \begin{tcolorbox}[colback=lightgray,colframe=accent,title=Example: $y = x^2 - 8x + 10$]
        \footnotesize
        Starting with $y = x^2 - 8x + 10$:
        \begin{itemize}
            \item \textbf{Bracket the first two terms!}
                \begin{align*}
                    y &= (x^2 - 8x) + 10
                \end{align*}
            \item \textbf{Divide the second term by 2 and square it!}
                \item The purpose is to make the expression in the bracket into a perfect square!
                \begin{align*}
                    y &= (x^2 - 8x + 16) + 10 - 16
                \end{align*}
            \item \textbf{Take the negative square outside of the brackets!}
                \begin{align*}
                    y &= (x^2 - 8x + 16) + (10 - 16)
                \end{align*}
        \end{itemize}
    \end{tcolorbox}
\end{frame}

\begin{frame}{Example: Completing the Square (Part 2)}
    \begin{tcolorbox}[colback=lightgray,colframe=accent,title=Example: $y = x^2 - 8x + 10$ (Cont.)]
        \footnotesize
        Continuing from $y = (x^2 - 8x + 16) + (10 - 16)$:
        \begin{itemize}
            \item \textbf{The trinomial becomes two equal binomials}
                \begin{align*}
                    y &= (x-4)(x-4) - 6
                \end{align*}
            \item \textbf{Simplified to Vertex Form:}
                \begin{align*}
                    y &= (x-4)^2 - 6
                \end{align*}
            \item Now the equation is in vertex form: $a=1$, $p=4$, $q=-6$.
            \item Vertex: $(4,-6)$
        \end{itemize}
    \end{tcolorbox}
\end{frame}

\begin{frame}{Practice: Complete the Square and Find the Vertex}
    \begin{tcolorbox}[colback=lightgray,colframe=primary,title=Problem]
        \footnotesize
        Complete the square and find the vertex for: 
        \[ y = x^2 + 10x + 15 \]
    \end{tcolorbox}
\end{frame}

\begin{frame}{Practice: Complete the Square and Find the Vertex (Solution) - Part 1}
    \begin{tcolorbox}[colback=lightgray,colframe=accent,title=Solution - Part 1]
        \footnotesize
        For $y = x^2 + 10x + 15$:
        \begin{itemize}
            \item \textbf{Bracket the first two terms!}
                \begin{align*}
                    y &= (x^2 + 10x) + 15
                \end{align*}
            \item \textbf{Divide the second term by 2 and square it!}
                \item Purpose: Make the expression in the bracket into a perfect square!
                \begin{align*}
                    y &= (x^2 + 10x + 25) + 15 - 25
                \end{align*}
        \end{itemize}
    \end{tcolorbox}
\end{frame}

\begin{frame}{Practice: Complete the Square and Find the Vertex (Solution) - Part 2}
    \begin{tcolorbox}[colback=lightgray,colframe=accent,title=Solution - Part 2 (Cont.)]
        \footnotesize
        Continuing for $y = (x^2 + 10x + 25) + 15 - 25$:
        \begin{itemize}
            \item \textbf{Take the negative square outside of the brackets!}
                \begin{align*}
                    y &= (x^2 + 10x + 25) + (15 - 25)
                \end{align*}
            \item \textbf{The trinomial becomes two equal binomials}
                \begin{align*}
                    y &= (x+5)(x+5) - 10
                \end{align*}
            \item \textbf{Now the equation is in vertex form:}
                \begin{align*}
                    y &= (x+5)^2 - 10
                \end{align*}
            \item Here, $a=1$, $p=-5$, $q=-10$.
            \item \textbf{Vertex:} $(-5, -10)$
        \end{itemize}
    \end{tcolorbox}
\end{frame}

% Section III: CTS with a Leading Coefficient
\section{CTS with a Leading Coefficient}

\begin{frame}{III) CTS with a Leading Coefficient}
    \begin{tcolorbox}[colback=lightgray,colframe=primary,title=Key Steps]
        \footnotesize
        \begin{itemize}
            \item \textbf{Factor out any coefficient for $x^2$}
                \begin{align*}
                    y &= 3x^2 - 12x + 15 \\
                    y &= 3(x^2 - 4x) + 15
                \end{align*}
            \item \textbf{Divide the second term by 2 and square it!}
                \item This makes the expression in the bracket into a perfect square!
                \begin{align*}
                    y &= 3\left(x^2 - 4x + \left(\frac{-4}{2}\right)^2\right) + 15 - 3\left(\frac{-4}{2}\right)^2
                \end{align*}
            \item \textbf{Take the negative square outside of the brackets and multiply with coefficient in front!}
                \begin{align*}
                    y &= 3\left(x^2 - 4x + 4\right) + 15 - 3(4) \\
                    y &= 3(x^2 - 4x + 4) + 15 - 12
                \end{align*}
        \end{itemize}
    \end{tcolorbox}
\end{frame}

\begin{frame}{III) CTS with a Leading Coefficient (Cont.)}
    \begin{tcolorbox}[colback=lightgray,colframe=primary,title=Key Steps (Cont.)]
        \footnotesize
        Continuing for $y = 3(x^2 - 4x + 4) + 15 - 12$:
        \begin{itemize}
            \item \textbf{The trinomial becomes two equal binomials}
                \begin{align*}
                    y &= 3(x-2)(x-2) + 3
                \end{align*}
            \item \textbf{Now the equation is in vertex form:}
                \begin{align*}
                    y &= 3(x-2)^2 + 3
                \end{align*}
            \item Here, $a=3$, $p=2$, $q=3$.
            \item \textbf{Vertex:} $(2, 3)$
        \end{itemize}
    \end{tcolorbox}
\end{frame}

\begin{frame}{Practice: Convert to APQ Form (Problem 1)}
    \begin{tcolorbox}[colback=lightgray,colframe=primary,title=Problem 1]
        \footnotesize
        Convert the following equation to APQ Form:
        \[ y = 4x^2 + 8x - 5 \]
    \end{tcolorbox}
\end{frame}

\begin{frame}{Practice: Convert to APQ Form (Solution 1) - Part 1}
    \begin{tcolorbox}[colback=lightgray,colframe=accent,title=Solution 1 - Part 1]
        \footnotesize
        For $y = 4x^2 + 8x - 5$:
        \begin{itemize}
            \item \textbf{Factor out any coefficient for $x^2$}
                \begin{align*}
                    y &= 4(x^2 + 2x) - 5
                \end{align*}
            \item \textbf{Divide the second term by 2 and square it!}
                \begin{align*}
                    y &= 4\left(x^2 + 2x + \left(\frac{2}{2}\right)^2\right) - 5 - 4\left(\frac{2}{2}\right)^2 \\
                    y &= 4(x^2 + 2x + 1) - 5 - 4(1)
                \end{align*}
        \end{itemize}
    \end{tcolorbox}
\end{frame}

\begin{frame}{Practice: Convert to APQ Form (Solution 1) - Part 2}
    \begin{tcolorbox}[colback=lightgray,colframe=accent,title=Solution 1 - Part 2 (Cont.)]
        \footnotesize
        Continuing for $y = 4(x^2 + 2x + 1) - 5 - 4$:
        \begin{itemize}
            \item \textbf{The trinomial becomes two equal binomials}
                \begin{align*}
                    y &= 4(x+1)(x+1) - 9
                \end{align*}
            \item \textbf{Vertex Form:}
                \begin{align*}
                    y &= 4(x+1)^2 - 9
                \end{align*}
        \end{itemize}
    \end{tcolorbox}
\end{frame}

\begin{frame}{Practice: Convert to APQ Form (Problem 2)}
    \begin{tcolorbox}[colback=lightgray,colframe=primary,title=Problem 2]
        \footnotesize
        Convert the following equation to APQ Form:
        \[ y = \frac{1}{3}x^2 - 6x + 20 \]
    \end{tcolorbox}
\end{frame}

\begin{frame}{Practice: Convert to APQ Form (Solution 2) - Part 1}
    \begin{tcolorbox}[colback=lightgray,colframe=accent,title=Solution 2 - Part 1]
        \footnotesize
        For $y = \frac{1}{3}x^2 - 6x + 20$:
        \begin{itemize}
            \item \textbf{Factor out any coefficient for $x^2$}
                \begin{align*}
                    y &= \frac{1}{3}(x^2 - 18x) + 20
                \end{align*}
            \item \textbf{Divide the second term by 2 and square it!}
                \begin{align*}
                    y &= \frac{1}{3}(x^2 - 18x + (-9)^2) + 20 - \frac{1}{3}(-9)^2 \\
                    y &= \frac{1}{3}(x^2 - 18x + 81) + 20 - \frac{1}{3}(81)
                \end{align*}
        \end{itemize}
    \end{tcolorbox}
\end{frame}

\begin{frame}{Practice: Convert to APQ Form (Solution 2) - Part 2}
    \begin{tcolorbox}[colback=lightgray,colframe=accent,title=Solution 2 - Part 2 (Cont.)]
        \footnotesize
        Continuing for $y = \frac{1}{3}(x^2 - 18x + 81) + 20 - \frac{1}{3}(81)$:
        \begin{itemize}
            \item \textbf{Simplify constants}
                \begin{align*}
                    y &= \frac{1}{3}(x^2 - 18x + 81) + 20 - 27 \\
                    y &= \frac{1}{3}(x^2 - 18x + 81) - 7
                \end{align*}
            \item \textbf{The trinomial becomes two equal binomials}
                \begin{align*}
                    y &= \frac{1}{3}(x-9)^2 - 7
                \end{align*}
        \end{itemize}
    \end{tcolorbox}
\end{frame}

\begin{frame}{Practice: Convert to APQ Form (Problem 3)}
    \begin{tcolorbox}[colback=lightgray,colframe=primary,title=Problem 3]
        \footnotesize
        Convert the following equation to APQ Form:
        \[ y = -\frac{1}{4}x^2 + 2x - 1 \]
    \end{tcolorbox}
\end{frame}

\begin{frame}{Practice: Convert to APQ Form (Solution 3) - Part 1}
    \begin{tcolorbox}[colback=lightgray,colframe=accent,title=Solution 3 - Part 1]
        \footnotesize
        For $y = -\frac{1}{4}x^2 + 2x - 1$:
        \begin{itemize}
            \item \textbf{Factor out any coefficient for $x^2$}
                \begin{align*}
                    y &= -\frac{1}{4}(x^2 - 8x) - 1
                \end{align*}
            \item \textbf{Divide the second term by 2 and square it!}
                \begin{align*}
                    y &= -\frac{1}{4}(x^2 - 8x + (-4)^2) - 1 - \left(-\frac{1}{4}\right)(-4)^2 \\
                    y &= -\frac{1}{4}(x^2 - 8x + 16) - 1 - \left(-\frac{1}{4}\right)(16)
                \end{align*}
        \end{itemize}
    \end{tcolorbox}
\end{frame}

\begin{frame}{Practice: Convert to APQ Form (Solution 3) - Part 2}
    \begin{tcolorbox}[colback=lightgray,colframe=accent,title=Solution 3 - Part 2 (Cont.)]
        \footnotesize
        Continuing for $y = -\frac{1}{4}(x^2 - 8x + 16) - 1 - \left(-\frac{1}{4}\right)(16)$:
        \begin{itemize}
            \item \textbf{Simplify constants}
                \begin{align*}
                    y &= -\frac{1}{4}(x^2 - 8x + 16) - 1 + 4
                \end{align*}
            \item \textbf{The trinomial becomes two equal binomials}
                \begin{align*}
                    y &= -\frac{1}{4}(x-4)^2 + 3
                \end{align*}
            \item \textbf{Vertex Form:}
                \begin{align*}
                    y &= -\frac{1}{4}(x-4)^2 + 3
                \end{align*}
        \end{itemize}
    \end{tcolorbox}
\end{frame}

% Section IV: Solving Equations in APQ Form
\section{Solving Equations in APQ Form}

\begin{frame}{IV) Solving Equations in APQ Form - Part 1}
    \begin{tcolorbox}[colback=lightgray,colframe=primary,title=Key Steps - Part 1]
        \footnotesize
        When an equation is in APQ form, solving for the x-intercepts requires very little algebra.
        \begin{enumerate}
            \item \textbf{Step 1:} The "y" coordinate is zero at the x-intercept, so the "Y variable should be zero."
                \begin{align*}
                    y &= 2(x+3)^2 - 18 \\
                    0 &= 2(x+3)^2 - 18
                \end{align*}
            \item \textbf{Step 2:} Isolate the brackets with the exponent.
                \begin{align*}
                    18 &= 2(x+3)^2 \\
                    9 &= (x+3)^2
                \end{align*}
        \end{enumerate}
    \end{tcolorbox}
\end{frame}

\begin{frame}{IV) Solving Equations in APQ Form - Part 2}
    \begin{tcolorbox}[colback=lightgray,colframe=primary,title=Key Steps - Part 2 (Cont.)]
        \footnotesize
        Continuing from $(x+3)^2 = 9$:
        \begin{enumerate}
            \item \textbf{Step 3:} Square root both sides to get rid of the exponent. Remember: There are two answers! Positive \& negative!
                \begin{align*}
                    \pm\sqrt{9} &= x+3 \\
                    \pm 3 &= x+3
                \end{align*}
            \item \textbf{Step 4:} Solve for "x" by adding/subtracting the constant.
                \begin{align*}
                    x &= -3 \pm 3
                \end{align*}
            \item You now have two answers:
                \begin{itemize}
                    \item \(x_1 = -3 + 3 = 0\)
                    \item \(x_2 = -3 - 3 = -6\)
                \end{itemize}
        \end{enumerate}
    \end{tcolorbox}
\end{frame}

\begin{frame}{Practice: Solve for "x" (Problem 1)}
    \begin{tcolorbox}[colback=lightgray,colframe=primary,title=Problem 1]
        \footnotesize
        Solve for "x":
        \[ (x+8)^2 - 25 = 0 \]
    \end{tcolorbox}
\end{frame}

\begin{frame}{Practice: Solve for "x" (Solution 1) - Part 1}
    \begin{tcolorbox}[colback=lightgray,colframe=accent,title=Solution 1 - Part 1]
        \footnotesize
        For $(x+8)^2 - 25 = 0$:
        \begin{align*}
            (x+8)^2 &= 25 \\
            x+8 &= \pm\sqrt{25} \\
            x+8 &= \pm 5
        \end{align*}
    \end{tcolorbox}
\end{frame}

\begin{frame}{Practice: Solve for "x" (Solution 1) - Part 2}
    \begin{tcolorbox}[colback=lightgray,colframe=accent,title=Solution 1 - Part 2 (Cont.)]
        \footnotesize
        Continuing for $x+8 = \pm 5$:
        \begin{align*}
            x &= -8 \pm 5
        \end{align*}
        So, $x_1 = -8 + 5 = -3$ and $x_2 = -8 - 5 = -13$.
    \end{tcolorbox}
\end{frame}

\begin{frame}{Practice: Solve for "x" (Problem 2)}
    \begin{tcolorbox}[colback=lightgray,colframe=primary,title=Problem 2]
        \footnotesize
        Solve for "x":
        \[ 5(x-4)^2 - 45 = 0 \]
    \end{tcolorbox}
\end{frame}

\begin{frame}{Practice: Solve for "x" (Solution 2)}
    \begin{tcolorbox}[colback=lightgray,colframe=accent,title=Solution 2]
        \footnotesize
        For $5(x-4)^2 - 45 = 0$:
        \begin{align*}
            5(x-4)^2 &= 45 \\
            (x-4)^2 &= 9 \\
            x-4 &= \pm\sqrt{9} \\
            x-4 &= \pm 3 \\
            x &= 4 \pm 3
        \end{align*}
        So, $x_1 = 4 + 3 = 7$ and $x_2 = 4 - 3 = 1$.
    \end{tcolorbox}
\end{frame}

\begin{frame}{Practice: Solve for "x" (Problem 3)}
    \begin{tcolorbox}[colback=lightgray,colframe=primary,title=Problem 3]
        \footnotesize
        Solve for "x":
        \[ 3x^2 + 7 = 0 \]
    \end{tcolorbox}
\end{frame}

\begin{frame}{Practice: Solve for "x" (Solution 3)}
    \begin{tcolorbox}[colback=lightgray,colframe=accent,title=Solution 3]
        \footnotesize
        For $3x^2 + 7 = 0$:
        \begin{align*}
            3x^2 &= -7 \\
            x^2 &= -\frac{7}{3}
        \end{align*}
        Since a square cannot be negative, there are \textbf{No Real Solutions}.
    \end{tcolorbox}
\end{frame}

\begin{frame}{Practice: Solve for "x" by CTS (Problem 4)}
    \begin{tcolorbox}[colback=lightgray,colframe=primary,title=Problem 4]
        \footnotesize
        Solve for "x" by completing the square:
        \[ 0 = x^2 + 4x + 6 \]
    \end{tcolorbox}
\end{frame}

\begin{frame}{Practice: Solve for "x" by CTS (Solution 4) - Part 1}
    \begin{tcolorbox}[colback=lightgray,colframe=accent,title=Solution 4 - Part 1]
        \footnotesize
        For $0 = x^2 + 4x + 6$:
        \begin{itemize}
            \item \textbf{Bracket the first two terms!}
                \begin{align*}
                    0 &= (x^2 + 4x) + 6
                \end{align*}
            \item \textbf{Divide the second term by 2 and square it!}
                \begin{align*}
                    0 &= (x^2 + 4x + 4) + 6 - 4
                \end{align*}
        \end{itemize}
    \end{tcolorbox}
\end{frame}

\begin{frame}{Practice: Solve for "x" by CTS (Solution 4) - Part 2}
    \begin{tcolorbox}[colback=lightgray,colframe=accent,title=Solution 4 - Part 2 (Cont.)]
        \footnotesize
        Continuing for $0 = (x^2 + 4x + 4) + 6 - 4$:
        \begin{itemize}
            \item \textbf{Take the negative square outside of the brackets!}
                \begin{align*}
                    0 &= (x^2 + 4x + 4) + 2
                \end{align*}
            \item \textbf{The trinomial becomes two equal binomials}
                \begin{align*}
                    0 &= (x+2)^2 + 2
                \end{align*}
            \item \textbf{Solve for "x" by square rooting both sides:}
                \begin{align*}
                    -2 &= (x+2)^2
                \end{align*}
            \item Since a square cannot be negative, there are \textbf{No Real Solutions}.
        \end{itemize}
    \end{tcolorbox}
\end{frame}

\begin{frame}{Practice: Solve for "x" by CTS (Problem 5)}
    \begin{tcolorbox}[colback=lightgray,colframe=primary,title=Problem 5]
        \footnotesize
        Solve for "x" by completing the square:
        \[ 0 = 3x^2 + 9x - 6 \]
    \end{tcolorbox}
\end{frame}

\begin{frame}{Practice: Solve for "x" by CTS (Solution 5) - Part 1}
    \begin{tcolorbox}[colback=lightgray,colframe=accent,title=Solution 5 - Part 1]
        \footnotesize
        For $0 = 3x^2 + 9x - 6$:
        \begin{itemize}
            \item \textbf{Factor out any coefficient for $x^2$}
                \begin{align*}
                    0 &= 3(x^2 + 3x) - 6
                \end{align*}
            \item \textbf{Divide the second term by 2 and square it!}
                \begin{align*}
                    0 &= 3\left(x^2 + 3x + (1.5)^2\right) - 6 - 3(1.5)^2 \\
                    0 &= 3(x^2 + 3x + 2.25) - 6 - 6.75
                \end{align*}
            \item \textbf{Take the negative square outside of the brackets and multiply with coefficient in front!}
                \begin{align*}
                    0 &= 3(x^2 + 3x + 2.25) - 12.75
                \end{align*}
        \end{itemize}
    \end{tcolorbox}
\end{frame}

\begin{frame}{Practice: Solve for "x" by CTS (Solution 5) - Part 2}
    \begin{tcolorbox}[colback=lightgray,colframe=accent,title=Solution 5 - Part 2 (Cont.)]
        \footnotesize
        Continuing for $0 = 3(x^2 + 3x + 2.25) - 12.75$:
        \begin{itemize}
            \item \textbf{The trinomial becomes two equal binomials}
                \begin{align*}
                    0 &= 3(x+1.5)^2 - 12.75
                \end{align*}
            \item \textbf{Solve for "x" by square rooting both sides:}
                \begin{align*}
                    12.75 &= 3(x+1.5)^2 \\
                    4.25 &= (x+1.5)^2 \\
                    \pm\sqrt{4.25} &= x+1.5 \\
                    x &= -1.5 \pm\sqrt{4.25}
                \end{align*}
            \item So, $x_1 = -1.5 + \sqrt{4.25}$ and $x_2 = -1.5 - \sqrt{4.25}$.
        \end{itemize}
    \end{tcolorbox}
\end{frame}

\begin{frame}{Practice: Solve for "x" (Problem 6)}
    \begin{tcolorbox}[colback=lightgray,colframe=primary,title=Problem 6]
        \footnotesize
        Solve for "x":
        \[ 2x^2 + 16x + 30 = 0 \]
    \end{tcolorbox}
\end{frame}

\begin{frame}{Practice: Solve for "x" (Solution 6) - Part 1}
    \begin{tcolorbox}[colback=lightgray,colframe=accent,title=Solution 6 - Part 1]
        \footnotesize
        For $2x^2 + 16x + 30 = 0$:
        \begin{itemize}
            \item \textbf{Factor out any coefficient for $x^2$}
                \begin{align*}
                    0 &= 2(x^2 + 8x) + 30
                \end{align*}
            \item \textbf{Divide the second term by 2 and square it!}
                \begin{align*}
                    0 &= 2(x^2 + 8x + (-4)^2) + 30 - 2(-4)^2 \\
                    0 &= 2(x^2 + 8x + 16) + 30 - 2(16)
                \end{align*}
            \item \textbf{Simplify constants}
                \begin{align*}
                    0 &= 2(x^2 + 8x + 16) + 30 - 32 \\
                    0 &= 2(x^2 + 8x + 16) - 2
                \end{align*}
        \end{itemize}
    \end{tcolorbox}
\end{frame}

\begin{frame}{Practice: Solve for "x" (Solution 6) - Part 2}
    \begin{tcolorbox}[colback=lightgray,colframe=accent,title=Solution 6 - Part 2 (Cont.)]
        \footnotesize
        Continuing for $0 = 2(x^2 + 8x + 16) - 2$:
        \begin{itemize}
            \item \textbf{The trinomial becomes two equal binomials}
                \begin{align*}
                    0 &= 2(x+4)^2 - 2
                \end{align*}
            \item \textbf{Solve for "x" by square rooting both sides:}
                \begin{align*}
                    2 &= 2(x+4)^2 \\
                    1 &= (x+4)^2 \\
                    \pm\sqrt{1} &= x+4 \\
                    x &= -4 \pm 1
                \end{align*}
            \item So, $x_1 = -4 + 1 = -3$ and $x_2 = -4 - 1 = -5$.
        \end{itemize}
    \end{tcolorbox}
\end{frame}

\begin{frame}{Practice: Solve for "x" (Problem 7)}
    \begin{tcolorbox}[colback=lightgray,colframe=primary,title=Problem 7]
        \footnotesize
        Solve for "x":
        \[ 0.25x^2 - 2x + 1 = 0 \]
    \end{tcolorbox}
\end{frame}

\begin{frame}{Practice: Solve for "x" (Solution 7) - Part 1}
    \begin{tcolorbox}[colback=lightgray,colframe=accent,title=Solution 7 - Part 1]
        \footnotesize
        For $0.25x^2 - 2x + 1 = 0$:
        \begin{itemize}
            \item \textbf{Factor out any coefficient for $x^2$}
                \begin{align*}
                    0 &= 0.25(x^2 - 8x) + 1
                \end{align*}
            \item \textbf{Divide the second term by 2 and square it!}
                \begin{align*}
                    0 &= 0.25(x^2 - 8x + (-4)^2) + 1 - 0.25(-4)^2 \\
                    0 &= 0.25(x^2 - 8x + 16) + 1 - 0.25(16)
                \end{align*}
            \item \textbf{Simplify constants}
                \begin{align*}
                    0 &= 0.25(x^2 - 8x + 16) + 1 - 4
                \end{align*}
        \end{itemize}
    \end{tcolorbox}
\end{frame}

\begin{frame}{Practice: Solve for "x" (Solution 7) - Part 2}
    \begin{tcolorbox}[colback=lightgray,colframe=accent,title=Solution 7 - Part 2 (Cont.)]
        \footnotesize
        Continuing for $0 = 0.25(x^2 - 8x + 16) + 1 - 4$:
        \begin{itemize}
            \item \textbf{The trinomial becomes two equal binomials}
                \begin{align*}
                    0 &= 0.25(x-4)^2 - 3
                \end{align*}
            \item \textbf{Solve for "x" by square rooting both sides:}
                \begin{align*}
                    3 &= 0.25(x-4)^2 \\
                    12 &= (x-4)^2 \\
                    \pm\sqrt{12} &= x-4 \\
                    \pm 2\sqrt{3} &= x-4 \\
                    x &= 4 \pm 2\sqrt{3}
                \end{align*}
            \item So, $x_1 = 4 + 2\sqrt{3}$ and $x_2 = 4 - 2\sqrt{3}$.
        \end{itemize}
    \end{tcolorbox}
\end{frame}

% Section V: CTS with Algebra Tiles
\section{CTS with Algebra Tiles}

\begin{frame}{IV) CTS with Algebra Tiles}
    \begin{tcolorbox}[colback=lightgray,colframe=primary,title=Concept: $y = x^2 + 10x + 25$]
        \footnotesize
        \begin{figure}[H]
            \centering
            % Placeholder for Algebra Tiles image for x^2 + 10x + 25, if available.
            % Replace with \includegraphics{path/to/image.png} if you have one.
            % For now, we will describe it textually or leave it as a conceptual point.
            The tiles can be organized together to become a square
            
            \begin{itemize}
                \item Side 1: $x+5$
                \item Side 2: $x+5$
                \item So, $(x+5)(x+5) = (x+5)^2$
            \end{itemize}
        \end{figure}
    \end{tcolorbox}
\end{frame}

\begin{frame}{IV) CTS with Algebra Tiles (Cont.)}
    \begin{tcolorbox}[colback=lightgray,colframe=primary,title=Concept: $y = x^2 - 6x + 2$]
        \footnotesize
        \begin{figure}[H]
            \centering
            % Placeholder for Algebra Tiles image for x^2 - 6x + 2, if available.
            % Replace with \includegraphics{path/to/image.png} if you have one.
            % For now, we will describe it textually or leave it as a conceptual point.
            The equation is now in vertex form.
            Create a bunch of zero pairs to complete the square.
            
            \begin{itemize}
                \item Side 1: $x-3$
                \item Side 2: $x-3$
                \item Additional constant: $-7$
                \item So, $(x-3)^2 - 7$
            \end{itemize}
        \end{figure}
    \end{tcolorbox}
\end{frame}

% Section VI: Word Problem
\section{Word Problem}

\begin{frame}{Word Problem: Rock Thrown into Air (Problem)}
    \begin{tcolorbox}[colback=lightgray,colframe=primary,title=Problem]
        \footnotesize
        A rock is thrown into the air. The height of the rock is given by the formula:
        \[ h(t) = -3.5t^2 + 18t + 4 \]
        Where "H" is the height in meters and "T" is the time after the rock is thrown in seconds.
        \begin{enumerate}
            \item Convert the following equation to APQ form.
            \item When will the rock be at the maximum height?
            \item What is the maximum height?
        \end{enumerate}
    \end{tcolorbox}
\end{frame}

\begin{frame}{Word Problem: Rock Thrown into Air (Solution Part A) - Part 1}
    \begin{tcolorbox}[colback=lightgray,colframe=accent,title=Solution Part A: Convert to APQ Form - Part 1]
        \footnotesize
        For $h(t) = -3.5t^2 + 18t + 4$:
        \begin{itemize}
            \item \textbf{Factor out any coefficient for $t^2$}
                \begin{align*}
                    h(t) &= -3.5\left(t^2 - \frac{18}{3.5}t\right) + 4 \\
                    h(t) &= -3.5(t^2 - 5.142857t) + 4 \quad \text{(approx.)}
                \end{align*}
            \item \textbf{Divide the second term by 2 and square it!}
                \begin{align*}
                    h(t) &= -3.5\left(t^2 - 5.142857t + \left(\frac{-5.142857}{2}\right)^2\right) + 4 - (-3.5)\left(\frac{-5.142857}{2}\right)^2 \\
                    h(t) &= -3.5(t^2 - 5.142857t + (2.5714285)^2) + 4 + 3.5(2.5714285)^2
                \end{align*}
        \end{itemize}
    \end{tcolorbox}
\end{frame}

\begin{frame}{Word Problem: Rock Thrown into Air (Solution Part A) - Part 2}
    \begin{tcolorbox}[colback=lightgray,colframe=accent,title=Solution Part A: Convert to APQ Form - Part 2 (Cont.)]
        \footnotesize
        Continuing for $h(t) = -3.5(t^2 - 5.142857t + (2.5714285)^2) + 4 + 3.5(2.5714285)^2$:
        \begin{itemize}
            \item \textbf{Simplify and combine constants}
                \begin{align*}
                    h(t) &= -3.5(t^2 - 5.142857t + 6.612244) + 4 + 23.142854
                \end{align*}
            \item \textbf{The trinomial becomes two equal binomials}
                \begin{align*}
                    h(t) &= -3.5(t - 2.5714285)^2 + 27.142854
                \end{align*}
            \item \textbf{APQ Form:}
                \begin{align*}
                    h(t) &= -3.5(t - 2.57)^2 + 27.14 \quad \text{(approx.)}
                \end{align*}
        \end{itemize}
    \end{tcolorbox}
\end{frame}

\begin{frame}{Word Problem: Rock Thrown into Air (Solution Part B & C)}
    \begin{tcolorbox}[colback=lightgray,colframe=accent,title=Solution Part B & C: Max Height]
        \footnotesize
        For $h(t) = -3.5(t - 2.57)^2 + 27.14$:
        \begin{itemize}
            \item The rock will be at maximum height at the vertex of the equation.
            \item From the APQ form, the vertex is $(p,q) = (2.57, 27.14)$.
            \item \textbf{When will the rock be at the maximum height?}
                \begin{itemize}
                    \item \textbf{Answer:} Approximately 2.57 seconds.
                \end{itemize}
            \item \textbf{What is the maximum height?}
                \begin{itemize}
                    \item \textbf{Answer:} Approximately 27.14 meters.
                \end{itemize}
        \end{itemize}
    \end{tcolorbox}
\end{frame}

\end{document} 