\documentclass[aspectratio=169]{beamer}

% Theme and colors
\usetheme{Madrid}
\usecolortheme{whale}
% Custom colors
\definecolor{primary}{RGB}{41, 128, 185}
\definecolor{secondary}{RGB}{52, 152, 219}
\definecolor{accent}{RGB}{231, 76, 60}
\definecolor{lightgray}{RGB}{236, 240, 241}

% Packages
\usepackage[utf8]{inputenc}
\usepackage{graphicx}
\usepackage{amsmath}
\usepackage{amsfonts}
\usepackage{amssymb}
\usepackage{mathpazo} % Palatino font
\usepackage{multirow}
\usepackage{tcolorbox}
\usepackage{tikz} % For diagrams
\usepackage{caption} % For figure captions
\usepackage{adjustbox} % For adjusting box
\usepackage{float} % For better float control

% Beamer specific configurations
\setbeamercolor{structure}{fg=primary}
\setbeamercolor{background canvas}{bg=white}
\setbeamercolor{normal text}{fg=black}

% Information for the title page
\title{Lesson 7: Word Problems with Quadratic Functions}
\author{Yi-Chen Lin}
\date{June 10, 2025}

\begin{document}

% Title page
\begin{frame}
    \titlepage
\end{frame}

% Table of Contents
\begin{frame}{Table of Contents}
    \tableofcontents
\end{frame}

% Section I: Recap
\section{Recap}

\begin{frame}{I) Recap on What We Learned So Far}
    \begin{tcolorbox}[colback=lightgray,colframe=primary,title=Key Concepts]
        \footnotesize
        \begin{itemize}
            \item The vertex is for finding the Max/Minimum of something
            \item The x-intercepts is when the height is zero
            \item The y-intercept is when t=0 or x=0, height at the beginning
            \item The axis of symmetry is the x-value or the time when an object is at the vertex (highest point) or lowest point
        \end{itemize}
    \end{tcolorbox}
\end{frame}

\begin{frame}{I) General Form and APQ Form}
    \begin{tcolorbox}[colback=lightgray,colframe=primary,title=Forms]
        \footnotesize
        \textbf{General Form:}
        \[y = ax^2 + bx + c\]
        \[x = -\frac{b}{2a} \text{ (Axis of Symmetry)}\]
        
        \textbf{APQ Form:}
        \[y = a(x-p)^2 + q\]
        Vertex: $(p,q)$
        A.O.S. $x = p$
    \end{tcolorbox}
\end{frame}

% Section II: Applications
\section{Applications}

\begin{frame}{II) Applications of Quadratic Functions}
    \begin{tcolorbox}[colback=lightgray,colframe=primary,title=Applications]
        \footnotesize
        \begin{itemize}
            \item Finance and Business:
            \begin{itemize}
                \item Maximize Revenue and Profit
            \end{itemize}
            \item Construction and Infrastructure
            \begin{itemize}
                \item Maximize the area of a park, building, room
            \end{itemize}
            \item Coding and Programming
            \begin{itemize}
                \item Maximize efficiency: minimize resources
            \end{itemize}
        \end{itemize}
    \end{tcolorbox}
\end{frame}

\begin{frame}{II) Common Terms in Word Problems}
    \begin{tcolorbox}[colback=lightgray,colframe=primary,title=Common Terms]
        \footnotesize
        \begin{itemize}
            \item Sum $\rightarrow$ Add: $x + y = 10$
            \item Difference $\rightarrow$ Subtract: $x - y = 8$
            \item Product $\rightarrow$ Multiply: $M = x \cdot y$
            \item Sum of squares $\rightarrow$ $Min = x^2 + y^2$
            \item Perimeter $\rightarrow$ $P = 2L + 2W$
        \end{itemize}
    \end{tcolorbox}
\end{frame}

% Section III: Example 1
\section{Example 1}

\begin{frame}{III) Example 1: Product of Numbers}
    \begin{tcolorbox}[colback=lightgray,colframe=primary,title=Problem]
        \footnotesize
        The sum of two numbers is 95. Their product is 2100. Find the numbers.
    \end{tcolorbox}
\end{frame}

\begin{frame}{III) Example 1: Solution - Part 1}
    \begin{tcolorbox}[colback=lightgray,colframe=accent,title=Solution - Part 1]
        \footnotesize
        \begin{enumerate}
            \item Write expressions:
            \begin{itemize}
                \item 1st value: $x$
                \item 2nd value: $95 - x$
            \end{itemize}
            \item Product equation:
            \[x(95 - x) = 2100\]
        \end{enumerate}
    \end{tcolorbox}
\end{frame}

\begin{frame}{III) Example 1: Solution - Part 2}
    \begin{tcolorbox}[colback=lightgray,colframe=accent,title=Solution - Part 2]
        \footnotesize
        \begin{align*}
            95x - x^2 &= 2100\\
            -x^2 + 95x - 2100 &= 0\\
            x^2 - 95x + 2100 &= 0\\
            (x - 47.5)^2 &= 156.25\\
            x - 47.5 &= \pm 12.5\\
            x &= 47.5 \pm 12.5\\
            x &= 60 \text{ or } 35
        \end{align*}
        The two numbers are 35 and 60.
    \end{tcolorbox}
\end{frame}

% Section IV: Example 2
\section{Example 2}

\begin{frame}{IV) Example 2: Sum of Squares}
    \begin{tcolorbox}[colback=lightgray,colframe=primary,title=Problem]
        \footnotesize
        The difference of two numbers is 12. The sum of their squares is 74. Find the numbers.
    \end{tcolorbox}
\end{frame}

\begin{frame}{IV) Example 2: Solution - Part 1}
    \begin{tcolorbox}[colback=lightgray,colframe=accent,title=Solution - Part 1]
        \footnotesize
        \begin{enumerate}
            \item Write expressions:
            \begin{itemize}
                \item 1st value: $x$
                \item 2nd value: $x + 12$
            \end{itemize}
            \item Sum of squares equation:
            \[x^2 + (x + 12)^2 = 74\]
        \end{enumerate}
    \end{tcolorbox}
\end{frame}

\begin{frame}{IV) Example 2: Solution - Part 2}
    \begin{tcolorbox}[colback=lightgray,colframe=accent,title=Solution - Part 2]
        \footnotesize
        \begin{align*}
            x^2 + x^2 + 24x + 144 &= 74\\
            2x^2 + 24x + 70 &= 0\\
            x^2 + 12x + 35 &= 0\\
            (x + 6)^2 &= 1\\
            x + 6 &= \pm 1\\
            x &= -6 \pm 1
        \end{align*}
        The two sets of numbers are:
        \[x = -5, y = 7\]
        \[x = -7, y = 5\]
    \end{tcolorbox}
\end{frame}

% Section V: Example 3
\section{Example 3}

\begin{frame}{V) Example 3: Projectile Motion}
    \begin{tcolorbox}[colback=lightgray,colframe=primary,title=Problem]
        \footnotesize
        The height, "H" metres, of a baseball "T" seconds after being hit is given by:
        \[H = 35T - 5T^2\]
        Find:
        \begin{enumerate}
            \item Height after 3 seconds
            \item Time at height of 25m
            \item Maximum height and when it occurs
            \item When the ball hits the ground
        \end{enumerate}
    \end{tcolorbox}
\end{frame}

\begin{frame}{V) Example 3: Solution - Part 1}
    \begin{tcolorbox}[colback=lightgray,colframe=accent,title=Solution - Part 1]
        \footnotesize
        \begin{enumerate}
            \item Height after 3 seconds:
            \[H = 35(3) - 5(3)^2 = 105 - 45 = 60 \text{ metres}\]
            
            \item Time at height of 25m:
            \[25 = 35T - 5T^2\]
            \[5T^2 - 35T + 25 = 0\]
            \[T = 0.82 \text{ sec or } 6.18 \text{ sec}\]
        \end{enumerate}
    \end{tcolorbox}
\end{frame}

\begin{frame}{V) Example 3: Solution - Part 2}
    \begin{tcolorbox}[colback=lightgray,colframe=accent,title=Solution - Part 2]
        \footnotesize
        \begin{enumerate}
            \item Maximum height:
            \[T = -\frac{b}{2a} = -\frac{35}{2(-5)} = 3.5 \text{ seconds}\]
            \[H = 35(3.5) - 5(3.5)^2 = 122.5 - 61.25 = 61.25 \text{ metres}\]
            
            \item Time to hit ground:
            \[0 = 35T - 5T^2\]
            \[T = 0 \text{ or } 7 \text{ seconds}\]
        \end{enumerate}
    \end{tcolorbox}
\end{frame}

% Section VI: Example 4
\section{Example 4}

\begin{frame}{VI) Example 4: Maximum Area}
    \begin{tcolorbox}[colback=lightgray,colframe=primary,title=Problem]
        \footnotesize
        A farmer wants to build a rectangular barn using 120 meters of fencing for his cows and chickens. He needs to separate the two groups and make the largest possible area. Determine the dimensions.
    \end{tcolorbox}
\end{frame}

\begin{frame}{VI) Example 4: Solution - Part 1}
    \begin{tcolorbox}[colback=lightgray,colframe=accent,title=Solution - Part 1]
        \footnotesize
        \begin{enumerate}
            \item Perimeter equation: $2L + 3W = 120$
            \item Area equation: $A = L \times W$
            \item Isolate L: $L = -1.5W + 60$
            \item Substitute into area equation: $A = W(-1.5W + 60)$
        \end{enumerate}
    \end{tcolorbox}
\end{frame}

\begin{frame}{VI) Example 4: Solution - Part 2}
    \begin{tcolorbox}[colback=lightgray,colframe=accent,title=Solution - Part 2]
        \footnotesize
        \begin{enumerate}
            \item Complete the square:
            \[A = -1.5(W - 20)^2 + 600\]
            \item Maximum area: $600 m^2$
            \item Dimensions:
            \begin{itemize}
                \item Width: $20m$
                \item Length: $30m$
            \end{itemize}
        \end{enumerate}
    \end{tcolorbox}
\end{frame}

% Section VII: Practice Problems
\section{Practice Problems}

\begin{frame}{VII) Practice Problem 1: Rectangle Area}
    \begin{tcolorbox}[colback=lightgray,colframe=primary,title=Problem]
        \footnotesize
        A rectangle has a perimeter of 80 meters. Find the dimensions that will give the maximum area.
    \end{tcolorbox}
\end{frame}

\begin{frame}{VII) Practice Problem 1: Solution}
    \begin{tcolorbox}[colback=lightgray,colframe=accent,title=Solution]
        \footnotesize
        \begin{enumerate}
            \item Perimeter equation: $2L + 2W = 80$
            \item Area equation: $A = L \times W$
            \item Isolate L: $L = 40 - W$
            \item Substitute: $A = W(40 - W)$
            \item Complete the square:
            \[A = -(W - 20)^2 + 400\]
            \item Maximum area: $400 m^2$
            \item Dimensions: $20m \times 20m$ (square)
        \end{enumerate}
    \end{tcolorbox}
\end{frame}

\begin{frame}{VII) Practice Problem 2: Number Product}
    \begin{tcolorbox}[colback=lightgray,colframe=primary,title=Problem]
        \footnotesize
        Find two numbers whose sum is 50 and whose product is a maximum.
    \end{tcolorbox}
\end{frame}

\begin{frame}{VII) Practice Problem 2: Solution}
    \begin{tcolorbox}[colback=lightgray,colframe=accent,title=Solution]
        \footnotesize
        \begin{enumerate}
            \item Let first number be $x$
            \item Second number is $50 - x$
            \item Product equation: $P = x(50 - x)$
            \item Complete the square:
            \[P = -(x - 25)^2 + 625\]
            \item Maximum product: 625
            \item Numbers: 25 and 25
        \end{enumerate}
    \end{tcolorbox}
\end{frame}

\begin{frame}{VII) Practice Problem 3: Projectile}
    \begin{tcolorbox}[colback=lightgray,colframe=primary,title=Problem]
        \footnotesize
        A ball is thrown upward from a height of 2 meters with an initial velocity of 40 m/s. The height is given by:
        \[H = -5T^2 + 40T + 2\]
        Find:
        \begin{enumerate}
            \item Maximum height
            \item Time to reach maximum height
            \item When the ball hits the ground
        \end{enumerate}
    \end{tcolorbox}
\end{frame}

\begin{frame}{VII) Practice Problem 3: Solution}
    \begin{tcolorbox}[colback=lightgray,colframe=accent,title=Solution]
        \footnotesize
        \begin{enumerate}
            \item Time to maximum height:
            \[T = -\frac{b}{2a} = -\frac{40}{2(-5)} = 4 \text{ seconds}\]
            \item Maximum height:
            \[H = -5(4)^2 + 40(4) + 2 = 82 \text{ meters}\]
            \item Time to hit ground:
            \[0 = -5T^2 + 40T + 2\]
            \[T = 8.05 \text{ seconds}\]
        \end{enumerate}
    \end{tcolorbox}
\end{frame}

\end{document} 