\documentclass[aspectratio=169]{beamer}
\usepackage{amsmath}
\usepackage{amssymb}
\usepackage{graphicx}
\usepackage{tcolorbox}
\usepackage{booktabs}
\usepackage{colortbl}
\usepackage{xcolor}
\usepackage{tikz}
\usepackage[utf8]{inputenc}

% Custom colors
\definecolor{primary}{RGB}{41, 128, 185}
\definecolor{secondary}{RGB}{52, 152, 219}
\definecolor{accent}{RGB}{231, 76, 60}
\definecolor{lightgray}{RGB}{236, 240, 241}

% Theme customization
\usetheme{Madrid}
\usecolortheme{whale}
\setbeamercolor{structure}{fg=primary}
\setbeamercolor{background canvas}{bg=white}
\setbeamercolor{normal text}{fg=black}

\title{Chapter 5.3: Adding and Subtracting Rational Expressions}
\subtitle{Rational Expressions - Lesson 3}
\author{Created by Yi-Chen Lin}
\date{\today}

\begin{document}

\begin{frame}
\titlepage
\end{frame}

% I) Adding and Subtracting Rational Expressions
\begin{frame}{Adding and Subtracting Rational Expressions}
\begin{tcolorbox}[colback=lightgray,colframe=primary,title=Key Concepts]
\footnotesize
\begin{itemize}
  \item To add or subtract two rational expressions, they MUST have a common denominator (LCD).
  \item LCD is ONLY needed for adding & subtracting, NOT for multiplying/dividing.
  \item To get a common denominator, multiply each fraction by the missing terms.
\end{itemize}
\end{tcolorbox}
\end{frame}

% Example: Find LCD and Add/Subtract
\begin{frame}{Example: Find LCD and Add/Subtract}
\begin{tcolorbox}[colback=lightgray,colframe=secondary,title=Example 1]
\footnotesize
Add:
\[
\frac{7}{2x} - \frac{5}{3y}
\]
Find LCD: $6xy$
\[
\frac{7}{2x} \times \frac{3y}{3y} = \frac{21y}{6xy}
\]
\[
\frac{5}{3y} \times \frac{2x}{2x} = \frac{10x}{6xy}
\]
Combine:
\[
\frac{21y - 10x}{6xy}
\]
\end{tcolorbox}
\end{frame}

% Practice: Add or Subtract by finding the LCD
\begin{frame}{Practice: Add or Subtract by finding the LCD}
\begin{tcolorbox}[colback=lightgray,colframe=primary,title=Practice Problems]
\footnotesize
\begin{enumerate}
  \item[Q1.] $\frac{4}{3x} - \frac{6}{x}$
  \item[Q2.] $\frac{5}{2m} + \frac{8}{3n}$
  \item[Q3.] $\frac{5}{6m} - \frac{3}{4m}$
  \item[Q4.] $\frac{y^2}{3x^2} + \frac{5x}{y}$
  \item[Q5.] $\frac{3}{2ab} - \frac{2}{3bc} + \frac{4}{abc}$
\end{enumerate}
\end{tcolorbox}
\end{frame}

% II) Finding LCD with Binomials
\begin{frame}{Finding LCD with Binomials}
\begin{tcolorbox}[colback=lightgray,colframe=primary,title=Key Concepts]
\footnotesize
\begin{itemize}
  \item If the denominator has a binomial, multiply in the missing binomial for each fraction.
  \item You cannot just add or subtract the missing term; you must multiply.
\end{itemize}
\end{tcolorbox}
\end{frame}

% Example: Binomial Denominators
\begin{frame}{Example: Binomial Denominators}
\begin{tcolorbox}[colback=lightgray,colframe=secondary,title=Example 2]
\footnotesize
Add:
\[
\frac{4}{x-3} + \frac{2}{x}
\]
LCD: $(x-3)x$
\[
\frac{4}{x-3} \times \frac{x}{x} = \frac{4x}{x(x-3)}
\]
\[
\frac{2}{x} \times \frac{x-3}{x-3} = \frac{2(x-3)}{x(x-3)}
\]
Combine:
\[
\frac{4x + 2(x-3)}{x(x-3)} = \frac{4x + 2x - 6}{x(x-3)} = \frac{6x-6}{x(x-3)}
\]
\end{tcolorbox}
\end{frame}

% Practice: Find the LCD and then Simplify
\begin{frame}{Practice: Find the LCD and then Simplify}
\begin{tcolorbox}[colback=lightgray,colframe=primary,title=Practice Problems]
\footnotesize
\begin{enumerate}
  \item[Q6.] $\frac{5}{x} + \frac{3}{x-2}$
  \item[Q7.] $\frac{4}{x+1} - \frac{2}{x}$
  \item[Q8.] $\frac{2}{x-1} + \frac{8}{x+4}$
  \item[Q9.] $\frac{5x}{x-3} - \frac{2x}{x}$
\end{enumerate}
\end{tcolorbox}
\end{frame}

% III) Factor the Denominators
\begin{frame}{Factor the Denominators}
\begin{tcolorbox}[colback=lightgray,colframe=primary,title=Key Concepts]
\footnotesize
\begin{itemize}
  \item If the denominator is a trinomial or a difference of squares, always factor them first!
  \item After factoring, look for any missing terms or binomials to get the LCD.
\end{itemize}
\end{tcolorbox}
\end{frame}

% Example: Factor Denominators
\begin{frame}{Example: Factor Denominators}
\begin{tcolorbox}[colback=lightgray,colframe=secondary,title=Example 3]
\footnotesize
Add:
\[
\frac{2}{x^2-16} + \frac{8}{x+4}
\]
Factor: $x^2-16 = (x+4)(x-4)$
LCD: $(x+4)(x-4)$
\[
\frac{2}{(x+4)(x-4)} + \frac{8(x-4)}{(x+4)(x-4)} = \frac{2 + 8(x-4)}{(x+4)(x-4)} = \frac{8x-30}{(x+4)(x-4)}
\]
\end{tcolorbox}
\end{frame}

% Practice: Factor, Find LCD, and Simplify
\begin{frame}{Practice: Factor, Find LCD, and Simplify}
\begin{tcolorbox}[colback=lightgray,colframe=primary,title=Practice Problems]
\footnotesize
\begin{enumerate}
  \item[Q10.] $\frac{2}{x^2-9} + \frac{5}{x+3}$
  \item[Q11.] $\frac{4}{x^2-4} - \frac{2}{x-2}$
  \item[Q12.] $\frac{3}{x^2-1} + \frac{2}{x-1}$
  \item[Q13.] $\frac{5x}{x^2-4x+3} - \frac{2x}{x-1}$
\end{enumerate}
\end{tcolorbox}
\end{frame}

% IV) Simplifying R.E. with Fractions
\begin{frame}{Simplifying R.E. with Fractions}
\begin{tcolorbox}[colback=lightgray,colframe=primary,title=Key Concepts]
\footnotesize
\begin{itemize}
  \item When simplifying rational expressions with fractions, multiply ALL terms by the LCD to cancel out denominators.
  \item Distribute and expand as needed.
\end{itemize}
\end{tcolorbox}
\end{frame}

% Example: Simplifying with Fractions
\begin{frame}{Example: Simplifying with Fractions}
\begin{tcolorbox}[colback=lightgray,colframe=secondary,title=Example 4]
\footnotesize
Simplify:
\[
\frac{4}{2x} - \frac{3}{6x} + 1
\]
LCD: $6x$
\[
\frac{4}{2x} \times 3 = \frac{12}{6x}
\]
\[
\frac{3}{6x} \times 1 = \frac{3}{6x}
\]
Combine:
\[
\frac{12-3}{6x} + 1 = \frac{9}{6x} + 1
\]
\end{tcolorbox}
\end{frame}

% Practice: Simplify Each of the Following
\begin{frame}{Practice: Simplify Each of the Following}
\begin{tcolorbox}[colback=lightgray,colframe=primary,title=Practice Problems]
\footnotesize
\begin{enumerate}
  \item[Q14.] $\frac{4}{2x} - \frac{3}{6x} + 1$
  \item[Q15.] $\frac{2}{2x} + \frac{8}{10x} + 1$
  \item[Q16.] $\frac{7}{3x} - \frac{4}{2x} + 2$
\end{enumerate}
\end{tcolorbox}
\end{frame}

% Challenge Problems
\begin{frame}{Challenge Problem Q17}
\begin{tcolorbox}[colback=lightgray,colframe=primary,title=Challenge Q17]
\footnotesize
Simplify and state all NPV's:
\[
\frac{2}{x^2-5x+6} + \frac{3}{x^2-4}
\]
\end{tcolorbox}
\end{frame}

\begin{frame}{Challenge Q17: Solution (Part 1)}
\begin{tcolorbox}[colback=lightgray,colframe=accent,title=Step-by-Step Solution (Part 1)]
\footnotesize
\textbf{Step 1: Factor denominators}
\[
x^2-5x+6 = (x-2)(x-3)\qquad x^2-4 = (x-2)(x+2)
\]
\textbf{Step 2: LCD is } $(x-2)(x-3)(x+2)$
\textbf{Step 3: Rewrite each fraction with LCD}
\[
\frac{2}{(x-2)(x-3)} \times \frac{x+2}{x+2} = \frac{2(x+2)}{(x-2)(x-3)(x+2)}
\]
\[
\frac{3}{(x-2)(x+2)} \times \frac{x-3}{x-3} = \frac{3(x-3)}{(x-2)(x-3)(x+2)}
\]
\end{tcolorbox}
\end{frame}

\begin{frame}{Challenge Q17: Solution (Part 2)}
\begin{tcolorbox}[colback=lightgray,colframe=accent,title=Step-by-Step Solution (Part 2)]
\footnotesize
\textbf{Step 4: Combine numerators}
\[
\frac{2(x+2) + 3(x-3)}{(x-2)(x-3)(x+2)} = \frac{2x+4+3x-9}{(x-2)(x-3)(x+2)} = \frac{5x-5}{(x-2)(x-3)(x+2)}
\]
\textbf{Step 5: Factor numerator if possible}
\[
5x-5 = 5(x-1)
\]
\textbf{Final Answer:}
\[
\frac{5(x-1)}{(x-2)(x-3)(x+2)}
\]
\textbf{NPV:} $x \neq 2, 3, -2$
\end{tcolorbox}
\end{frame}

% Challenge Problem 2
\begin{frame}{Challenge Problem Q18}
\begin{tcolorbox}[colback=lightgray,colframe=primary,title=Challenge Q18]
\footnotesize
Simplify and state all NPV's:
\[
\frac{3x}{x^2-4x+3} - \frac{2}{x^2-1} + \frac{5}{x^2-9}
\]
\end{tcolorbox}
\end{frame}

\begin{frame}{Challenge Q18: Solution (Part 1)}
\begin{tcolorbox}[colback=lightgray,colframe=accent,title=Step-by-Step Solution (Part 1)]
\footnotesize
\textbf{Step 1: Factor denominators}
\[
x^2-4x+3 = (x-3)(x-1)\qquad x^2-1 = (x-1)(x+1)\qquad x^2-9 = (x-3)(x+3)
\]
\textbf{Step 2: LCD is } $(x-3)(x-1)(x+1)(x+3)$
\textbf{Step 3: Rewrite each fraction with LCD}
\[
\frac{3x}{(x-3)(x-1)} \times \frac{(x+1)(x+3)}{(x+1)(x+3)} = \frac{3x(x+1)(x+3)}{(x-3)(x-1)(x+1)(x+3)}
\]
\[
\frac{2}{(x-1)(x+1)} \times \frac{(x-3)(x+3)}{(x-3)(x+3)} = \frac{2(x-3)(x+3)}{(x-3)(x-1)(x+1)(x+3)}
\]
\[
\frac{5}{(x-3)(x+3)} \times \frac{(x-1)(x+1)}{(x-1)(x+1)} = \frac{5(x-1)(x+1)}{(x-3)(x-1)(x+1)(x+3)}
\]
\end{tcolorbox}
\end{frame}

\begin{frame}{Challenge Q18: Solution (Part 2)}
\begin{tcolorbox}[colback=lightgray,colframe=accent,title=Step-by-Step Solution (Part 2)]
\footnotesize
\textbf{Step 4: Combine numerators}
\[
\frac{3x(x+1)(x+3) - 2(x-3)(x+3) + 5(x-1)(x+1)}{(x-3)(x-1)(x+1)(x+3)}
\]
\textbf{Step 5: Expand and simplify numerator}
\[
3x(x+1)(x+3) = 3x(x^2+4x+3) = 3x^3+12x^2+9x
\]
\[
-2(x-3)(x+3) = -2(x^2-9) = -2x^2+18
\]
\[
5(x-1)(x+1) = 5(x^2-1) = 5x^2-5
\]
\[
\text{Sum: } 3x^3 + 12x^2 + 9x - 2x^2 + 18 + 5x^2 - 5
= 3x^3 + (12x^2 - 2x^2 + 5x^2) + 9x + (18-5)
= 3x^3 + 15x^2 + 9x + 13
\]
\textbf{Final Answer:}
\[
\frac{3x^3 + 15x^2 + 9x + 13}{(x-3)(x-1)(x+1)(x+3)}
\]
\textbf{NPV:} $x \neq 3, 1, -1, -3$
\end{tcolorbox}
\end{frame}

\end{document} 