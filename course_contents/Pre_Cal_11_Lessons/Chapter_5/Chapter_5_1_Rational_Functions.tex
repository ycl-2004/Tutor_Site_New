\documentclass[aspectratio=169]{beamer}
\usepackage{amsmath}
\usepackage{amssymb}
\usepackage{graphicx}
\usepackage{tcolorbox}
\usepackage{booktabs}
\usepackage{colortbl}
\usepackage{xcolor}
\usepackage{tikz}
\usepackage[utf8]{inputenc}

% Custom colors
\definecolor{primary}{RGB}{41, 128, 185}
\definecolor{secondary}{RGB}{52, 152, 219}
\definecolor{accent}{RGB}{231, 76, 60}
\definecolor{lightgray}{RGB}{236, 240, 241}

% Theme customization
\usetheme{Madrid}
\usecolortheme{whale}
\setbeamercolor{structure}{fg=primary}
\setbeamercolor{background canvas}{bg=white}
\setbeamercolor{normal text}{fg=black}

\title{Chapter 5.1: Rational Functions}
\subtitle{Simplifying Rational Expressions and NPV's}
\author{Created by Yi-Chen Lin}
\date{\today}

\begin{document}

\begin{frame}
\titlepage
\end{frame}

% I) What are Rational Expressions?
\begin{frame}{What are Rational Expressions?}
\begin{tcolorbox}[colback=lightgray,colframe=primary,title=Definition]
\footnotesize
\begin{itemize}
  \item A rational expression is a fraction where both the numerator and denominator are polynomials
  \item In this section, the numerator and denominator are binomials and trinomials that can be factored
  \item "x" can't be an exponent or inside a radical, exponents of 'x' must be integers [not fractions]
\end{itemize}
\end{tcolorbox}

\begin{tcolorbox}[colback=lightgray,colframe=secondary,title=Example]
\footnotesize
\begin{align*}
\frac{x^2 - 3x - 4}{x^2 - 16} &= \frac{(x-4)(x+1)}{(x+4)(x-4)} \\
&= \frac{x+1}{x+4}
\end{align*}
\end{tcolorbox}
\end{frame}

% II) Identifying Rational Expressions
\begin{frame}{Are These Rational Expressions?}
\begin{tcolorbox}[colback=lightgray,colframe=primary,title=Practice]
\footnotesize
Indicate whether the following expressions are rational expressions:
\begin{enumerate}
  \item $\frac{x^2 - 4x - 9}{x^3 - 3}$
  \item $\frac{x^3 - 8}{x^4 - 5}$
  \item $\frac{x^2 - 7}{x^3 - 10}$
  \item $\frac{x^3 - 9x^2 + 4}{x^2 - 12}$
  \item $\frac{x^4 - 1}{x^3 + 3}$
  \item $\frac{x^2 - 3}{x^4 + 5}$
  \item $\frac{x^4 - 7x^2 + 2}{x^3 - 8}$
  \item $\frac{x^3 - 5}{x^2 + 3}$
  \item $\frac{x^4 - 6}{x^3 + x}$
\end{enumerate}
\end{tcolorbox}
\end{frame}

\begin{frame}{Are These Rational Expressions? - Solutions Part 1}
\begin{tcolorbox}[colback=lightgray,colframe=accent,title=Detailed Solutions]
\footnotesize
\begin{enumerate}
  \item $\frac{x^2 - 4x - 9}{x^3 - 3}$ \\
  \textbf{Yes} - Both numerator and denominator are polynomials with integer exponents

  \item $\frac{x^3 - 8}{x^4 - 5}$ \\
  \textbf{Yes} - Both numerator and denominator are polynomials with integer exponents

  \item $\frac{x^2 - 7}{x^3 - 10}$ \\
  \textbf{Yes} - Both numerator and denominator are polynomials with integer exponents

  \item $\frac{x^3 - 9x^2 + 4}{x^2 - 12}$ \\
  \textbf{Yes} - Both numerator and denominator are polynomials with integer exponents
\end{enumerate}
\end{tcolorbox}
\end{frame}

\begin{frame}{Are These Rational Expressions? - Solutions Part 2}
\begin{tcolorbox}[colback=lightgray,colframe=accent,title=Detailed Solutions]
\footnotesize
\begin{enumerate}
  \setcounter{enumi}{4}
  \item $\frac{x^4 - 1}{x^3 + 3}$ \\
  \textbf{Yes} - Both numerator and denominator are polynomials with integer exponents

  \item $\frac{x^2 - 3}{x^4 + 5}$ \\
  \textbf{Yes} - Both numerator and denominator are polynomials with integer exponents

  \item $\frac{x^4 - 7x^2 + 2}{x^3 - 8}$ \\
  \textbf{Yes} - Both numerator and denominator are polynomials with integer exponents

  \item $\frac{x^3 - 5}{x^2 + 3}$ \\
  \textbf{Yes} - Both numerator and denominator are polynomials with integer exponents

  \item $\frac{x^4 - 6}{x^3 + x}$ \\
  \textbf{Yes} - Both numerator and denominator are polynomials with integer exponents
\end{enumerate}
\end{tcolorbox}
\end{frame}

\begin{frame}{Are These Rational Expressions? - Explanation}
\begin{tcolorbox}[colback=lightgray,colframe=primary,title=Key Points]
\footnotesize
\begin{itemize}
  \item All expressions are rational expressions because:
  \begin{itemize}
    \item Both numerator and denominator are polynomials
    \item All exponents are integers
    \item No radicals in the expressions
    \item No fractional exponents
  \end{itemize}
  \item Examples of expressions that would NOT be rational:
  \begin{itemize}
    \item $\frac{\sqrt{x}}{x^2}$ (contains radical)
    \item $\frac{x^{1/2}}{x^3}$ (contains fractional exponent)
    \item $\frac{\sin x}{x^2}$ (contains trigonometric function)
  \end{itemize}
\end{itemize}
\end{tcolorbox}
\end{frame}

% --- Split: More Examples of Non-Rational Expressions (Page 1) ---

\begin{frame}{More Examples of Non-Rational Expressions (1/2)}
\begin{tcolorbox}[colback=lightgray,colframe=accent,title=Examples: Radicals & Fractional Exponents]
\footnotesize
\begin{enumerate}
  \item Expressions with Radicals:
  \begin{itemize}
    \item $\frac{\sqrt{x+1}}{x^2}$ (square root in numerator)
    \item $\frac{x^2}{\sqrt[3]{x-2}}$ (cube root in denominator)
    \item $\frac{\sqrt{x^2+1}}{x+1}$ (square root of polynomial)
  \end{itemize}

  \item Expressions with Fractional Exponents:
  \begin{itemize}
    \item $\frac{x^{3/2}}{x^2}$ (fractional exponent in numerator)
    \item $\frac{x^2}{x^{-1/3}}$ (negative fractional exponent)
    \item $\frac{(x+1)^{1/4}}{x^2}$ (fractional exponent of binomial)
  \end{itemize}
\end{enumerate}
\end{tcolorbox}
\end{frame}

% --- Split: More Examples of Non-Rational Expressions (Page 2) ---

\begin{frame}{More Examples of Non-Rational Expressions (2/2)}
\begin{tcolorbox}[colback=lightgray,colframe=accent,title=Examples: Trig, Logarithms, Mixed]
\footnotesize
\begin{enumerate}
  \setcounter{enumi}{2}
  \item Expressions with Trigonometric Functions:
  \begin{itemize}
    \item $\frac{\sin x}{x^2}$ (sine function)
    \item $\frac{x^2}{\cos x}$ (cosine function)
    \item $\frac{\tan x}{x+1}$ (tangent function)
  \end{itemize}

  \item Expressions with Logarithms:
  \begin{itemize}
    \item $\frac{\ln x}{x^2}$ (natural logarithm)
    \item $\frac{x^2}{\log x}$ (common logarithm)
    \item $\frac{\log_2(x+1)}{x}$ (logarithm with base 2)
  \end{itemize}

  \item Mixed Non-Rational Expressions:
  \begin{itemize}
    \item $\frac{\sqrt{x} + \sin x}{x^2}$ (combination of radical and trig)
    \item $\frac{x^{1/2} + \ln x}{x+1}$ (combination of fractional exponent and log)
    \item $\frac{\sqrt{x^2+1} + \cos x}{x^3}$ (combination of radical and trig)
  \end{itemize}
\end{enumerate}
\end{tcolorbox}
\end{frame}

% --- Split: Why These Are Not Rational Expressions (Page 1) ---

\begin{frame}{Why These Are Not Rational Expressions (1/2)}
\begin{tcolorbox}[colback=lightgray,colframe=primary,title=Explanation: Radicals & Fractional Exponents]
\footnotesize
\begin{itemize}
  \item \textbf{Radicals}: 
  \begin{itemize}
    \item Cannot be written as polynomials
    \item Involve non-integer exponents
    \item Example: $\sqrt{x} = x^{1/2}$
  \end{itemize}

  \item \textbf{Fractional Exponents}:
  \begin{itemize}
    \item Not allowed in polynomials
    \item Cannot be simplified to integer exponents
    \item Example: $x^{3/2} = \sqrt{x^3}$
  \end{itemize}
\end{itemize}
\end{tcolorbox}
\end{frame}

% --- Split: Why These Are Not Rational Expressions (Page 2) ---

\begin{frame}{Why These Are Not Rational Expressions (2/2)}
\begin{tcolorbox}[colback=lightgray,colframe=primary,title=Explanation: Trig Functions & Logarithms]
\footnotesize
\begin{itemize}
  \item \textbf{Trigonometric Functions}:
  \begin{itemize}
    \item Not polynomials
    \item Cannot be expressed as finite sums of terms
    \item Example: $\sin x = x - \frac{x^3}{3!} + \frac{x^5}{5!} - ...$
  \end{itemize}

  \item \textbf{Logarithms}:
  \begin{itemize}
    \item Not polynomials
    \item Cannot be expressed as finite sums of terms
    \item Example: $\ln x$ cannot be written as a polynomial
  \end{itemize}
\end{itemize}
\end{tcolorbox}
\end{frame}

% III) Simplifying Rational Expressions
\begin{frame}{Simplifying Rational Expressions}
\begin{tcolorbox}[colback=lightgray,colframe=primary,title=Rules]
\footnotesize
\begin{itemize}
  \item You can only simplify fractions when you have a common factor in both the numerator \& denominator
  \item When simplifying binomials, factor out the common factor first, then simplify
  \item You can only cancel out a binomial when it's a common factor in both the Numerator \& Denominator
\end{itemize}
\end{tcolorbox}

\begin{tcolorbox}[colback=lightgray,colframe=accent,title=Common Mistakes]
\footnotesize
\begin{itemize}
  \item You CANNOT cancel a common term from the top and bottom if it is added or subtracted!
  \item Example: $\frac{x+3}{x+5} \neq \frac{3}{5}$
  \item Example: $\frac{x+5}{x-5} \neq 1$
  \item Example: $\frac{2x+10}{x+5} \neq 4$
\end{itemize}
\end{tcolorbox}
\end{frame}

% IV) Practice Problems
\begin{frame}{Practice: Factor and Simplify}
\begin{tcolorbox}[colback=lightgray,colframe=primary,title=Problems]
\footnotesize
Simplify each expression:
\begin{enumerate}
  \item $\frac{a^2 - 6a - 3}{a - 3}$
  \item $\frac{a^2 - 2ab - b^2}{a^2 + ab}$
  \item $\frac{3xy^2 - 18y}{y^2 - 2xy}$
  \item $\frac{3a^2 + 3ab - 60b^2}{2a^2 + 4ab - 48b^2}$
\end{enumerate}
\end{tcolorbox}
\end{frame}

\begin{frame}{Practice: Solutions Part 1}
\begin{tcolorbox}[colback=lightgray,colframe=accent,title=Detailed Solutions]
\footnotesize
\begin{enumerate}
  \item $\frac{a^2 - 6a - 3}{a - 3}$
  \begin{align*}
    &= \frac{(a-3)(a+1)}{a-3} \\
    &= a+1
  \end{align*}
  \item $\frac{a^2 - 2ab - b^2}{a^2 + ab}$
  \begin{align*}
    &= \frac{(a-b)(a+b)}{a(a+b)} \\
    &= \frac{a-b}{a}
  \end{align*}
\end{enumerate}
\end{tcolorbox}
\end{frame}

\begin{frame}{Practice: Solutions Part 2}
\begin{tcolorbox}[colback=lightgray,colframe=accent,title=Detailed Solutions]
\footnotesize
\begin{enumerate}
  \setcounter{enumi}{2}
  \item $\frac{3xy^2 - 18y}{y^2 - 2xy}$
  \begin{align*}
    &= \frac{3y(xy-6)}{y(y-2x)} \\
    &= \frac{3(xy-6)}{y-2x}
  \end{align*}
  \item $\frac{3a^2 + 3ab - 60b^2}{2a^2 + 4ab - 48b^2}$
  \begin{align*}
    &= \frac{3(a^2 + ab - 20b^2)}{2(a^2 + 2ab - 24b^2)} \\
    &= \frac{3(a+5b)(a-4b)}{2(a+6b)(a-4b)} \\
    &= \frac{3(a+5b)}{2(a+6b)}
  \end{align*}
\end{enumerate}
\end{tcolorbox}
\end{frame}

% V) Non-Permissible Values (NPV)
\begin{frame}{Non-Permissible Values (NPV)}
\begin{tcolorbox}[colback=lightgray,colframe=primary,title=Definition]
\footnotesize
\begin{itemize}
  \item Permissible → Allowed, Non-Permissible → Not Allowed
  \item When evaluating rational expressions, the denominator is not allowed to be Zero
  \item Can't divide by zero → Undefined!
  \item Any value of "x" that makes the denominator equal to zero is called a NPV
\end{itemize}
\end{tcolorbox}

\begin{tcolorbox}[colback=lightgray,colframe=secondary,title=Steps to Find NPV]
\footnotesize
\begin{enumerate}
  \item Take the entire denominator
  \item Make it equal to zero
  \item Solve for "x"
  \item Factor the denominator
  \item These are values that "x" cannot be
\end{enumerate}
\end{tcolorbox}
\end{frame}

% VI) NPV Practice
\begin{frame}{Find the NPV}
\begin{tcolorbox}[colback=lightgray,colframe=primary,title=Practice]
\footnotesize
Find the non-permissible values for each expression:
\begin{enumerate}
  \item $\frac{3x - 6}{6x + 2}$
  \item $\frac{2x^2 - 2x - 8}{4x^2 - 81}$
  \item $\frac{5x^2 - 10x + 12}{x^2 - 11x + 30}$
  \item $\frac{2x^2 - 2x - 8}{x^2 - 7xy + 10y^2}$
\end{enumerate}
\end{tcolorbox}
\end{frame}

\begin{frame}{NPV Solutions Part 1}
\begin{tcolorbox}[colback=lightgray,colframe=accent,title=Detailed Solutions]
\footnotesize
\begin{enumerate}
  \item $\frac{3x - 6}{6x + 2}$
  \begin{align*}
    6x + 2 &= 0 \\
    6x &= -2 \\
    x &= -\frac{1}{3}
  \end{align*}
  NPV: $x \neq -\frac{1}{3}$

  \item $\frac{2x^2 - 2x - 8}{4x^2 - 81}$
  \begin{align*}
    4x^2 - 81 &= 0 \\
    (2x+9)(2x-9) &= 0 \\
    x &= -\frac{9}{2} \text{ or } x = \frac{9}{2}
  \end{align*}
  NPV: $x \neq \pm\frac{9}{2}$
\end{enumerate}
\end{tcolorbox}
\end{frame}

\begin{frame}{NPV Solutions Part 2}
\begin{tcolorbox}[colback=lightgray,colframe=accent,title=Detailed Solutions]
\footnotesize
\begin{enumerate}
  \setcounter{enumi}{2}
  \item $\frac{5x^2 - 10x + 12}{x^2 - 11x + 30}$
  \begin{align*}
    x^2 - 11x + 30 &= 0 \\
    (x-5)(x-6) &= 0 \\
    x &= 5 \text{ or } x = 6
  \end{align*}
  NPV: $x \neq 5, 6$

  \item $\frac{2x^2 - 2x - 8}{x^2 - 7xy + 10y^2}$
  \begin{align*}
    x^2 - 7xy + 10y^2 &= 0 \\
    (x-5y)(x-2y) &= 0 \\
    x &= 5y \text{ or } x = 2y
  \end{align*}
  NPV: $x \neq 5y, 2y$
\end{enumerate}
\end{tcolorbox}
\end{frame}

% Additional Challenging NPV Problems
\begin{frame}{Additional Challenging NPV Problems}
\begin{tcolorbox}[colback=lightgray,colframe=primary,title=Advanced Practice]
\footnotesize
Find the non-permissible values for each expression:
\begin{enumerate}
  \item $\frac{x^3 - 8}{x^4 - 16}$
  \item $\frac{x^2 - 9}{x^3 - 27}$
  \item $\frac{x^4 - 1}{x^3 + x^2 - x - 1}$
  \item $\frac{x^3 - 2x^2 - 5x + 6}{x^4 - 5x^2 + 4}$
\end{enumerate}
\end{tcolorbox}
\end{frame}

\begin{frame}{Challenging NPV Solutions Part 1}
\begin{tcolorbox}[colback=lightgray,colframe=accent,title=Detailed Solutions]
\footnotesize
\begin{enumerate}
  \item $\frac{x^3 - 8}{x^4 - 16}$
  \begin{align*}
    x^4 - 16 &= 0 \\
    (x^2+4)(x^2-4) &= 0 \\
    (x^2+4)(x+2)(x-2) &= 0 \\
    x &= \pm 2
  \end{align*}
  NPV: $x \neq \pm 2$

  \item $\frac{x^2 - 9}{x^3 - 27}$
  \begin{align*}
    x^3 - 27 &= 0 \\
    (x-3)(x^2+3x+9) &= 0 \\
    x &= 3
  \end{align*}
  NPV: $x \neq 3$
\end{enumerate}
\end{tcolorbox}
\end{frame}

\begin{frame}{Challenging NPV Solutions Part 2}
\begin{tcolorbox}[colback=lightgray,colframe=accent,title=Detailed Solutions]
\footnotesize
\begin{enumerate}
  \setcounter{enumi}{2}
  \item $\frac{x^4 - 1}{x^3 + x^2 - x - 1}$
  \begin{align*}
    x^3 + x^2 - x - 1 &= 0 \\
    (x+1)(x^2-1) &= 0 \\
    (x+1)(x+1)(x-1) &= 0 \\
    x &= \pm 1
  \end{align*}
  NPV: $x \neq \pm 1$

  \item $\frac{x^3 - 2x^2 - 5x + 6}{x^4 - 5x^2 + 4}$
  \begin{align*}
    x^4 - 5x^2 + 4 &= 0 \\
    (x^2-4)(x^2-1) &= 0 \\
    (x+2)(x-2)(x+1)(x-1) &= 0 \\
    x &= \pm 2, \pm 1
  \end{align*}
  NPV: $x \neq \pm 2, \pm 1$
\end{enumerate}
\end{tcolorbox}
\end{frame}

% VII) Summary
\begin{frame}{Summary}
\begin{tcolorbox}[colback=lightgray,colframe=primary,title=Key Points]
\footnotesize
\begin{itemize}
  \item Rational expressions are fractions with polynomials
  \item Simplify by factoring and canceling common factors
  \item Watch out for common mistakes in simplification
  \item Always find and state non-permissible values
  \item Remember: denominator cannot be zero
\end{itemize}
\end{tcolorbox}
\end{frame}

\end{document} 