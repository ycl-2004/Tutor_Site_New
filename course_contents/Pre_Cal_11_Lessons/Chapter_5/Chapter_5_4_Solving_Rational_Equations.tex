\documentclass[aspectratio=169]{beamer}
\usepackage{amsmath}
\usepackage{amssymb}
\usepackage{graphicx}
\usepackage{tcolorbox}
\usepackage{booktabs}
\usepackage{colortbl}
\usepackage{xcolor}
\usepackage{tikz}
\usepackage[utf8]{inputenc}

% Custom colors
\definecolor{primary}{RGB}{41, 128, 185}
\definecolor{secondary}{RGB}{52, 152, 219}
\definecolor{accent}{RGB}{231, 76, 60}
\definecolor{lightgray}{RGB}{236, 240, 241}

% Theme customization
\usetheme{Madrid}
\usecolortheme{whale}
\setbeamercolor{structure}{fg=primary}
\setbeamercolor{background canvas}{bg=white}
\setbeamercolor{normal text}{fg=black}

\title{Chapter 5.4: Solving Rational Equations}
\subtitle{Rational Expressions - Lesson 4}
\author{Created by Yi-Chen Lin}
\date{\today}

\begin{document}

\begin{frame}
\titlepage
\end{frame}

% I) Solving Rational Equations
\begin{frame}{Solving Rational Equations}
\begin{tcolorbox}[colback=lightgray,colframe=primary,title=Key Concepts]
\footnotesize
\begin{itemize}
  \item Rational equations have an equal sign and require finding the value of $x$ that makes both sides equal.
  \item To solve, find the LCD and multiply all terms by the LCD to clear denominators (only when solving, not simplifying).
  \item Always check for NPV (Non-Permissible Values). If a solution is an NPV, it is called an "Extraneous Root" and must be excluded.
\end{itemize}
\end{tcolorbox}
\end{frame}

% Example: Using LCD to Clear Denominators (Q1)
\begin{frame}{Example Q1: Using LCD to Clear Denominators (Part 1)}
\begin{tcolorbox}[colback=lightgray,colframe=secondary,title=Example Q1 (Part 1)]
\footnotesize
Solve for $x$:
\[
\frac{2}{x} + \frac{3}{x+1} = 4
\]
\textbf{Step 1: Find LCD:} $x(x+1)$
\[
\frac{2}{x} \times (x+1) + \frac{3}{x+1} \times x = 4(x)(x+1)
\]
\textbf{Step 2: Multiply both sides by LCD:}
\[
2(x+1) + 3x = 4x(x+1)
\]
\textbf{Step 3: Expand:}
\[
2x+2+3x = 4x^2+4x
5x+2 = 4x^2+4x
\]
\end{tcolorbox}
\end{frame}

\begin{frame}{Example Q1: Using LCD to Clear Denominators (Part 2)}
\begin{tcolorbox}[colback=lightgray,colframe=secondary,title=Example Q1 (Part 2)]
\footnotesize
\textbf{Step 4: Rearrange and solve:}
\[
0 = 4x^2-x-2
\]
\textbf{Step 5: Solve the quadratic:}
\[
4x^2-x-2=0
\]
\textbf{Step 6: Check for NPV:} $x \neq 0, -1$
\end{tcolorbox}
\end{frame}

% Example: Cross-Multiplication (Q2)
\begin{frame}{Example Q2: Cross-Multiplication (Part 1)}
\begin{tcolorbox}[colback=lightgray,colframe=secondary,title=Example Q2 (Part 1)]
\footnotesize
Solve for $x$:
\[
\frac{4}{x+1} = \frac{3}{x}
\]
\textbf{Step 1: Cross-multiply:}
\[
4x = 3(x+1)
\]
\textbf{Step 2: Expand:}
\[
4x = 3x + 3
\]
\end{tcolorbox}
\end{frame}

\begin{frame}{Example Q2: Cross-Multiplication (Part 2)}
\begin{tcolorbox}[colback=lightgray,colframe=secondary,title=Example Q2 (Part 2)]
\footnotesize
\textbf{Step 3: Solve:}
\[
4x - 3x = 3
x = 3
\]
\textbf{Step 4: NPV:} $x \neq 0, -1$
\textbf{Final Answer:} $x=3$ (valid)
\end{tcolorbox}
\end{frame}

% Example: LCM (Q3)
\begin{frame}{Example Q3: Using LCM (Part 1)}
\begin{tcolorbox}[colback=lightgray,colframe=secondary,title=Example Q3 (Part 1)]
\footnotesize
Solve for $x$:
\[
\frac{1}{6} + \frac{1}{8} = \frac{5}{x}
\]
\textbf{Step 1: Find LCM of 6, 8, and $x$ (assume $x$ is a positive integer):} LCM$=24x$
\textbf{Step 2: Multiply both sides by LCM:}
\[
\frac{1}{6} \times 24x + \frac{1}{8} \times 24x = \frac{5}{x} \times 24x
\]
\textbf{Step 3: Simplify:}
\[
4x + 3x = 5 \times 24
7x = 120
\]
\end{tcolorbox}
\end{frame}

\begin{frame}{Example Q3: Using LCM (Part 2)}
\begin{tcolorbox}[colback=lightgray,colframe=secondary,title=Example Q3 (Part 2)]
\footnotesize
\textbf{Step 4: Solve:}
\[
x = \frac{120}{7}
\]
\textbf{Step 5: NPV:} $x \neq 0$
\textbf{Final Answer:} $x=\frac{120}{7}$ (valid)
\end{tcolorbox}
\end{frame}

% Practice: Solve for x
\begin{frame}{Practice: Solve for $x$}
\begin{tcolorbox}[colback=lightgray,colframe=primary,title=Practice Problems]
\footnotesize
\begin{enumerate}
  \item[Q4.] $\frac{3}{x} + \frac{4}{x+2} = 1$
  \item[Q5.] $\frac{2}{x} + \frac{1}{x-1} = 3$
  \item[Q6.] $\frac{5}{x-2} - \frac{2}{x+3} = 1$
  \item[Q7.] $\frac{1}{6} + \frac{1}{8} = \frac{5}{x}$
  \item[Q8.] $\frac{2}{x+1} = \frac{3}{x-2}$
\end{enumerate}
\end{tcolorbox}
\end{frame}

% II) Extraneous, No, and Infinite Solutions
\begin{frame}{Extraneous, No, and Infinite Solutions}
\begin{tcolorbox}[colback=lightgray,colframe=primary,title=Special Cases]
\footnotesize
\begin{itemize}
  \item If a solution is an NPV, it is extraneous and not valid.
  \item If all $x$ cancel and both sides are not equal, there are no solutions.
  \item If all $x$ cancel and both sides are equal, there are infinite solutions.
\end{itemize}
\end{tcolorbox}
\end{frame}

% Example: Extraneous Solution (Q9)
\begin{frame}{Example Q9: Extraneous Solution (Part 1)}
\begin{tcolorbox}[colback=lightgray,colframe=secondary,title=Example Q9 (Part 1)]
\footnotesize
Solve for $x$:
\[
\frac{3}{x+2} = \frac{5}{x+2}
\]
\textbf{Step 1: LCD is } $x+2$
\[
3 = 5
\]
\end{tcolorbox}
\end{frame}

\begin{frame}{Example Q9: Extraneous Solution (Part 2)}
\begin{tcolorbox}[colback=lightgray,colframe=secondary,title=Example Q9 (Part 2)]
\footnotesize
No solution (all $x$ cancel, both sides not equal).
\end{tcolorbox}
\end{frame}

% Practice: Special Cases
\begin{frame}{Practice: Special Cases}
\begin{tcolorbox}[colback=lightgray,colframe=primary,title=Practice Problems]
\footnotesize
\begin{enumerate}
  \item[Q10.] $\frac{2}{x-1} = \frac{2}{x-1}$
  \item[Q11.] $\frac{4}{x+2} = \frac{4}{x+2}$
  \item[Q12.] $\frac{3}{x} = \frac{5}{x}$
\end{enumerate}
\end{tcolorbox}
\end{frame}

% III) Solving by Graphs
\begin{frame}{Solving Rational Equations by Graphs}
\begin{tcolorbox}[colback=lightgray,colframe=primary,title=Graphical Solution]
\footnotesize
\begin{itemize}
  \item Let $y_1$ be the left side and $y_2$ be the right side of the equation.
  \item The solution is the $x$-value where $y_1 = y_2$ (the intersection point).
  \item Always check for NPV.
\end{itemize}
\end{tcolorbox}
\end{frame}

% Challenge Problem Q13
\begin{frame}{Challenge Problem Q13}
\begin{tcolorbox}[colback=lightgray,colframe=primary,title=Challenge Q13]
\footnotesize
Solve for $x$ and state all NPV's:
\[
\frac{2}{x^2-4} + \frac{3}{x+2} = 1
\]
\end{tcolorbox}
\end{frame}

\begin{frame}{Challenge Q13: Solution (Part 1)}
\begin{tcolorbox}[colback=lightgray,colframe=accent,title=Step-by-Step Solution (Part 1)]
\footnotesize
\textbf{Step 1: Factor denominators}
\[
x^2-4 = (x+2)(x-2)
\]
\textbf{Step 2: LCD is } $(x+2)(x-2)$
\textbf{Step 3: Rewrite each fraction with LCD}
\[
\frac{2}{(x+2)(x-2)} + \frac{3(x-2)}{(x+2)(x-2)} = 1
\]
\end{tcolorbox}
\end{frame}

\begin{frame}{Challenge Q13: Solution (Part 2)}
\begin{tcolorbox}[colback=lightgray,colframe=accent,title=Step-by-Step Solution (Part 2)]
\footnotesize
\textbf{Step 4: Combine numerators}
\[
\frac{2 + 3(x-2)}{(x+2)(x-2)} = 1
\]
\[
2 + 3x - 6 = (x+2)(x-2)
\]
\[
3x - 4 = x^2 - 4
\]
\[
0 = x^2 - 3x
\]
\[
x(x-3) = 0
\]
\textbf{Step 5: Solutions:} $x=0, x=3$
\textbf{Step 6: NPV:} $x \neq 2, -2$
\textbf{Final Answer:} $x=0, x=3$ (both are valid)
\end{tcolorbox}
\end{frame}

% Challenge Problem Q14 (multi-page)
\begin{frame}{Challenge Problem Q14}
\begin{tcolorbox}[colback=lightgray,colframe=primary,title=Challenge Q14]
\footnotesize
Solve for $x$ and state all NPV's:
\[
\frac{1}{x-1} + \frac{2}{x+2} = \frac{3x+1}{x^2+x-2}
\]
\end{tcolorbox}
\end{frame}

\begin{frame}{Challenge Q14: Solution (Part 1)}
\begin{tcolorbox}[colback=lightgray,colframe=accent,title=Step-by-Step Solution (Part 1)]
\footnotesize
\textbf{Step 1: Factor denominators}
\[
x^2+x-2 = (x-1)(x+2)
\]
\textbf{Step 2: LCD is } $(x-1)(x+2)$
\textbf{Step 3: Rewrite each fraction with LCD}
\[
\frac{1}{x-1} \times \frac{x+2}{x+2} = \frac{x+2}{(x-1)(x+2)}
\]
\[
\frac{2}{x+2} \times \frac{x-1}{x-1} = \frac{2(x-1)}{(x-1)(x+2)}
\]
\end{tcolorbox}
\end{frame}

\begin{frame}{Challenge Q14: Solution (Part 2)}
\begin{tcolorbox}[colback=lightgray,colframe=accent,title=Step-by-Step Solution (Part 2)]
\footnotesize
\textbf{Step 4: Combine numerators}
\[
\frac{x+2 + 2(x-1)}{(x-1)(x+2)} = \frac{x+2+2x-2}{(x-1)(x+2)} = \frac{3x}{(x-1)(x+2)}
\]
\textbf{Step 5: Set equal to right side and solve:}
\[
\frac{3x}{(x-1)(x+2)} = \frac{3x+1}{(x-1)(x+2)}
\]
\[
3x = 3x+1
\]
No solution (contradiction).
\textbf{Step 6: NPV:} $x \neq 1, -2$
\end{tcolorbox}
\end{frame}

% Solution for Q4
\begin{frame}{Solution Q4 (Part 1)}
\begin{tcolorbox}[colback=lightgray,colframe=accent,title=Solution Q4 (Part 1)]
\footnotesize
Solve for $x$:
\[
\frac{3}{x} + \frac{4}{x+2} = 1
\]
\textbf{Step 1: Find LCD:} $x(x+2)$
\textbf{Step 2: Multiply both sides by LCD:}
\[
\frac{3}{x} \times (x+2) + \frac{4}{x+2} \times x = 1 \times x(x+2)
\]
\textbf{Step 3: Expand:}
\[
3(x+2) + 4x = x^2 + 2x
\]
\end{tcolorbox}
\end{frame}

\begin{frame}{Solution Q4 (Part 2)}
\begin{tcolorbox}[colback=lightgray,colframe=accent,title=Solution Q4 (Part 2)]
\footnotesize
\textbf{Step 4: Simplify and solve:}
\[
3x + 6 + 4x = x^2 + 2x
7x + 6 = x^2 + 2x
0 = x^2 - 5x - 6
\]
\textbf{Step 5: Factor and solve:}
\[
x^2 - 5x - 6 = 0 \\
(x-6)(x+1) = 0 \\
x = 6,\ -1
\]
\textbf{Step 6: NPV:} $x \neq 0, -2$
\textbf{Final Answer:} $x=6$ (valid), $x=-1$ (valid)
\end{tcolorbox}
\end{frame}

% Solution for Q5
\begin{frame}{Solution Q5 (Part 1)}
\begin{tcolorbox}[colback=lightgray,colframe=accent,title=Solution Q5 (Part 1)]
\footnotesize
Solve for $x$:
\[
\frac{2}{x} + \frac{1}{x-1} = 3
\]
\textbf{Step 1: Find LCD:} $x(x-1)$
\textbf{Step 2: Multiply both sides by LCD:}
\[
\frac{2}{x} \times (x-1) + \frac{1}{x-1} \times x = 3x(x-1)
\]
\textbf{Step 3: Expand:}
\[
2(x-1) + x = 3x^2 - 3x
\]
\end{tcolorbox}
\end{frame}

\begin{frame}{Solution Q5 (Part 2)}
\begin{tcolorbox}[colback=lightgray,colframe=accent,title=Solution Q5 (Part 2)]
\footnotesize
\textbf{Step 4: Simplify and solve:}
\[
2x - 2 + x = 3x^2 - 3x
3x - 2 = 3x^2 - 3x
0 = 3x^2 - 6x + 2
\]
\textbf{Step 5: Quadratic formula:}
\[
x = \frac{6 \pm \sqrt{36 - 24}}{6} = \frac{6 \pm 2\sqrt{3}}{6} = 1 \pm \frac{\sqrt{3}}{3}
\]
\textbf{Step 6: NPV:} $x \neq 0, 1$
\textbf{Final Answer:} $x = 1 + \frac{\sqrt{3}}{3}$, $x = 1 - \frac{\sqrt{3}}{3}$ (both valid)
\end{tcolorbox}
\end{frame}

% Solution for Q6
\begin{frame}{Solution Q6 (Part 1)}
\begin{tcolorbox}[colback=lightgray,colframe=accent,title=Solution Q6 (Part 1)]
\footnotesize
Solve for $x$:
\[
\frac{5}{x-2} - \frac{2}{x+3} = 1
\]
\textbf{Step 1: Find LCD:} $(x-2)(x+3)$
\textbf{Step 2: Multiply both sides by LCD:}
\[
\frac{5}{x-2} \times (x+3) - \frac{2}{x+3} \times (x-2) = (x-2)(x+3)
\]
\textbf{Step 3: Expand:}
\[
5(x+3) - 2(x-2) = (x-2)(x+3)
\]
\end{tcolorbox}
\end{frame}

\begin{frame}{Solution Q6 (Part 2)}
\begin{tcolorbox}[colback=lightgray,colframe=accent,title=Solution Q6 (Part 2)]
\footnotesize
\textbf{Step 4: Simplify and solve:}
\[
5x + 15 - 2x + 4 = x^2 + 3x - 2x - 6
3x + 19 = x^2 + x - 6
0 = x^2 - 2x - 25
\]
\textbf{Step 5: Factor or quadratic formula:}
\[
x = \frac{2 \pm \sqrt{4 + 100}}{2} = \frac{2 \pm \sqrt{104}}{2} = \frac{2 \pm 2\sqrt{26}}{2} = 1 \pm \sqrt{26}
\]
\textbf{Step 6: NPV:} $x \neq 2, -3$
\textbf{Final Answer:} $x = 1 + \sqrt{26}$, $x = 1 - \sqrt{26}$ (both valid)
\end{tcolorbox}
\end{frame}

% Solution for Q7
\begin{frame}{Solution Q7 (Part 1)}
\begin{tcolorbox}[colback=lightgray,colframe=accent,title=Solution Q7 (Part 1)]
\footnotesize
Solve for $x$:
\[
\frac{1}{6} + \frac{1}{8} = \frac{5}{x}
\]
\textbf{Step 1: Find LCM:} $24x$
\textbf{Step 2: Multiply both sides by LCM:}
\[
4x + 3x = 5 \times 24
\]
\textbf{Step 3: Simplify:}
\[
7x = 120
\]
\end{tcolorbox}
\end{frame}

\begin{frame}{Solution Q7 (Part 2)}
\begin{tcolorbox}[colback=lightgray,colframe=accent,title=Solution Q7 (Part 2)]
\footnotesize
\textbf{Step 4: Solve:}
\[
x = \frac{120}{7}
\]
\textbf{Step 5: NPV:} $x \neq 0$
\textbf{Final Answer:} $x=\frac{120}{7}$ (valid)
\end{tcolorbox}
\end{frame}

% Solution for Q8
\begin{frame}{Solution Q8 (Part 1)}
\begin{tcolorbox}[colback=lightgray,colframe=accent,title=Solution Q8 (Part 1)]
\footnotesize
Solve for $x$:
\[
\frac{2}{x+1} = \frac{3}{x-2}
\]
\textbf{Step 1: Cross-multiply:}
\[
2(x-2) = 3(x+1)
\]
\textbf{Step 2: Expand:}
\[
2x - 4 = 3x + 3
\]
\end{tcolorbox}
\end{frame}

\begin{frame}{Solution Q8 (Part 2)}
\begin{tcolorbox}[colback=lightgray,colframe=accent,title=Solution Q8 (Part 2)]
\footnotesize
\textbf{Step 3: Solve:}
\[
2x - 4 = 3x + 3
-4 - 3 = 3x - 2x
-7 = x
x = -7
\]
\textbf{Step 4: NPV:} $x \neq -1, 2$
\textbf{Final Answer:} $x = -7$ (valid)
\end{tcolorbox}
\end{frame}

% Solution for Q10
\begin{frame}{Solution Q10 (Part 1)}
\begin{tcolorbox}[colback=lightgray,colframe=accent,title=Solution Q10 (Part 1)]
\footnotesize
Solve for $x$:
\[
\frac{2}{x-1} = \frac{2}{x-1}
\]
\textbf{Step 1: Both sides are identical.}
\textbf{Step 2: Any $x \neq 1$ is a solution.}
\end{tcolorbox}
\end{frame}

\begin{frame}{Solution Q10 (Part 2)}
\begin{tcolorbox}[colback=lightgray,colframe=accent,title=Solution Q10 (Part 2)]
\footnotesize
\textbf{Final Answer:} Infinite solutions for $x \neq 1$ (NPV: $x \neq 1$)
\end{tcolorbox}
\end{frame}

% Solution for Q11
\begin{frame}{Solution Q11 (Part 1)}
\begin{tcolorbox}[colback=lightgray,colframe=accent,title=Solution Q11 (Part 1)]
\footnotesize
Solve for $x$:
\[
\frac{4}{x+2} = \frac{4}{x+2}
\]
\textbf{Step 1: Both sides are identical.}
\textbf{Step 2: Any $x \neq -2$ is a solution.}
\end{tcolorbox}
\end{frame}

\begin{frame}{Solution Q11 (Part 2)}
\begin{tcolorbox}[colback=lightgray,colframe=accent,title=Solution Q11 (Part 2)]
\footnotesize
\textbf{Final Answer:} Infinite solutions for $x \neq -2$ (NPV: $x \neq -2$)
\end{tcolorbox}
\end{frame}

% Solution for Q12
\begin{frame}{Solution Q12 (Part 1)}
\begin{tcolorbox}[colback=lightgray,colframe=accent,title=Solution Q12 (Part 1)]
\footnotesize
Solve for $x$:
\[
\frac{3}{x} = \frac{5}{x}
\]
\textbf{Step 1: Subtract $\frac{5}{x}$ from both sides:}
\[
\frac{3}{x} - \frac{5}{x} = 0
\]
\textbf{Step 2: Combine:}
\[
\frac{-2}{x} = 0
\]
\end{tcolorbox}
\end{frame}

\begin{frame}{Solution Q12 (Part 2)}
\begin{tcolorbox}[colback=lightgray,colframe=accent,title=Solution Q12 (Part 2)]
\footnotesize
\textbf{Step 3: $-2 = 0$ is a contradiction.}
\textbf{Final Answer:} No solution (NPV: $x \neq 0$)
\end{tcolorbox}
\end{frame}

\end{document} 