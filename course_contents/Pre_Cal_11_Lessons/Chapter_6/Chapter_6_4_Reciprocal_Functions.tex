% !TEX program = xelatex
\documentclass[aspectratio=169]{beamer}
\usepackage{amsmath}
\usepackage{amssymb}
\usepackage{graphicx}
\usepackage{tcolorbox}
\usepackage{booktabs}
\usepackage{colortbl}
\usepackage{xcolor}
\usepackage{tikz}
\usetikzlibrary{intersections,decorations.pathreplacing,angles,quotes}
\usepackage[utf8]{inputenc}

% Custom colors
\definecolor{primary}{RGB}{41, 128, 185}
\definecolor{secondary}{RGB}{52, 152, 219}
\definecolor{accent}{RGB}{231, 76, 60}
\definecolor{lightgray}{RGB}{236, 240, 241}

% Theme customization
\usetheme{Madrid}
\usecolortheme{whale}
\setbeamercolor{structure}{fg=primary}
\setbeamercolor{background canvas}{bg=white}
\setbeamercolor{normal text}{fg=black}

% Title page info
\title{Pre-Calculus 11}
\subtitle{\textbf{6.4 Reciprocal Functions}}
\author{Created by Yi-Chen Lin}
\date{\today}

\begin{document}

% Title Page
\begin{frame}
    \titlepage
    \vfill
\end{frame}

% Overview
\begin{frame}{Overview 概述}
    \begin{tcolorbox}[colback=lightgray,colframe=primary,title=What We Will Learn 学习目标]
        \footnotesize
        \begin{enumerate}
            \item Understand the concept of reciprocal functions
            \item Learn about asymptotes and invariant points
            \item Graph reciprocal functions of linear and quadratic functions
            \item Determine domain and range of reciprocal functions
            \item Apply these concepts to solve problems
        \end{enumerate}
    \end{tcolorbox}
    \vspace{1em}
    \textbf{Key Questions:}
    \begin{itemize}
        \item How do we find the reciprocal of a function?
        \item What happens to the graph when we take the reciprocal?
        \item Where do asymptotes occur and why?
        \item What are invariant points and why are they important?
    \end{itemize}
\end{frame}

% Section: What is a Reciprocal?
\begin{frame}{What is a Reciprocal? 什么是倒数?}
    \begin{tcolorbox}[colback=lightgray,colframe=primary,title=Definition 定义]
        \footnotesize
        The reciprocal of a number $a$ is $\frac{1}{a}$, where $a \neq 0$.
        \par
        一个数$a$的倒数是$\frac{1}{a}$,其中$a \neq 0$。
    \end{tcolorbox}
    \vspace{1em}
    \textbf{Key Properties 重要性质:}
    \begin{itemize}
        \item $a \cdot \frac{1}{a} = 1$ (product of a number and its reciprocal is 1)
        \item The reciprocal of 1 is 1, the reciprocal of -1 is -1
        \item The reciprocal of 0 is undefined
        \item Taking the reciprocal preserves the sign
    \end{itemize}
\end{frame}

\begin{frame}{Examples of Reciprocals 倒数的例子}
    \footnotesize
    \textbf{Complete the table:}
    \begin{center}
        \begin{tabular}{|c|c|c|}
            \hline
            \textbf{Number} & \textbf{Reciprocal} & \textbf{Check} \\
            \hline
            2 & $\frac{1}{2}$ & $2 \cdot \frac{1}{2} = 1$ \\
            \hline
            -5 & $-\frac{1}{5}$ & $-5 \cdot (-\frac{1}{5}) = 1$ \\
            \hline
            0.5 & 2 & $0.5 \cdot 2 = 1$ \\
            \hline
            1 & 1 & $1 \cdot 1 = 1$ \\
            \hline
            -1 & -1 & $-1 \cdot (-1) = 1$ \\
            \hline
            0 & Undefined & Cannot divide by zero \\
            \hline
        \end{tabular}
    \end{center}
    \vspace{1em}
    \textbf{Pattern:} Small numbers have large reciprocals, large numbers have small reciprocals.
\end{frame}

% Section: Reciprocal of a Function
\begin{frame}{The Reciprocal of a Function 函数的倒数}
    \begin{tcolorbox}[colback=lightgray,colframe=primary,title=Definition 定义]
        \footnotesize
        Given a function $f(x)$, its reciprocal function is $y = \frac{1}{f(x)}$.
        \par
        给定函数$f(x)$,其倒数函数是$y = \frac{1}{f(x)}$。
    \end{tcolorbox}
    \vspace{1em}
    \textbf{Examples 例子:}
    \begin{itemize}
        \item If $f(x) = x + 3$, then $y = \frac{1}{x + 3}$
        \item If $f(x) = 2x^2 - 5$, then $y = \frac{1}{2x^2 - 5}$
        \item If $f(x) = \sqrt{x}$, then $y = \frac{1}{\sqrt{x}}$
    \end{itemize}
    \vspace{1em}
    \textbf{Important Notes 重要注意:}
    \begin{itemize}
        \item The reciprocal function is undefined where $f(x) = 0$
        \item The sign of $f(x)$ is preserved in the reciprocal
        \item The reciprocal function has different behavior than the original function
    \end{itemize}
\end{frame}

% Section: Asymptotes and Invariant Points
\begin{frame}{Asymptotes and Invariant Points 渐近线与不变点}
    \begin{tcolorbox}[colback=lightgray,colframe=primary,title=Key Concepts 关键概念]
        \footnotesize
        \begin{enumerate}
            \item \textbf{Vertical Asymptotes 垂直渐近线}: Occur where $f(x) = 0$
            \item \textbf{Horizontal Asymptote 水平渐近线}: Usually $y = 0$ as $x \to \pm\infty$
            \item \textbf{Invariant Points 不变点}: Where $f(x) = 1$ or $f(x) = -1$
        \end{enumerate}
    \end{tcolorbox}
    \vspace{1em}
    \textbf{Why These Points Are Important 为什么这些点重要:}
    \begin{itemize}
        \item Vertical asymptotes: Division by zero is undefined
        \item Invariant points: $\frac{1}{1} = 1$ and $\frac{1}{-1} = -1$
        \item These points help us sketch the graph accurately
    \end{itemize}
\end{frame}

\begin{frame}{Understanding Asymptotes 理解渐近线}
    \footnotesize
    \textbf{Vertical Asymptotes 垂直渐近线:}
    \begin{itemize}
        \item Occur when $f(x) = 0$ because $\frac{1}{0}$ is undefined
        \item The graph approaches but never touches these lines
        \item Example: For $f(x) = x - 2$, the reciprocal has a vertical asymptote at $x = 2$
    \end{itemize}
    \vspace{1em}
    \textbf{Horizontal Asymptotes 水平渐近线:}
    \begin{itemize}
        \item Usually $y = 0$ because as $|f(x)|$ gets large, $\frac{1}{f(x)}$ gets small
        \item The graph approaches this line as $x \to \pm\infty$
        \item Example: For $f(x) = x^2$, as $x \to \infty$, $y = \frac{1}{x^2} \to 0$
    \end{itemize}
\end{frame}

% Section: Graphing the Reciprocal of Linear Functions
\begin{frame}{Graphing the Reciprocal of a Linear Function 线性函数的倒数图像}
    \footnotesize
    \textbf{Step-by-Step Process 逐步过程:}
    \begin{enumerate}
        \item Graph the original linear function $f(x)$
        \item Find where $f(x) = 0$ (vertical asymptote)
        \item Find where $f(x) = 1$ and $f(x) = -1$ (invariant points)
        \item Take the reciprocal of y-coordinates for other points
        \item Sketch the reciprocal function
    \end{enumerate}
    \vspace{1em}
    \textbf{Example:} $f(x) = x - 1$
    \begin{itemize}
        \item $f(x) = 0$ when $x = 1$ (vertical asymptote)
        \item $f(x) = 1$ when $x = 2$ (invariant point)
        \item $f(x) = -1$ when $x = 0$ (invariant point)
    \end{itemize}
\end{frame}

\begin{frame}{Graphing Example: $f(x) = x - 1$}
    \footnotesize
    \textbf{Original Function:} $f(x) = x - 1$ (linear)\\
    \textbf{Reciprocal Function:} $y = \frac{1}{x - 1}$\\
    \vspace{1em}
    \textbf{Key Points:}
    \begin{itemize}
        \item Vertical asymptote: $x = 1$
        \item Invariant points: $(2, 1)$ and $(0, -1)$
        \item As $x \to \infty$, $y \to 0^+$
        \item As $x \to -\infty$, $y \to 0^-$
    \end{itemize}
    \vspace{0.3em}
    \begin{center}
        \includegraphics[width=0.5\textwidth]{y_x-1.png}
    \end{center}
\end{frame}

% Enlarged image only for linear example
\begin{frame}{Graphing Example: $f(x) = x - 1$ (Enlarged)}
    \begin{center}
        \includegraphics[width=1.0\textwidth]{y_x-1.png}
    \end{center}
\end{frame}

% Section: Graphing the Reciprocal of a Quadratic Function
\begin{frame}{Graphing the Reciprocal of a Quadratic Function 二次函数的倒数图像}
    \footnotesize
    \textbf{Key Differences from Linear Functions 与线性函数的主要区别:}
    \begin{itemize}
        \item Quadratic functions can have 0, 1, or 2 x-intercepts
        \item Each x-intercept becomes a vertical asymptote
        \item Can have up to 4 invariant points (2 for $f(x) = 1$, 2 for $f(x) = -1$)
        \item The graph has more complex behavior
    \end{itemize}
    \vspace{1em}
    \textbf{Example:} $f(x) = x^2 - 4$
    \begin{itemize}
        \item $f(x) = 0$ at $x = 2$ and $x = -2$ (two vertical asymptotes)
        \item $f(x) = 1$ at $x = \pm\sqrt{5}$ (two invariant points)
        \item $f(x) = -1$ at $x = \pm\sqrt{3}$ (two invariant points)
    \end{itemize}
\end{frame}

\begin{frame}{Graphing Example: $f(x) = x^2 - 4$}
    \footnotesize
    \textbf{Original Function:} $f(x) = x^2 - 4$ (parabola)\\
    \textbf{Reciprocal Function:} $y = \frac{1}{x^2 - 4}$\\
    \vspace{1em}
    \textbf{Key Features:}
    \begin{itemize}
        \item Vertical asymptotes: $x = 2$ and $x = -2$
        \item Invariant points: $(\pm\sqrt{5}, 1)$ and $(\pm\sqrt{3}, -1)$
        \item Horizontal asymptote: $y = 0$
        \item Three distinct regions: left, middle, right
    \end{itemize}
    \vspace{0.3em}
    \begin{center}
        \includegraphics[width=0.5\textwidth]{y_x^2.png}
    \end{center}
\end{frame}

% Enlarged image only for quadratic example
\begin{frame}{Graphing Example: $f(x) = x^2 - 4$ (Enlarged)}
    \begin{center}
        \includegraphics[width=1.0\textwidth]{y_x^2.png}
    \end{center}
\end{frame}

% Section: Domain and Range
\begin{frame}{Domain and Range 定义域与值域}
    \begin{tcolorbox}[colback=lightgray,colframe=primary,title=Domain 定义域]
        \footnotesize
        All real numbers $x$ such that $f(x) \neq 0$.
        \par
        所有使$f(x) \neq 0$的实数$x$。
    \end{tcolorbox}
    \vspace{1em}
    \begin{tcolorbox}[colback=lightgray,colframe=primary,title=Range 值域]
        \footnotesize
        All real numbers $y \neq 0$ (except possibly at invariant points).
        \par
        所有$y \neq 0$的实数(除不变点外)。
    \end{tcolorbox}
    \vspace{1em}
    \textbf{Examples 例子:}
    \begin{itemize}
        \item For $y = \frac{1}{x + 2}$: Domain = $\{x | x \neq -2\}$, Range = $\{y | y \neq 0\}$
        \item For $y = \frac{1}{x^2}$: Domain = $\{x | x \neq 0\}$, Range = $\{y | y > 0\}$
    \end{itemize}
\end{frame}

% Section: Practice Problems
\begin{frame}{Practice: Find the Reciprocal Function and Key Features}
    \footnotesize
    For each function $f(x)$ below, complete the following:
    \begin{enumerate}
        \item Write the reciprocal function $y = \frac{1}{f(x)}$
        \item Find all vertical asymptotes
        \item Find all invariant points
        \item State the domain and range
        \item Describe the key features of the graph
    \end{enumerate}
    \vspace{1em}
    \textbf{Functions to analyze:}
    \begin{enumerate}[label=Q\arabic*]
        \item $f(x) = x + 2$
        \item $f(x) = 2x - 4$
        \item $f(x) = x^2 - 1$
        \item $f(x) = x^2 + 2x + 1$
        \item $f(x) = x^2 + 4$
    \end{enumerate}
\end{frame}

% Section: Practice Solutions (1 per slide)
\begin{frame}{Solution Q1: $f(x) = x + 2$}
    \footnotesize
    \textbf{Reciprocal Function:} $y = \frac{1}{x + 2}$\\
    \vspace{0.5em}
    \textbf{Vertical Asymptote:} $x = -2$ (where $f(x) = 0$)\\
    \vspace{0.5em}
    \textbf{Invariant Points:}
    \begin{itemize}
        \item $f(x) = 1$ when $x + 2 = 1$, so $x = -1$. Point: $(-1, 1)$
        \item $f(x) = -1$ when $x + 2 = -1$, so $x = -3$. Point: $(-3, -1)$
    \end{itemize}
    \vspace{0.5em}
    \textbf{Domain:} $\{x | x \neq -2\}$\\
    \textbf{Range:} $\{y | y \neq 0\}$\\
    \vspace{0.5em}
    \textbf{Graph Features:} Hyperbola with vertical asymptote at $x = -2$, approaches $y = 0$ as $x \to \pm\infty$
\end{frame}

\begin{frame}{Solution Q2: $f(x) = 2x - 4$}
    \footnotesize
    \textbf{Reciprocal Function:} $y = \frac{1}{2x - 4}$\\
    \vspace{0.5em}
    \textbf{Vertical Asymptote:} $x = 2$ (where $f(x) = 0$)\\
    \vspace{0.5em}
    \textbf{Invariant Points:}
    \begin{itemize}
        \item $f(x) = 1$ when $2x - 4 = 1$, so $x = 2.5$. Point: $(2.5, 1)$
        \item $f(x) = -1$ when $2x - 4 = -1$, so $x = 1.5$. Point: $(1.5, -1)$
    \end{itemize}
    \vspace{0.5em}
    \textbf{Domain:} $\{x | x \neq 2\}$\\
    \textbf{Range:} $\{y | y \neq 0\}$\\
    \vspace{0.5em}
    \textbf{Graph Features:} Hyperbola with vertical asymptote at $x = 2$, approaches $y = 0$ as $x \to \pm\infty$
\end{frame}

\begin{frame}{Solution Q3: $f(x) = x^2 - 1$}
    \footnotesize
    \textbf{Reciprocal Function:} $y = \frac{1}{x^2 - 1}$\\
    \vspace{0.5em}
    \textbf{Vertical Asymptotes:} $x = 1$ and $x = -1$ (where $f(x) = 0$)\\
    \vspace{0.5em}
    \textbf{Invariant Points:}
    \begin{itemize}
        \item $f(x) = 1$ when $x^2 - 1 = 1$, so $x^2 = 2$, $x = \pm\sqrt{2}$. Points: $(\sqrt{2}, 1)$, $(-\sqrt{2}, 1)$
        \item $f(x) = -1$ when $x^2 - 1 = -1$, so $x^2 = 0$, $x = 0$. Point: $(0, -1)$
    \end{itemize}
    \vspace{0.5em}
    \textbf{Domain:} $\{x | x \neq 1, x \neq -1\}$\\
    \textbf{Range:} $\{y | y \neq 0\}$\\
    \vspace{0.5em}
    \textbf{Graph Features:} Three regions separated by vertical asymptotes, approaches $y = 0$ as $x \to \pm\infty$
\end{frame}

\begin{frame}{Solution Q4: $f(x) = x^2 + 2x + 1 = (x + 1)^2$}
    \footnotesize
    \textbf{Reciprocal Function:} $y = \frac{1}{(x + 1)^2}$\\
    \vspace{0.5em}
    \textbf{Vertical Asymptote:} $x = -1$ (where $f(x) = 0$)\\
    \vspace{0.5em}
    \textbf{Invariant Points:}
    \begin{itemize}
        \item $f(x) = 1$ when $(x + 1)^2 = 1$, so $x + 1 = \pm 1$, $x = 0$ or $x = -2$. Points: $(0, 1)$, $(-2, 1)$
        \item $f(x) = -1$ when $(x + 1)^2 = -1$ (no real solution)
    \end{itemize}
    \vspace{0.5em}
    \textbf{Domain:} $\{x | x \neq -1\}$\\
    \textbf{Range:} $\{y | y > 0\}$ (since $(x + 1)^2 \geq 0$)\\
    \vspace{0.5em}
    \textbf{Graph Features:} Always positive, vertical asymptote at $x = -1$, approaches $y = 0$ as $x \to \pm\infty$
\end{frame}

\begin{frame}{Solution Q5: $f(x) = x^2 + 4$}
    \footnotesize
    \textbf{Reciprocal Function:} $y = \frac{1}{x^2 + 4}$\\
    \vspace{0.5em}
    \textbf{Vertical Asymptotes:} None (since $x^2 + 4 > 0$ for all real $x$)\\
    \vspace{0.5em}
    \textbf{Invariant Points:}
    \begin{itemize}
        \item $f(x) = 1$ when $x^2 + 4 = 1$, so $x^2 = -3$ (no real solution)
        \item $f(x) = -1$ when $x^2 + 4 = -1$, so $x^2 = -5$ (no real solution)
    \end{itemize}
    \vspace{0.5em}
    \textbf{Domain:} All real numbers $\mathbb{R}$\\
    \textbf{Range:} $\{y | 0 < y \leq \frac{1}{4}\}$ (maximum at $x = 0$)\\
    \vspace{0.5em}
    \textbf{Graph Features:} Bell-shaped curve, maximum at $(0, \frac{1}{4})$, approaches $y = 0$ as $x \to \pm\infty$
\end{frame}

% Additional Practice
\begin{frame}{Additional Practice: Match Functions with Their Reciprocals}
    \footnotesize
    Match each function with its reciprocal function:
    \begin{enumerate}[label=Q\arabic*]
        \item $f(x) = x - 3$
        \item $f(x) = x^2 - 9$
        \item $f(x) = x^2 + 1$
        \item $f(x) = 3x + 6$
    \end{enumerate}
    \vspace{1em}
    \textbf{Reciprocal Functions:}
    \begin{enumerate}[label=A\alph*]
        \item $y = \frac{1}{x - 3}$
        \item $y = \frac{1}{x^2 - 9}$
        \item $y = \frac{1}{x^2 + 1}$
        \item $y = \frac{1}{3x + 6}$
    \end{enumerate}
\end{frame}

% Summary
\begin{frame}{Summary 总结}
    \begin{tcolorbox}[colback=lightgray,colframe=primary,title=Key Points 要点]
        \footnotesize
        \begin{enumerate}
            \item The reciprocal of $f(x)$ is $y = \frac{1}{f(x)}$
            \item Vertical asymptotes occur where $f(x) = 0$
            \item Invariant points occur where $f(x) = 1$ or $f(x) = -1$
            \item Domain excludes values where $f(x) = 0$
            \item Range is usually all real numbers except $y = 0$
            \item The graph behavior changes significantly from the original function
        \end{enumerate}
    \end{tcolorbox}
    \vspace{1em}
    \textbf{Graphing Strategy 绘图策略:}
    \begin{enumerate}
        \item Find vertical asymptotes (where $f(x) = 0$)
        \item Find invariant points (where $f(x) = \pm 1$)
        \item Determine behavior as $x \to \pm\infty$
        \item Sketch the graph using these key features
    \end{enumerate}
\end{frame}

\end{document} 