\documentclass[12pt]{article}
\usepackage{amsmath}
\usepackage{amssymb}
\usepackage{graphicx}
\usepackage{tcolorbox}
\usepackage{enumitem}
\usepackage{geometry}
\usepackage{xcolor}

% Custom colors
\definecolor{primary}{RGB}{41, 128, 185}
\definecolor{secondary}{RGB}{52, 152, 219}
\definecolor{accent}{RGB}{231, 76, 60}
\definecolor{lightgray}{RGB}{236, 240, 241}

% Page setup
\geometry{a4paper, margin=1in}

% Title page info
\title{Pre-Calculus 11 \\ Chapter 6: Absolute Values and Reciprocal Functions}
\author{Created by Yi-Chen Lin}
\date{\today}

\begin{document}

\maketitle

\section*{Chapter Overview}
This chapter covers absolute values and reciprocal functions, including:
\begin{itemize}
    \item Understanding and evaluating absolute values
    \item Graphing absolute value functions and transformations
    \item Solving absolute value equations and inequalities
    \item Understanding reciprocal functions and their properties
    \item Solving equations with multiple absolute values
\end{itemize}

\section{6.1 Basics with Absolute Values}
\subsection*{Key Concepts}
\begin{tcolorbox}[colback=lightgray,colframe=primary,title=Absolute Value Definition]
    \begin{itemize}
        \item $|x| = \begin{cases} x, & \text{if } x \geq 0 \\ -x, & \text{if } x < 0 \end{cases}$
        \item $|x| = \sqrt{x^2}$
        \item Absolute value represents distance from zero on the number line
        \item Always non-negative: $|x| \geq 0$ for all real numbers $x$
    \end{itemize}
\end{tcolorbox}

\subsection*{Properties}
\begin{itemize}
    \item $|a \cdot b| = |a| \cdot |b|$
    \item $|\frac{a}{b}| = \frac{|a|}{|b|}$ (where $b \neq 0$)
    \item $|a + b| \leq |a| + |b|$ (Triangle Inequality)
    \item $|x| = |-x|$
\end{itemize}

\subsection*{Examples}
\begin{enumerate}
    \item $|-15| = 15$
    \item $|25 - 17| = |8| = 8$
    \item $|x + 5| = 12$ has solutions $x = 7$ and $x = -17$
\end{enumerate}

\section{6.2 Absolute Value Functions}
\subsection*{Key Concepts}
\begin{tcolorbox}[colback=lightgray,colframe=primary,title=Graphing Absolute Value Functions]
    \begin{itemize}
        \item $y = |f(x)|$ reflects any negative parts of $f(x)$ above the $x$-axis
        \item Find $x$-intercepts of $f(x)$ to determine reflection points
        \item Piecewise definition: $y = \begin{cases} f(x), & f(x) \geq 0 \\ -f(x), & f(x) < 0 \end{cases}$
    \end{itemize}
\end{tcolorbox}

\subsection*{Common Transformations}
\begin{itemize}
    \item $y = |x|$: V-shape with vertex at $(0,0)$
    \item $y = |x - h| + k$: V-shape with vertex at $(h,k)$
    \item $y = a|x - h| + k$: V-shape with vertex at $(h,k)$, slope $\pm a$
\end{itemize}

\subsection*{Examples}
\begin{enumerate}
    \item $y = |x^2 - 4|$: Reflects the parabola below $x$-axis upward
    \item $y = |x + 2|$: V-shape with vertex at $(-2,0)$
    \item $y = |x^3 - x|$: Reflects negative parts of cubic function
\end{enumerate}

\section{6.3 Solving Absolute Value Equations}
\subsection*{Key Concepts}
\begin{tcolorbox}[colback=lightgray,colframe=primary,title=Solving Absolute Value Equations]
    \begin{enumerate}
        \item For $|x| = a$, set up two cases: $x = a$ and $x = -a$
        \item For $|f(x)| = a$, solve $f(x) = a$ and $f(x) = -a$
        \item Always check for extraneous solutions
        \item $|x| = -a$ has no solution if $a > 0$
    \end{enumerate}
\end{tcolorbox}

\subsection*{Solution Process}
\begin{enumerate}
    \item Isolate the absolute value expression
    \item Set up two equations (positive and negative cases)
    \item Solve each equation separately
    \item Check all solutions in the original equation
\end{enumerate}

\subsection*{Examples}
\begin{enumerate}
    \item $|x-3| = 7$: $x-3 = 7$ or $x-3 = -7$; $x = 10$ or $x = -4$
    \item $|2x-5| = |x+4|$: Consider all sign combinations; $x = 9$ or $x = \frac{1}{3}$
    \item $|x-1| = x+1$: Only $x = 0$ is valid (check for extraneous roots)
\end{enumerate}

\section{6.4 Reciprocal Functions}
\subsection*{Key Concepts}
\begin{tcolorbox}[colback=lightgray,colframe=primary,title=Reciprocal Functions]
    \begin{itemize}
        \item $y = \frac{1}{f(x)}$ is the reciprocal of $f(x)$
        \item Vertical asymptotes occur where $f(x) = 0$
        \item Invariant points occur where $f(x) = 1$ or $f(x) = -1$
        \item Horizontal asymptote is usually $y = 0$
    \end{itemize}
\end{tcolorbox}

\subsection*{Graphing Process}
\begin{enumerate}
    \item Graph the original function $f(x)$
    \item Mark vertical asymptotes where $f(x) = 0$
    \item Mark invariant points where $f(x) = \pm 1$
    \item Take reciprocals of other $y$-values
    \item Sketch the reciprocal function
\end{enumerate}

\subsection*{Examples}
\begin{enumerate}
    \item $y = \frac{1}{x-2}$: Vertical asymptote at $x = 2$, invariant point at $(3,1)$
    \item $y = \frac{1}{x^2-4}$: Vertical asymptotes at $x = \pm 2$, horizontal asymptote at $y = 0$
    \item $y = \frac{1}{\sqrt{x-3}}$: Domain $x > 3$, vertical asymptote at $x = 3$
\end{enumerate}

\section{6.5 Solving Equations with Two Absolute Values}
\subsection*{Key Concepts}
\begin{tcolorbox}[colback=lightgray,colframe=primary,title=Multiple Absolute Values]
    \begin{itemize}
        \item Consider all possible sign combinations for each absolute value
        \item Use number line to determine valid intervals
        \item Check all solutions for extraneous roots
        \item For $|x-a| + |x-b| = c$, consider 3 cases based on relative positions
    \end{itemize}
\end{tcolorbox}

\subsection*{Solution Strategy}
\begin{enumerate}
    \item Identify critical points (where each absolute value equals zero)
    \item Divide number line into intervals based on critical points
    \item For each interval, determine signs of absolute value expressions
    \item Set up and solve equations for each valid case
    \item Check all solutions in original equation
\end{enumerate}

\subsection*{Examples}
\begin{enumerate}
    \item $|x-2| + |x+6| = 11$: Solutions $x = 3.5$ and $x = -7.5$
    \item $|x-1| + |x+5| = 8$: Solutions $x = 2$ and $x = -6$
    \item $|x+3| - |x-2| = 4$: Solution $x = 1.5$
\end{enumerate}

\section*{Practice Problems}
\subsection*{Basic Absolute Values}
\begin{enumerate}
    \item Evaluate: $|-23| + |17| - |8|$
    \item Solve: $|3x + 7| = 13$
    \item Order from least to greatest: $|-5.2|, |-3.8|, |-7.1|$
\end{enumerate}

\subsection*{Absolute Value Functions}
\begin{enumerate}
    \item Graph $y = |x^2 - 9|$
    \item Write piecewise form for $y = |2x - 6|$
    \item Find domain and range of $y = |x^3 - 4x|$
\end{enumerate}

\subsection*{Solving Equations}
\begin{enumerate}
    \item Solve $|x-4| = 2x + 1$
    \item Solve $|x^2-5x+6| = 2$
    \item Solve $|x+2| + |x-3| = 7$
\end{enumerate}

\subsection*{Reciprocal Functions}
\begin{enumerate}
    \item Find asymptotes of $y = \frac{1}{x^2-16}$
    \item Find invariant points of $y = \frac{1}{x-5}$
    \item Graph $y = \frac{1}{x^2+1}$
\end{enumerate}

\section*{Chapter Summary}
\begin{tcolorbox}[colback=lightgray,colframe=primary,title=Key Takeaways]
    \begin{itemize}
        \item Master the definition and properties of absolute values
        \item Understand how to graph absolute value functions and transformations
        \item Practice solving absolute value equations with proper case analysis
        \item Learn to identify asymptotes and invariant points in reciprocal functions
        \item Develop systematic approach for equations with multiple absolute values
        \item Always check for extraneous solutions and state domains appropriately
    \end{itemize}
\end{tcolorbox}

\section*{Common Mistakes to Avoid}
\begin{tcolorbox}[colback=lightgray,colframe=accent,title=Watch Out For]
    \begin{itemize}
        \item Forgetting to check for extraneous solutions
        \item Not considering all cases when solving absolute value equations
        \item Confusing absolute value functions with reciprocal functions
        \item Missing vertical asymptotes in reciprocal functions
        \item Incorrectly applying absolute value properties
    \end{itemize}
\end{tcolorbox}

\end{document} 