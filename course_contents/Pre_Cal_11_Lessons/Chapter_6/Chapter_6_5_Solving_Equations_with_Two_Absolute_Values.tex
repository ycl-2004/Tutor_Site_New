% !TEX program = xelatex
\documentclass[aspectratio=169]{beamer}
\usepackage{amsmath}
\usepackage{amssymb}
\usepackage{graphicx}
\usepackage{tcolorbox}
\usepackage{booktabs}
\usepackage{colortbl}
\usepackage{xcolor}
\usepackage{tikz}
\usetikzlibrary{intersections,decorations.pathreplacing,angles,quotes}
\usepackage[utf8]{inputenc}

% Custom colors
\definecolor{primary}{RGB}{41, 128, 185}
\definecolor{secondary}{RGB}{52, 152, 219}
\definecolor{accent}{RGB}{231, 76, 60}
\definecolor{lightgray}{RGB}{236, 240, 241}

% Theme customization
\usetheme{Madrid}
\usecolortheme{whale}
\setbeamercolor{structure}{fg=primary}
\setbeamercolor{background canvas}{bg=white}
\setbeamercolor{normal text}{fg=black}

% Title page info
\title{Pre-Calculus 11}
\subtitle{\textbf{6.5 Solving Equations with Two or More Absolute Values}}
\author{Created by Yi-Chen Lin}
\date{\today}

\begin{document}

% Title Page
\begin{frame}
    \titlepage
    \vfill
\end{frame}

% Overview
\begin{frame}{Overview 概述}
    \begin{tcolorbox}[colback=lightgray,colframe=primary,title=What We Will Learn 学习目标]
        \footnotesize
        \begin{enumerate}
            \item Understand how to solve equations with two or more absolute values
            \item Learn to set up and solve all possible cases
            \item Check solutions for extraneous roots
            \item Use graphical methods to verify solutions
        \end{enumerate}
    \end{tcolorbox}
    \vspace{1em}
    \textbf{Key Questions:}
    \begin{itemize}
        \item How do we set up cases for multiple absolute values?
        \item Why do we need to check for extraneous solutions?
        \item How can graphs help us solve these equations?
    \end{itemize}
\end{frame}

% Concept
\begin{frame}{Solving Equations with Two Absolute Values 含两个绝对值的方程}
    \begin{tcolorbox}[colback=lightgray,colframe=primary,title=Key Idea 关键思路]
        \footnotesize
        When an equation contains two absolute values, consider all possible sign combinations (cases) for each absolute value expression.
        \par
        含有两个绝对值的方程,需要考虑每个绝对值表达式为正或负的所有组合(共4种情况)。
    \end{tcolorbox}
    \vspace{1em}
    \textbf{General Steps:}
    \begin{enumerate}
        \item Isolate absolute value expressions if possible
        \item Set up all possible cases (both positive/negative)
        \item Solve each case for $x$
        \item Check each solution in the original equation (extraneous roots)
    \end{enumerate}
\end{frame}

% Example 1: Step-by-step
\begin{frame}{Example: $|x-2| + |x+6| = 11$}
    \footnotesize
    Since there are two absolute values, consider all four cases:
    \begin{itemize}
        \item Case 1: $x-2 \geq 0$, $x+6 \geq 0$ ($x \geq -6$)
        \item Case 2: $x-2 \geq 0$, $x+6 < 0$ ($2 \leq x < -6$) (impossible)
        \item Case 3: $x-2 < 0$, $x+6 \geq 0$ ($-6 \leq x < 2$)
        \item Case 4: $x-2 < 0$, $x+6 < 0$ ($x < -6$)
    \end{itemize}
    \vspace{1em}
    \textbf{Set up equations for each valid case:}
    \begin{itemize}
        \item Case 1: $(x-2) + (x+6) = 11$
        \item Case 3: $-(x-2) + (x+6) = 11$
        \item Case 4: $-(x-2) - (x+6) = 11$
    \end{itemize}
\end{frame}

% Example 1: Solution (Page 1)
\begin{frame}{Example Solution: $|x-2| + |x+6| = 11$ (Part 1)}
    \footnotesize
    \textbf{Case 1:} $x \geq -6$
    \begin{align*}
        (x-2) + (x+6) &= 11 \\
        2x + 4 &= 11 \\
        2x &= 7 \\
        x &= 3.5
    \end{align*}
    Check: $|3.5-2| + |3.5+6| = 1.5 + 9.5 = 11$ (Valid)
    \vspace{1em}
    \textbf{Case 3:} $-6 \leq x < 2$
    \begin{align*}
        -(x-2) + (x+6) &= 11 \\
        -x+2+x+6 &= 11 \\
        8 &= 11
    \end{align*}
    No solution in this case.
\end{frame}

% Example 1: Solution (Page 2)
\begin{frame}{Example Solution: $|x-2| + |x+6| = 11$ (Part 2)}
    \footnotesize
    \textbf{Case 4:} $x < -6$
    \begin{align*}
        -(x-2) - (x+6) &= 11 \\
        -x+2-x-6 &= 11 \\
        -2x-4 &= 11 \\
        -2x &= 15 \\
        x &= -7.5
    \end{align*}
    Check: $|-7.5-2| + |-7.5+6| = |-9.5| + |-1.5| = 9.5 + 1.5 = 11$ (Valid)
\end{frame}

% Example 1: Graphical Solution
\begin{frame}{Graphical Solution 图像解法}
    \footnotesize
    \textbf{Graph:} $y_1 = |x-2| + |x+6|$, $y_2 = 11$
    \begin{center}
        \includegraphics[width=0.8\textwidth]{absval_2_graph1.png}
    \end{center}
    \textbf{Intersections:} $x = -7.5$, $x = 3.5$
\end{frame}

% Example 1: Enlarged Graph
\begin{frame}{Graphical Solution (Enlarged)}
    \begin{center}
        \includegraphics[width=1.0\textwidth]{absval_2_graph1.png}
    \end{center}
\end{frame}

% Practice Problem 1
\begin{frame}{Practice: $|x-1| + |x+5| = 8$}
    \footnotesize
    \textbf{Try:} Solve $|x-1| + |x+5| = 8$ by considering all cases.\par
    \textbf{(Solutions on next slide)}
\end{frame}

% Practice Problem 1 Solution (Page 1)
\begin{frame}{Solution: $|x-1| + |x+5| = 8$ (Part 1)}
    \footnotesize
    \textbf{Case 1:} $x \geq -5$
    \begin{align*}
        (x-1) + (x+5) &= 8 \\
        2x + 4 &= 8 \\
        2x &= 4 \\
        x &= 2
    \end{align*}
    Check: $|2-1| + |2+5| = 1 + 7 = 8$ (Valid)
    \vspace{1em}
    \textbf{Case 3:} $-5 \leq x < 1$
    \begin{align*}
        -(x-1) + (x+5) &= 8 \\
        -x+1+x+5 &= 8 \\
        6 &= 8
    \end{align*}
    No solution in this case.
\end{frame}

% Practice Problem 1 Solution (Page 2)
\begin{frame}{Solution: $|x-1| + |x+5| = 8$ (Part 2)}
    \footnotesize
    \textbf{Case 4:} $x < -5$
    \begin{align*}
        -(x-1) - (x+5) &= 8 \\
        -x+1-x-5 &= 8 \\
        -2x-4 &= 8 \\
        -2x &= 12 \\
        x &= -6
    \end{align*}
    Check: $|-6-1| + |-6+5| = 7 + 1 = 8$ (Valid)
\end{frame}

% Practice Problem 1: Graph
\begin{frame}{Graphical Solution}
    \footnotesize
    \textbf{Graph:} $y_1 = |x-1| + |x+5|$, $y_2 = 8$
    \begin{center}
        \includegraphics[width=0.8\textwidth]{absval_2_graph2.png}
    \end{center}
    \textbf{Intersections:} $x = -6$, $x = 2$
\end{frame}

% Practice Problem 1: Enlarged Graph
\begin{frame}{Graphical Solution (Enlarged)}
    \begin{center}
        \includegraphics[width=1.0\textwidth]{absval_2_graph2.png}
    \end{center}
\end{frame}

% Practice Problem 2
\begin{frame}{Practice: $|x-3| + |x+4| = 9$}
    \footnotesize
    \textbf{Try:} Solve $|x-3| + |x+4| = 9$ by considering all cases.\par
    \textbf{(Solutions on next slide)}
\end{frame}

% Practice Problem 2 Solution (Page 1)
\begin{frame}{Solution: $|x-3| + |x+4| = 9$ (Part 1)}
    \footnotesize
    \textbf{Case 1:} $x \geq -4$
    \begin{align*}
        (x-3) + (x+4) &= 9 \\
        2x + 1 &= 9 \\
        2x &= 8 \\
        x &= 4
    \end{align*}
    Check: $|4-3| + |4+4| = 1 + 8 = 9$ (Valid)
    \vspace{1em}
    \textbf{Case 3:} $-4 \leq x < 3$
    \begin{align*}
        -(x-3) + (x+4) &= 9 \\
        -x+3+x+4 &= 9 \\
        7 &= 9
    \end{align*}
    No solution in this case.
\end{frame}

% Practice Problem 2 Solution (Page 2)
\begin{frame}{Solution: $|x-3| + |x+4| = 9$ (Part 2)}
    \footnotesize
    \textbf{Case 4:} $x < -4$
    \begin{align*}
        -(x-3) - (x+4) &= 9 \\
        -x+3-x-4 &= 9 \\
        -2x-1 &= 9 \\
        -2x &= 10 \\
        x &= -5
    \end{align*}
    Check: $|-5-3| + |-5+4| = 8 + 1 = 9$ (Valid)
\end{frame}

% Practice Problem 2: Graph
\begin{frame}{Graphical Solution}
    \footnotesize
    \textbf{Graph:} $y_1 = |x-3| + |x+4|$, $y_2 = 9$
    \begin{center}
        \includegraphics[width=0.8\textwidth]{absval_2_graph3.png}
    \end{center}
    \textbf{Intersections:} $x = -5$, $x = 4$
\end{frame}

% Practice Problem 2: Enlarged Graph
\begin{frame}{Graphical Solution (Enlarged)}
    \begin{center}
        \includegraphics[width=1.0\textwidth]{absval_2_graph3.png}
    \end{center}
\end{frame}

% Summary
\begin{frame}{Summary 总结}
    \begin{tcolorbox}[colback=lightgray,colframe=primary,title=Key Points 要点]
        \footnotesize
        \begin{enumerate}
            \item For equations with two or more absolute values, always consider all possible sign cases
            \item Solve each case and check for extraneous solutions
            \item Use graphs to visualize and confirm solutions
            \item Solutions must satisfy the original equation and the domain of each case
        \end{enumerate}
    \end{tcolorbox}
\end{frame}

% --- New Practice Problems ---
% Practice 3
\begin{frame}{Practice: $|x-7| + |x+2| = 13$}
    \footnotesize
    Solve $|x-7| + |x+2| = 13$ by considering all cases.\par
    \textbf{(Solution on next slide)}
\end{frame}

% Practice 3 Solution (Page 1)
\begin{frame}{Solution: $|x-7| + |x+2| = 13$ (Part 1)}
    \footnotesize
    \textbf{Case 1:} $x \geq 7$
    \begin{align*}
        (x-7) + (x+2) &= 13 \\
        2x - 5 &= 13 \\
        2x &= 18 \\
        x &= 9
    \end{align*}
    Check: $|9-7| + |9+2| = 2 + 11 = 13$ (Valid)
    \vspace{1em}
    \textbf{Case 3:} $-2 \leq x < 7$
    \begin{align*}
        -(x-7) + (x+2) &= 13 \\
        -x+7+x+2 &= 13 \\
        9 &= 13
    \end{align*}
    No solution in this case.
\end{frame}

% Practice 3 Solution (Page 2)
\begin{frame}{Solution: $|x-7| + |x+2| = 13$ (Part 2)}
    \footnotesize
    \textbf{Case 4:} $x < -2$
    \begin{align*}
        -(x-7) - (x+2) &= 13 \\
        -x+7-x-2 &= 13 \\
        -2x+5 &= 13 \\
        -2x &= 8 \\
        x &= -4
    \end{align*}
    Check: $|-4-7| + |-4+2| = 11 + 2 = 13$ (Valid)
\end{frame}

% Practice 4
\begin{frame}{Practice: $|x-6| + |x+1| = 10$}
    \footnotesize
    Solve $|x-6| + |x+1| = 10$ by considering all cases.\par
    \textbf{(Solution on next slide)}
\end{frame}

% Practice 4 Solution (Page 1)
\begin{frame}{Solution: $|x-6| + |x+1| = 10$ (Part 1)}
    \footnotesize
    \textbf{Case 1:} $x \geq 6$
    \begin{align*}
        (x-6) + (x+1) &= 10 \\
        2x - 5 &= 10 \\
        2x &= 15 \\
        x &= 7.5
    \end{align*}
    Check: $|7.5-6| + |7.5+1| = 1.5 + 8.5 = 10$ (Valid)
    \vspace{1em}
    \textbf{Case 3:} $-1 \leq x < 6$
    \begin{align*}
        -(x-6) + (x+1) &= 10 \\
        -x+6+x+1 &= 10 \\
        7 &= 10
    \end{align*}
    No solution in this case.
\end{frame}

% Practice 4 Solution (Page 2)
\begin{frame}{Solution: $|x-6| + |x+1| = 10$ (Part 2)}
    \footnotesize
    \textbf{Case 4:} $x < -1$
    \begin{align*}
        -(x-6) - (x+1) &= 10 \\
        -x+6-x-1 &= 10 \\
        -2x+5 &= 10 \\
        -2x &= 5 \\
        x &= -2.5
    \end{align*}
    Check: $|-2.5-6| + |-2.5+1| = 8.5 + 1.5 = 10$ (Valid)
\end{frame}

% Practice 5
\begin{frame}{Practice: $|x-8| + |x+3| = 15$}
    \footnotesize
    Solve $|x-8| + |x+3| = 15$ by considering all cases.\par
    \textbf{(Solution on next slide)}
\end{frame}

% Practice 5 Solution (Page 1)
\begin{frame}{Solution: $|x-8| + |x+3| = 15$ (Part 1)}
    \footnotesize
    \textbf{Case 1:} $x \geq 8$
    \begin{align*}
        (x-8) + (x+3) &= 15 \\
        2x - 5 &= 15 \\
        2x &= 20 \\
        x &= 10
    \end{align*}
    Check: $|10-8| + |10+3| = 2 + 13 = 15$ (Valid)
    \vspace{1em}
    \textbf{Case 3:} $-3 \leq x < 8$
    \begin{align*}
        -(x-8) + (x+3) &= 15 \\
        -x+8+x+3 &= 15 \\
        11 &= 15
    \end{align*}
    No solution in this case.
\end{frame}

% Practice 5 Solution (Page 2)
\begin{frame}{Solution: $|x-8| + |x+3| = 15$ (Part 2)}
    \footnotesize
    \textbf{Case 4:} $x < -3$
    \begin{align*}
        -(x-8) - (x+3) &= 15 \\
        -x+8-x-3 &= 15 \\
        -2x+5 &= 15 \\
        -2x &= 10 \\
        x &= -5
    \end{align*}
    Check: $|-5-8| + |-5+3| = 13 + 2 = 15$ (Valid)
\end{frame}

\end{document} 