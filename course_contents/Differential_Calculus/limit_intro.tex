\documentclass[aspectratio=169]{beamer}
\usetheme{Madrid}
\usecolortheme{default}
\usepackage{tikz}

% Custom colors
\definecolor{myblue}{RGB}{0,114,178}
\definecolor{myorange}{RGB}{230,159,0}
\definecolor{mygreen}{RGB}{0,158,115}
\definecolor{myred}{RGB}{213,94,0}
\definecolor{mypurple}{RGB}{204,121,167}
\definecolor{mybrown}{RGB}{240,228,66}
\definecolor{mygray}{RGB}{189,189,189}

% Set colors
\setbeamercolor{title}{fg=white,bg=myblue}
\setbeamercolor{frametitle}{fg=white,bg=myblue}
\setbeamercolor{section title}{fg=white,bg=myblue}
\setbeamercolor{subsection title}{fg=white,bg=myorange}
\setbeamercolor{item}{fg=myblue}
\setbeamercolor{subitem}{fg=myorange}

% Remove navigation symbols
\setbeamertemplate{navigation symbols}{}

% Custom commands
\newcommand{\limx}[2]{\lim_{x \to #1} #2}
\newcommand{\limxa}[1]{\lim_{x \to a} #1}
\newcommand{\limxap}[1]{\lim_{x \to a^+} #1}
\newcommand{\limxam}[1]{\lim_{x \to a^-} #1}

% Title page
\title{1.3 The Limit of a Function}
\subtitle{Introduction to Limits}
\author{Differential Calculus}
\date{}

\begin{document}

\begin{frame}
\titlepage
\end{frame}

% Table of Contents
\begin{frame}{Outline}
\tableofcontents
\end{frame}

\section{Notation and Basic Concepts}

\begin{frame}{Limit Notation}
\frametitle{1.3 The Limit of a Function}

\begin{block}{Limit Notation}
We write:
$$\limx{a}{f(x)} = L$$

This should be read as:\\
\textbf{The limit of $f(x)$ as $x$ approaches $a$ is $L$.}
\end{block}

\begin{itemize}
\item This is shorthand notation to avoid writing long sentences
\item Mathematically precise and language-independent
\item Can also be written as: $f(x) \to L$ as $x \to a$
\item Both notations mean exactly the same thing
\end{itemize}

\end{frame}

\begin{frame}{Example 1.3.2: Piecewise Function}
\frametitle{Understanding Limits with a Simple Example}

\begin{block}{Consider the function:}
$$f(x) = \begin{cases}
2x & \text{if } x < 3 \\
9 & \text{if } x = 3 \\
2x & \text{if } x > 3
\end{cases}$$
\end{block}

\begin{center}
\begin{tikzpicture}[scale=0.3]
\draw[->] (-1,0) -- (6,0) node[right] {$x$};
\draw[->] (0,-1) -- (0,10) node[above] {$y$};
\draw[grid=both,color=mygray!30] (-1,-1) grid (6,10);

% Draw the function
\draw[color=myblue,thick] (-0.5,-1) -- (2.9,5.8);
\draw[color=myblue,thick] (3.1,6.2) -- (5.5,11);

% Open circle at (3,6)
\draw[color=myblue,thick,fill=white] (3,6) circle (0.1);

% Closed circle at (3,9)
\draw[color=myred,thick,fill=myred] (3,9) circle (0.1);

% Labels

\node at (3,-0.5) {3};
\node at (-0.5,6) {6};
\node at (-0.5,9) {9};
\end{tikzpicture}
\end{center}

\end{frame}

\begin{frame}{Analyzing the Function Near x = 3}
\frametitle{What happens as x approaches 3?}

\begin{block}{Let's plug in values close to 3:}
\begin{center}
\begin{tabular}{|c|c|c|c|c|c|c|}
\hline
$x$ & 2.9 & 2.99 & 2.999 & 3.001 & 3.01 & 3.1 \\
\hline
$f(x)$ & 5.8 & 5.98 & 5.998 & 6.002 & 6.02 & 6.2 \\
\hline
\end{tabular}
\end{center}
\end{block}

\begin{itemize}
\item As $x$ gets closer to 3, $f(x)$ gets closer to 6
\item We write: $\limx{3}{f(x)} = 6$
\item Note: $f(3) = 9$, but the limit is 6
\item The limit does NOT depend on the value at $x = 3$
\end{itemize}

\end{frame}

\section{Informal Definition}

\begin{frame}{Informal Definition of Limit}
\frametitle{Definition 1.3.3}

\begin{block}{Informal Definition}
We write $\limx{a}{f(x)} = L$ if the value of the function $f(x)$ is sure to be arbitrarily close to $L$ whenever the value of $x$ is close enough to $a$, \textbf{without being exactly $a$}.
\end{block}

\begin{itemize}
\item This is an informal definition sufficient for most purposes
\item The condition "without being exactly $a$" is important
\item We'll see why this matters when we study derivatives
\item For now, this gives us a working understanding of limits
\end{itemize}

\end{frame}

\section{More Examples}

\begin{frame}{Example 1.3.4: Rational Function}
\frametitle{Computing a Limit}

\begin{block}{Consider:}
$$\limx{2}{\frac{x-2}{x^2+x-6}}$$
\end{block}

\begin{itemize}
\item If we try to compute $f(2)$, we get $\frac{0}{0}$ which is \textbf{undefined}
\item This is exactly why we need limits!
\item We must "sneak up" on points where functions are not defined
\end{itemize}

\begin{block}{Important:}
$\frac{0}{0}$ is \textbf{not} $\infty$ and it is \textbf{not} 1. It is \textbf{undefined}.
\end{block}

\end{frame}

\begin{frame}{Example 1.3.4: Numerical Approach}
\frametitle{Let's plug in values close to 2}

\begin{block}{Numerical Analysis:}
\begin{center}
\begin{tabular}{|c|c|c|c|c|c|c|}
\hline
$x$ & 1.9 & 1.99 & 1.999 & 2.001 & 2.01 & 2.1 \\
\hline
$f(x)$ & 0.20408 & 0.20040 & 0.20004 & 0.19996 & 0.19960 & 0.19608 \\
\hline
\end{tabular}
\end{center}
\end{block}

\begin{itemize}
\item As $x$ approaches 2, $f(x)$ approaches 0.2
\item Therefore: $\limx{2}{\frac{x-2}{x^2+x-6}} = 0.2$
\item The limit exists even though the function is not defined at $x = 2$
\end{itemize}

\end{frame}

\section{When Limits Don't Exist}

\begin{frame}{Example 1.3.5: Oscillating Function}
\frametitle{When Limits Don't Exist - Case 1}

\begin{block}{Consider:}
$$\limx{0}{\sin\left(\frac{\pi}{x}\right)}$$
\end{block}

\begin{center}
\begin{tikzpicture}[scale=0.6]
\draw[->] (-3,0) -- (3,0) node[right] {$x$};
\draw[->] (0,-2) -- (0,2) node[above] {$y$};
\draw[grid=both,color=mygray!30] (-3,-2) grid (3,2);

% Draw oscillating function
\draw[color=myblue,thick,domain=-2.5:-0.1,samples=200] plot (\x,{sin(3.14159/\x r)});
\draw[color=myblue,thick,domain=0.1:2.5,samples=200] plot (\x,{sin(3.14159/\x r)});

\node at (0,-2.5) {Oscillates faster and faster as $x \to 0$};
\end{tikzpicture}
\end{center}

\begin{itemize}
\item As $x \to 0$, $\frac{\pi}{x}$ becomes larger and larger
\item $\sin$ oscillates faster and faster
\item Function doesn't approach a single number
\item Therefore: $\limx{0}{\sin\left(\frac{\pi}{x}\right)} = \text{DNE}$
\end{itemize}

\end{frame}

\begin{frame}{Example 1.3.6: Jump Discontinuity}
\frametitle{When Limits Don't Exist - Case 2}

\begin{block}{Consider:}
$$f(x) = \begin{cases}
x & \text{if } x < 2 \\
-1 & \text{if } x = 2 \\
x + 3 & \text{if } x > 2
\end{cases}$$
\end{block}

\begin{center}
\begin{tikzpicture}[scale=0.4]
\draw[->] (-1,0) -- (6,0) node[right] {$x$};
\draw[->] (0,-2) -- (0,6) node[above] {$y$};
\draw[grid=both,color=mygray!30] (-1,-2) grid (6,6);

% Draw the function
\draw[color=myblue,thick] (-0.5,-0.5) -- (1.9,1.9);
\draw[color=myblue,thick] (2.1,5.1) -- (5.5,8.5);

% Point at x=2
\draw[color=myred,thick,fill=myred] (2,-1) circle (0.1);

% Open circles
\draw[color=myblue,thick,fill=white] (2,2) circle (0.1);
\draw[color=myblue,thick,fill=white] (2,5) circle (0.1);

\node at (2,-1.5) {$(2,-1)$};
\node at (2,2.5) {$(2,2)$};
\node at (2,5.5) {$(2,5)$};
\end{tikzpicture}
\end{center}

\end{frame}

\begin{frame}{Example 1.3.6: Numerical Analysis}
\frametitle{Approaching from Different Sides}

\begin{block}{Let's plug in values close to 2:}
\begin{center}
\begin{tabular}{|c|c|c|c|c|c|c|}
\hline
$x$ & 1.9 & 1.99 & 1.999 & 2.001 & 2.01 & 2.1 \\
\hline
$f(x)$ & 1.9 & 1.99 & 1.999 & 5.001 & 5.01 & 5.1 \\
\hline
\end{tabular}
\end{center}
\end{block}

\begin{itemize}
\item From below: $f(x) \to 2$
\item From above: $f(x) \to 5$
\item Since we get different values, the limit does not exist
\item $\limx{2}{f(x)} = \text{DNE}$
\end{itemize}

\end{frame}

\section{One-Sided Limits}

\begin{frame}{One-Sided Limits}
\frametitle{Definition 1.3.7}

\begin{block}{Left-Hand Limit:}
$$\limxam{f(x)} = K$$
When $f(x)$ gets closer to $K$ as $x < a$ approaches $a$ from below.
\end{block}

\begin{block}{Right-Hand Limit:}
$$\limxap{f(x)} = L$$
When $f(x)$ gets closer to $L$ as $x > a$ approaches $a$ from above.
\end{block}

\begin{itemize}
\item Also called "left-hand" and "right-hand" limits
\item Be careful to include the superscript $+$ and $-$ 
\item Alternative notations exist but we'll use the standard ones
\end{itemize}

\end{frame}

\begin{frame}{Relationship Between Limits}
\frametitle{Theorem 1.3.8}

\begin{block}{Important Theorem:}
$$\limxa{f(x)} = L \text{ if and only if } \limxam{f(x)} = L \text{ and } \limxap{f(x)} = L$$
\end{block}

\begin{itemize}
\item The two-sided limit exists only if both one-sided limits exist and are equal
\item If either one-sided limit doesn't exist, or if they're different, then the two-sided limit doesn't exist
\item This gives us a systematic way to check if limits exist
\end{itemize}

\end{frame}

\begin{frame}{Two Functions - Which Limits Exist?}
\frametitle{Function $f(x)$}

\begin{center}
\begin{tikzpicture}[scale=0.6]
\draw[->] (-1,0) -- (3,0) node[right] {$x$};
\draw[->] (0,0) -- (0,4) node[above] {$y$};
\draw[grid=both,color=mygray!30] (-1,0) grid (3,4);
\draw[color=myblue,thick] (-0.5,1) -- (0.9,1.9);
\draw[color=myblue,thick] (1.1,2.1) -- (2.5,3);
\draw[color=myred,thick,fill=white] (1,2) circle (0.1);
\node at (1,2.5) {$(1,2)$};
\end{tikzpicture}
\end{center}

\textbf{Function $f(x)$:}
\begin{align*}
\lim_{x\to 1^-} f(x) &= 2 \\
\lim_{x\to 1^+} f(x) &= 2 \\
\therefore \lim_{x\to 1} f(x) &= 2
\end{align*}

\end{frame}

\begin{frame}{Two Functions - Which Limits Exist?}
\frametitle{Function $g(t)$}

\begin{center}
\begin{tikzpicture}[scale=0.6]
\draw[->] (-1,0) -- (3,0) node[right] {$t$};
\draw[->] (0,-3) -- (0,3) node[above] {$y$};
\draw[grid=both,color=mygray!30] (-1,-3) grid (3,3);
\draw[color=myblue,thick] (-0.5,1) -- (0.9,1.9);
\draw[color=myblue,thick] (1.1,-2.1) -- (2.5,-3);
\draw[color=myred,thick,fill=white] (1,2) circle (0.1);
\draw[color=myred,thick,fill=white] (1,-2) circle (0.1);
\node at (1,2.5) {$(1,2)$};
\node at (1,-2.5) {$(1,-2)$};
\end{tikzpicture}
\end{center}

\textbf{Function $g(t)$:}
\begin{align*}
\lim_{t\to 1^-} g(t) &= 2 \\
\lim_{t\to 1^+} g(t) &= -2 \\
\therefore \lim_{t\to 1} g(t) &= \text{DNE}
\end{align*}

\end{frame}

\section{Infinite Limits}

\begin{frame}{Definition 1.3.10: Positive Infinity}
\begin{block}{Positive Infinity:}
$$\lim_{x\to a} f(x) = +\infty$$
When $f(x)$ becomes arbitrarily large and positive as $x$ approaches $a$.
\end{block}
\end{frame}

\begin{frame}{Definition 1.3.10: Negative Infinity}
\begin{block}{Negative Infinity:}
$$\lim_{x\to a} f(x) = -\infty$$
When $f(x)$ becomes arbitrarily large and negative as $x$ approaches $a$.
\end{block}
\end{frame}

\begin{frame}{Graph: Infinite Limits}
\begin{center}
\begin{tikzpicture}[scale=0.7]
  \draw[->] (-2,0) -- (2,0) node[right] {$x$};
  \draw[->] (0,0) -- (0,4) node[above] {$y$};
  \draw[grid=both,color=mygray!30] (-2,0) grid (2,4);
  \draw[color=myblue,thick,domain=-1.9:-0.1,samples=100] plot (\x,{1/(\x*\x)});
  \draw[color=myblue,thick,domain=0.1:1.9,samples=100] plot (\x,{1/(\x*\x)});
  \node at (0,4.3) {$f(x) = \frac{1}{x^2}$};
\end{tikzpicture}

\vspace{1em}
$\displaystyle \lim_{x\to 0} \frac{1}{x^2} = +\infty$
\end{center}
\end{frame}

\begin{frame}{One-Sided Infinite Limits}
\frametitle{Definition 1.3.11}

\begin{block}{Right-Hand Infinite Limits:}
$$\limxap{f(x)} = +\infty \text{ or } \limxap{f(x)} = -\infty$$
\end{block}

\begin{block}{Left-Hand Infinite Limits:}
$$\limxam{f(x)} = +\infty \text{ or } \limxam{f(x)} = -\infty$$
\end{block}

\begin{itemize}
\item These occur when functions approach infinity from only one side
\item Very common in rational functions and trigonometric functions
\item Important for understanding vertical asymptotes
\end{itemize}

\end{frame}

\begin{frame}{Example 1.3.12: Trigonometric Function}
\frametitle{One-Sided Infinite Limits}

\begin{block}{Consider:}
$$g(x) = \frac{1}{\sin(x)}$$
Find the one-sided limits as $x \to \pi$.
\end{block}

\begin{center}
\begin{tikzpicture}[scale=0.4]
\draw[->] (0,0) -- (7,0) node[right] {$x$};
\draw[->] (0,-4) -- (0,4) node[above] {$y$};
\draw[grid=both,color=mygray!30] (0,-4) grid (7,4);

% Draw sin(x)
\draw[color=myorange,thick,domain=0:6.28,samples=100] plot (\x,{sin(\x r)});
\node at (6.5,0.5) {$\sin(x)$};

% Draw 1/sin(x) with asymptotes
\draw[color=myblue,thick,domain=0.1:3.04,samples=100] plot (\x,{1/sin(\x r)});
\draw[color=myblue,thick,domain=3.24:6.18,samples=100] plot (\x,{1/sin(\x r)});

% Vertical asymptotes
\draw[color=myred,dashed] (3.14159,-4) -- (3.14159,4);
\draw[color=myred,dashed] (6.28318,-4) -- (6.28318,4);

\node at (3.14159,-4.5) {$\pi$};
\node at (6.28318,-4.5) {$2\pi$};
\end{tikzpicture}
\end{center}

\end{frame}

\begin{frame}{Understanding the Behavior: Analysis}
\begin{block}{Analysis:}
\begin{itemize}
  \item As $x \to \pi^-$: $\sin(x) \to 0^+$ (small positive numbers)
  \item Therefore: $\dfrac{1}{\sin(x)} \to +\infty$
  \item As $x \to \pi^+$: $\sin(x) \to 0^-$ (small negative numbers)
  \item Therefore: $\dfrac{1}{\sin(x)} \to -\infty$
\end{itemize}
\end{block}
\end{frame}

\begin{frame}{Understanding the Behavior: Result}
\begin{block}{Result:}
\[
\lim_{x \to \pi^-} \frac{1}{\sin(x)} = +\infty \\
\lim_{x \to \pi^+} \frac{1}{\sin(x)} = -\infty
\]
\end{block}

\begin{itemize}
  \item Since the one-sided limits are different, $\lim_{x \to \pi} \frac{1}{\sin(x)} = \text{DNE}$
  \item But the one-sided infinite limits give us more information than just "DNE"
\end{itemize}
\end{frame}

\section{Summary}

\begin{frame}{Summary: Types of Limits}
\frametitle{What We've Learned}

\begin{block}{Limits That Exist:}
\begin{itemize}
\item $\limxa{f(x)} = L$ where $L$ is a finite number
\item Both one-sided limits exist and are equal
\end{itemize}
\end{block}

\begin{block}{Limits That Don't Exist:}
\begin{enumerate}
\item \textbf{Oscillation}: Function oscillates wildly (like $\sin(\frac{\pi}{x})$)
\item \textbf{Jump}: One-sided limits are different
\item \textbf{Infinite}: Function approaches $\pm\infty$
\end{enumerate}
\end{block}

\begin{block}{One-Sided Limits:}
\begin{itemize}
\item $\limxam{f(x)} = L$ (left-hand limit)
\item $\limxap{f(x)} = L$ (right-hand limit)
\item Can be finite numbers or $\pm\infty$
\end{itemize}
\end{block}

\end{frame}

\begin{frame}{Key Takeaways}
\frametitle{Important Concepts}

\begin{block}{Notation:}
$$\limxa{f(x)} = L \quad \text{vs} \quad f(a) = L$$
\end{block}

\begin{block}{Critical Points:}
\begin{itemize}
\item Limits describe behavior \textbf{near} a point, not \textbf{at} the point
\item A function can have a limit at $x = a$ even if $f(a)$ is undefined
\item Infinity is not a number - it's a description of behavior
\item One-sided limits help us understand discontinuities
\end{itemize}
\end{block}

\begin{block}{Next Steps:}
We'll develop systematic methods for computing limits using limit laws and algebraic techniques.
\end{block}

\end{frame}

\end{document} 